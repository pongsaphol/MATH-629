\chapter{Product measure}
\section{Introduction}

Given two (or finitely many) measure space $(X_1, \SM_1, \mu_1),(X_2, \SM_2, \mu_2)$
$X = X_1 \times X_2$, On $X$ there is a natural $\sigma$-algebra $\SM = \SM_1 \otimes \SM_2 =:$ $\sigma$-algebra generated by set of the form.
$E_1 \times E_2, E_1 \in \SM_1, E_2 \in \SM_2$ and 
$\mu(E_1 \times E_2) = \mu_1(E_1) \mu_2(E_2)$


Extend the set function $\mu$, defined on the general rectangles to all of $\SM_1 \otimes \SM_2$.

\begin{claim}
  The rectangles form a semiring (use cartesian product $\SM_1 \times \SM_2$ of semirings are semiring)
\end{claim}

\begin{theorem}
  Given $(X_1, \SM_1, \mu_1), (X_2, \SM_2, \mu_2)$, $\sigma$-finite measure spaces, there exists a unique measure $\mu$ on $\SM_1 \otimes \SM_2$ such that $\mu(E_1 \times E_2) = \mu_1(E_1) \mu_2(E_2)$ for all $E_1 \in \SM_1, E_2 \in \SM_2$
\end{theorem}

\begin{proof}
  $\SSS_{\text{rect}} = \SM_1 \times \SM_2$ semiring of rectangles.
  Need to show that $\nu: \SSS_{\text{rect}} \to [0, \infty)$ is a premeasure
  verify $\sigma$-additivity on $\SSS_{\text{rect}}$ If 
  $$E_1 \times E_2 = \biguplus_{j=1}^\infty E_{j, 1} \times E_{j, 2}$$then
  $$\nu(E_1 \times E_2) = \sum_{j=1}^\infty \nu(E_{j, 1} \times E_{j, 2})$$
  \[\mathbbm{1}_{E_1 \times E_2}(x_1, x_2) = \sum_{j=1}^\infty \mathbbm{1}_{E_{j, 1} \times E_{j, 2}}(x_1, x_2)\iff \mathbbm{1}_{E_1}(x_1) \mathbbm{1}_{E_2}(x_2) = \sum_{j=1}^\infty \mathbbm{1}_{E_{j, 1}}(x_1) \mathbbm{1}_{E_{j, 2}}(x_2)\]
  \begin{align*}
    \mathbbm{1}_{E_1}(x_1) \mathbbm{1}_{E_2}(x_2) &= \sum_{j=1}^\infty \mathbbm{1}_{E_{j, 1}}(x_1) \mathbbm{1}_{E_{j, 2}}(x_2) \\
    \int \mathbbm{1}_{E_1}(x_1) \mathbbm{1}_{E_2}(x_2)\ \mathrm{d}\mu_2 &= \int \sum_{j=1}^\infty \mathbbm{1}_{E_{j, 1}}(x_1) \mathbbm{1}_{E_{j, 2}}(x_2) \ \mathrm{d}\mu_2\\
    \mathbbm{1}_{E_1}(x_1)\mu_2(E_2) &= \sum_{j=1}^\infty\int \mathbbm{1}_{E_{j, 1}}(x_1) \mathbbm{1}_{E_{j, 2}}(x_2) \ \mathrm{d}\mu_2\\
    &= \sum_{j=1}^\infty \mathbbm{1}_{E_{j, 1}}(x_1) \mu_2(E_{j, 2}) \\
    \int \mathbbm{1}_{E_1}(x_1)\mu_2(E_2) \ \mathrm{d}\mu_1 &= \int \sum_{j=1}^\infty \mathbbm{1}_{E_{j, 1}}(x_1) \mu_2(E_{j, 2}) \ \mathrm{d}\mu_1\\
    \mu_1(E_1)\mu_2(E_2) &= \sum_{j=1}^\infty \mu_1(E_{j, 1})\mu_2(E_{j, 2}) \\
    \nu(E_1 \times E_2) &= \sum_{j=1}^\infty \nu(E_{j, 1} \times E_{j, 2})
  \end{align*}
  % \[\mathbbm{1}_{E_1}(x_1) \mu_2(E_2) = \int \sum_{j=1}^\infty \mathbbm{1}_{E_{j, 1}}(x_1) \mathbbm{1}_{E_{j, 2}}(x_2)\ \mathrm{d}\mu_2\ = \sum_{j}\mathbbm{1}_{E_{j, 1}}(x_1)\mu_2(E_{j, 2})\]
  % By Monotone Convergence Theorem then Integrate in $x_1$
  % \[ \nu(E_1 \times E_2) = \mu_1(E_1)\mu_2(E_2) = \int\sum_j \mathbbm{1}_{E_{j, 1}}\mu_2(E_{j, 2}) \ \mathrm{d}\mu_1 = \sum_j \mu_1(E_{j, 1})\mu_2(E_{j, 2}) = \sum_j \nu(E_{j, 1}\times E_{j, 2})\]
\end{proof}

\begin{lemma}
  If $\mu_1$ is $\sigma$-finite and $\mu_2$ is $\sigma$-finite, then $\nu$ is $\sigma$-finite.
\end{lemma}
\begin{proof}
  $X_1 = \biguplus_{j=1}^\infty X_{1, j}$, $X_2 = \biguplus_{k=1}^\infty X_{2, k}$, $\mu_1(X_{1, j}) < \infty$, $\mu_2(X_{2, k}) < \infty$
  $X_1 \times X_2 = \biguplus_{j, k} X_{1, j} \times X_{2, k}$, $\nu(X_{1, j} \times X_{2, k}) = \mu_1(X_{1, j})\mu_2(X_{2, k}) < \infty$
\end{proof}

\begin{remark}
  In Summary, we have proved that the product pre-measure extends uniquely to $\SM_1 \otimes \SM_2$. (under the assumption of $\sigma$-finiteness of $\mu_1, \mu_2$)
\end{remark}

\begin{remark}
  Constructed a product measure on $\SM_1 \otimes \SM_2$ by $\mu(E_1 \times E_2) = \mu_1(E_1)\mu_2(E_2)$
  (this is unique if $\SM_1, \SM_2$ are $\sigma$-finite)
\end{remark}

\begin{lemma}
  Let $E \in \SM_1 \otimes \SM_2$, Define  
  \[sl_{x_1}(E) = \{x_2: (x_1, x_2) \in E\}\]
  \[sl^{x_2}(E) = \{x_1 : (x_1, x_2) \in E\}\]
  Then $sl_{x_1}(E) \in \SM_2$, $sl^{x_2}(E) \in \SM_1$
\end{lemma}

\begin{proof}
  Considering the rectangle case, this is ``easy''
  \[sl_{x_1}(E_1 \times E_2) = \{x_2 : (x_1, x_2) \in E_1 \times E_2\} = \begin{cases}
    \emptyset, & x_1 \notin E_1\\
    E_2, & x_1 \in E_1
  \end{cases}\]
  Define
  \[\mathfrak{C}_{x_1} = \{E \in \kP(X_1 \times X_2) : sl_{x_1}(E) \in \SM_2\}\]
  we know that $\SSS_{\text{rect}} \subseteq \mathfrak{C}_{x_1}$

  Check $\mathfrak{C}_{x_1}$ is a $\sigma$-algebra (If so, $\mathfrak{C}_{x_1}$ contains $\kM(\SSS_{\text{rect}}) = \SM_1 \otimes \SM_2$)
  \begin{enumerate}[(i)]
    \item $sl_{x_1}(X_1 \times X_2) = X_2 \in \SM_2$, so, $X_1 \times X_2 \in \mathfrak{C}_{x_1}$
    \item Let $E \in \mathfrak{C}_{x_1}$, we know that $sl_{x_1}(E) \in \SM_2$, consider $E^\complement = X \setminus E$  
  \[sl_{x_1}(E^\complement) = \{x_2 : (x_1, x_2) \not\in E\} = \{x_2 : x_2 \not\in sl_{x_1}(E)\} = (sl_{x_1}(E))^\complement \in \SM_2\]
    \item For $E_j \in \mathfrak{C}_{x_1}$, we know that $sl_{x_1}(E_j) \in \SM_2$, then
  \[sl_{x_1}\left(\bigcup_{j=1}^\infty E_j\right) = \bigcup_{j=1}^\infty sl_{x_1}\left( E_j\right) \in \SM_2\]
  \end{enumerate}
  So, $\mathfrak{C}_{x_1}$ is a $\sigma$-algebra, and $\mathfrak{C}_{x_1}$ contains $\SSS_{\text{rect}}$, so $\mathfrak{C}_{x_1}$ contains $\SM_1 \otimes \SM_2$.
  Hence, for any $E \in \SM_1 \otimes \SM_2$, $sl_{x_1}(E) \in \SM_2$.
  % and is a $\sigma$-algebra. 
\end{proof}

\begin{lemma}
  Let $f$ be a $(\SM_1 \otimes \SM_2)$-measurable. Then $f_{x_1}(x_2) = f(x_1, x_2)$ is an $\SM_2$-measurable,
  and $f^{x_2}(x_1) = f(x_1, x_2)$ is an $\SM_1$-measurable
\end{lemma}

\begin{proof}
  Check: $f^{-1}_{x_1}(I) \in \SM_2$, for any Borel set $I$
  \begin{align*}
    f^{-1}_{x_1}(I) &= \{x_2 : f_{x_1}(x_2) = f(x_1, x_2) \in I\} \\
    &= sl_{x_1}(\{(x_1, x_2) : f(x_1, x_2) \in I\}) \\
    &= sl_{x_1}(\underbrace{f^{-1}(I)}_{\in \SM_1 \otimes \SM_2})
  \end{align*}
\end{proof}

\section{Extension of Measure on Integral}

\begin{theorem}[Cavalieri's Principle]
  Let $(X_1, \SM_1, \mu_1), (X_2, \SM_2, \mu_2)$ be $\sigma$-finite and let $\mu$  be a product measure of two measure space.
  For any $E \in \SM_1 \otimes \SM_2$, then
  \begin{enumerate}[(i)]
    \item $g_E : x_1 \mapsto \mu_2(sl_{x_1}(E))$ is $\SM_1$-measurable
    \item $h_E : x_2 \mapsto \mu_1(sl^{x_2}(E))$ is $\SM_2$-measurable
  \end{enumerate}
  and 
  \begin{enumerate}[(i)]
    \item $\mu(E) = \int g_E \ \mathrm{d}\mu_1 = \int \mu_2(sl_{x_1}(E)) \ \mathrm{d}\mu_1$
    \item $\mu(E) = \int h_E \ \mathrm{d}\mu_2 = \int \mu_1(sl^{x_2}(E)) \ \mathrm{d}\mu_2$
  \end{enumerate}
\end{theorem}

\begin{proof}
  First, consider the case that $\mu_1(X_1) < \infty$ and $\mu_2(X_2) < \infty$. 
  Similarly, considering the rectangle case, for any $E \in \SSS_{\text{rect}}$, 
  \[g_E(x_1) = \begin{cases}
    \mu_2(E_2) & \text{if } x_1 \in E_1 \\
    0 & \text{if } x_1 \notin E_1
  \end{cases}\]
  So, $g_E$ is $\SM_1$-measurable.
  Moreover, 
  \[\int g_E \ \mathrm{d}\mu_1 = \int_{E_1}\mu_2(E_2) \ \mathrm{d}\mu_1 = \mu_1(E_1)\mu_2(E_2) = \mu(E)\]
  proved the theorem for the rectangle case.

  Next, define 
  \[\mathfrak{C} = \left\{E \in \kP(X) : g_E \text{ is }\SM_1\text{-measurable, } \mu(E) = \int g_E \ \mathrm{d}\mu_1\right\}\]
  I will show that $\mathfrak{C}$ is a Dynkin system.
  For the case $\SM_1$-measurable,
  \begin{enumerate}[(i)]
    \item $g_X(x_1) = \mu_2(sl_{x_1}(X)) = \mu_2(X_2)$ is a constant function, so it is $\SM_1$-measurable.
    \item For $E \in \mathfrak{C}$, $g_{E^\complement}(x_1) = \mu_2(sl_{x_1}(E^\complement)) = \mu_2(X_2 \setminus sl_{x_1}(E)) = \mu_2(X_2) - \mu_2(sl_{x_1}(E))$ is $\SM_1$-measurable.
    \item For $E_j \in \mathfrak{C}$, where $E_j$ are disjoint, 
    \[g_{\biguplus_{j=1}^\infty E_j}(x_1) = \mu_2\left(sl_{x_1}\left(\biguplus_{j=1}^\infty E_j\right)\right) = \sum_{j=1}^\infty \mu_2(sl_{x_1}(E_j))\]
     is $\SM_1$-measurable.
  \end{enumerate}
  For the case of $\mu(E) = \int g_E \ \mathrm{d}\mu_1$,
  \begin{enumerate}[(i)]
    \item $$\int g_X \ \mathrm{d}\mu_1 = \int \mu_2(sl_{x_1}(X)) \ \mathrm{d}\mu_1 = \mu_1(X_1)\mu_2(X_2)= \mu(X)$$
    \item For $E \in \mathfrak{C}$,
    \begin{align*}
      \int g_{E^\complement} \ \mathrm{d}\mu &= \int \mu_2(sl_{x_1}(E^\complement)) \ \mathrm{d}\mu = \int \mu_2(X_2) - \mu_2(sl_{x_1}(E)) \ \mathrm{d}\mu \\
      &= \mu(X) - \mu(E) = \mu(E^\complement)
    \end{align*}
    \item For $E_j \in \mathfrak{C}$, where $E_j$ are disjoint,
    \begin{align*}
      \int g_{\biguplus_{j=1}^\infty E_j} \ \mathrm{d}\mu &= \int \sum_{j=1}^\infty \mu_2(sl_{x_1}(E_j)) \ \mathrm{d}\mu = \sum_{j=1}^\infty \int \mu_2(sl_{x_1}(E_j)) \ \mathrm{d}\mu\\
      &= \sum_{j=1}^\infty \mu(E_j) = \mu\left(\biguplus_{j=1}^\infty E_j\right)
    \end{align*}
  \end{enumerate}
  We have a theorem stating that if $\SE$ is $\cap$-stable then $\mathcal{D}(\SE) = \kM(\SE)$. 
  I will show that $\mathfrak{C}$ is a Dynkin system containing $\SSS_{\text{rect}}$ and $\kM(\SSS_{\text{rect}}) \subseteq \mathfrak{C}$.
  Since the ractangle cases are $\SM_1$-measurable, $\mathfrak{C}$ contains $\SSS_{\text{rect}}$. Moreover, since $\SSS_{\text{rect}}$ is 
  semiring, so it is $\cap$-stable,
  and from the definition of Dynkin system, 
  $\kM(\SSS_{\text{rect}}) = \mathcal{D}(\SSS_{\text{rect}}) \subseteq \mathfrak{C}$. 
  Hence, for any $E \in \SM_1 \otimes \SM_2$, $g_E$ is $\SM_1$-measurable and $\mu(E) = \int g_E \ \mathrm{d}\mu_1$.


    % Using standard trick $\varphi \subseteq \kP(X_1 \times X_2)$ for which part (i) of the theorem is true
  % check $\varphi$ is a $\sigma$-algebra and it contains the $\SSS_{\text{rect}}$

  % Cor of claim: $\varphi$ contains $\kM(\SSS_{\text{rect}}) \overset{\text{def}}{=} \SM_1 \otimes \SM_2$

  % Consider $E \in \SSS_{\text{rect}}$, i.e., $E = E_1 \times E_2$, $E_1 \in \SM_1, E_2 \in \SM_2$
  % then \[sl_{x_1}(E) = \begin{cases}
  %   E_2  & \text{if } x_1 \in E_1 \\
  %   \emptyset & \text{if } x_1 \notin E_1
  % \end{cases}\]
  % \[g_E(x_1) = \mu_2(sl_{x_1}(E)) = \mathbbm{1}_{E_1}(x_1) \mu_2(E_2)\]
  % \[\int g_E \ \mathrm{d}\mu_1 = \mu_1(E_1)\mu_2(E_2) = \mu(E)\implies \SSS_{\text{rect}} \in \varphi\]

  % We do the proof only for finite measure spaces ($\mu_1(X_1) < \infty, \mu_2(X_2) < \infty$)

  % Let $E \in \varphi$, $sl_{x_1}(E^\complement)$, $\mu_2(sl_{x_1}(E^\complement)) = \mu_2(X_2 \setminus sl_{x_1}(E)) = \mu_2(X_2) - \mu_2(sl_{x_1}(E))$.

  % \[\int \mu(sl_{x_1}(E^\complement)) = \int \mu(X_2) - \mu_2(sl_{x_1}(E)) \ \mathrm{d}\mu_1 = \underbrace{\mu_1(X_1)\mu_2(X_2)}_{\mu(X)} - \mu(E) = \mu(E^\complement)\]

  % Suppose $E_1, E_2$ satisfy the theorem (i), does $E_1 \cup E_2$ satisfy the theorem
  % Hard (implssible) to check directly theorem (i) for the union $E_1 \cup E_2$ 

  % To go from finite to $\sigma$-finite, 
  % \[X_1 = \biguplus_{j=1}^\infty A_{j, 1}\]
  % \[X_2 = \biguplus_{k=1}^\infty A_{k, 2}\]
  % then 
  % \[sl_{x_1}(E) = \biguplus_{k=1}^\infty (sl_{x_1}(E)\cap A_{k, 2})\]
  % By using countable additivity of $\mu_2$ we get
  % \[\mu_2(sl_{x_1}(E)) = \sum_{k=1}^\infty \mu_2(sl_{x_1}(E) \cap A_{k, 2})\]
  % then for each $k$ we konw that $\mu_2(sl_{x_1}(E) \cap A_{k, 2})$ is measurable in $X_1$ then applying 
  % countable sum of measurable function is measurable.
  % \begin{align*}
  %   \int \mu_2(sl_{x_1}(E))\ \mathrm{d}x_1 &= \int \sum_{k=1}^\infty \mu_2(sl_{x_1}(E) \cap A_{k, 2})\ \mathrm{d}x_1\\ 
  %   &= \sum_{k=1}^\infty \int \mu_2(sl_{x_1}(E) \cap A_{k, 2})\ \mathrm{d}x_1\\
  %   &= \sum_{k=1}^\infty \mu(E \cap A_{k, 2})\\
  %   &= \mu(E)
  % \end{align*}
\end{proof}

\begin{theorem}[Tonelli's Theorem]
  Let $(X_i, \SM_i, \mu_i), i=1,2$ $\sigma$-finite measure spaces, $\mu = \mu_1 \times \mu_2$ be a product measure on $\SM_1 \otimes \SM_2$. 
  Let $f \in \mathcal{L}^+(X, \mu)$ then 
  \begin{itemize}
    \item $x_1 \mapsto \int f_{x_1} \ \mathrm{d}\mu_2$ belongs to $\mathcal{L}^+(X_1, \mu_1)$
    \item $x_2 \mapsto \int f^{x_2} \ \mathrm{d}\mu_1$ belongs to $\mathcal{L}^+(X_2, \mu_2)$
  \end{itemize}
  and 
  \[\int f \ \mathrm{d}\mu = \int \int f_{x_1} \ \mathrm{d}\mu_2 \mathrm{d}\mu_1\]
\end{theorem}

\begin{proof}
  We have proved this theorem for simple functions / indicator functions of measureable sets.
  Let $\{s_n\}_{n=1}^\infty$ be a sequence of simple functions such that $s_n(x) \nearrow f(x)$ for all $x$. 
  define 
  \[g_n(x_1) = \int s_n(x_1, x_2) \ \mathrm{d}\mu_2,\ h_n(x_2) = \int s_n(x_1, x_2) \ \mathrm{d}\mu_1\]
  \[g(x_1) = \int f(x_1, x_2) \ \mathrm{d}\mu_2,\ h(x_2) = \int f(x_1, x_2) \ \mathrm{d}\mu_1\]
  By Monotone Convergence Theorem, we have $g_n \nearrow g$, $h_n \nearrow h$ and limit of measureable functions is measureable.
  By the fact that $g_n \nearrow g$, 
  \[\lim_{n\to\infty}\int g_n \ \mathrm{d}\mu_1 = \int g \ \mathrm{d}\mu_1 = \int \int f(x_1, x_2) \ \mathrm{d}\mu_2 \mathrm{d}\mu_1 \]
  \[\lim_{n\to\infty}\int g_n \ \mathrm{d}\mu_1 = \lim_{n\to\infty} \int\int s_n(x_1, x_2) \ \mathrm{d}\mu_2 \mathrm{d}\mu_1 \overset{Cavalieri}= \lim_{n\to\infty} \int s_n \ \mathrm{d}\mu \overset{\text{def}}= \int f \ \mathrm{d}\mu\]
\end{proof}

\begin{theorem}[Fubini's Theorem]
  $f \in \mathcal{L}^1(X, \mu)$ then $f_{x_1} \in \mathcal{L}^1(X_2, \mu_2)$ for $\mu_1$ almost every $x_1$ and $f^{x_2} \in \mathcal{L}^1(X_1, \mu_1)$ for $\mu_2$ almost every $x_2$
  and 
  \[\int f \ \mathrm{d}\mu = \int \int f_{x_1} \ \mathrm{d}\mu_2 \ \mathrm{d}\mu_1 = \int \int f^{x_2} \ \mathrm{d}\mu_1 \ \mathrm{d}\mu_2\]
\end{theorem}

\section{Distribution Function}
\begin{definition}
  Given $(X, \SM, \mu)$, given measurable function $f$, define
  \[\alpha \mapsto \mu_f(\alpha) = \mu(\{x : |f| > \alpha\})\]
  $\alpha > 0$
\end{definition}

\begin{theorem}
  Given $(X, \SM, \mu)$, 
  \[\int |f|^p \ \mathrm{d}\mu = \int_0^\infty p\alpha^{p-1} \mu_f(\alpha) \ \mathrm{d}\alpha\]
\end{theorem}

\begin{example}
  example where $\mu_f$ shows
  \[\mu_f(\alpha) = \frac{\|f\|^p_p}{\alpha^p}\]
  Chebyshev's inequality

  $L^{p, \infty} =$ ``weake type p-spaces'' $\iff$ $f$ measurable $\iff$ $\sup_{\alpha} \alpha^p \mu_f(\alpha) < \infty$
\end{example}

\begin{proof}
  % careful about non-sigma finite 
  We first assume $\mu$ is $\sigma$-finite.
  Considering Right-hand-side 
  \begin{align*}
    \int_0^\infty p\alpha^{p-1} \mu_f(\alpha) \ \mathrm{d}\alpha &= \int_0^\infty p\alpha^{p-1}\int_{|f(x)| > \alpha} \ \mathrm{d}\mu \mathrm{d}\alpha\\
    &= \int_0^\infty \int_X p\alpha^{p-1} \mathbbm{1}_{\{x : |f(x)| - \alpha > 0\}} (x, \alpha) \ \mathrm{d}\mu \mathrm{d}\alpha
  \end{align*}
  $x \mapsto f(x)$ is measurable as a function on $X$, and as a function on $X\times [0, \infty)$. 
  define $g: \RR^2 \to \RR, (t, \alpha) \mapsto (|t| - \alpha)\mathbbm{1}_{\alpha > 0}$ then $g \circ f$ is $\SM$-measurable 
  because $(g \circ f)^{-1} = f^{-1} \circ g^{-1}$ then
  \begin{align*}
    \int_0^\infty p\alpha^{p-1} \mu_f(\alpha) \ \mathrm{d}\alpha &= \int_X \underbrace{\int_{\alpha=0}^{|f(x)|} p\alpha^{p-1} \ \mathrm{d}\alpha}_{|f(x)|^p}\mathrm{d}\mu  
  \end{align*}
  This proves the formula for $\sigma$-finite $X$.

  For the general cases, let assume that there exists $\alpha$ such that $\mu_f(\alpha) = \infty$,
  $\mu(\{x : |f(x) > \alpha\}) = \infty$ $\implies \mu_f(\beta) = \infty$ for $0 < \beta < \alpha$.
  In this case, we have $\infty = \infty$

  Assume $\mu_f(\alpha) < \infty$ for all $\alpha > 0$

  claim: $\{x : |f(x)| > 0\} = \bigcup_{n} \{x : |f(x) > \frac1n \}$, $\sigma$-finite because 
  $ \mu(\{x : |f(x) > \frac1n \}) = \mu_f(\frac{1}n)< \infty$
\end{proof}

\begin{example}
  \[\int_{-\infty}^\infty e^{-x^2} \ \mathrm{d}x = \sqrt\pi\]
  equivalent to
  \[\int_{0}^\infty e^{-x^2} \ \mathrm{d}x = \frac{\sqrt\pi}2\]
 

\end{example}

\begin{proof}
  Let 
  \[
    F(x, y) = \begin{cases}
      ye^{-y^2(1+x^2)} & x \ge 0, y \ge 0 \\
      0 & \text{otherwise}
    \end{cases}
  \]
  \begin{align*}
    \int F \ \mathrm{d}m(x, y) = \int_{y=0}^\infty e^{-y^2}\int_{x=0}^\infty e^{-y^2x^2}y \ \mathrm{d}x\mathrm{d}y
  \end{align*}
  By changing variable, let $w = yx, \mathrm{d}w = y \ \mathrm{d}x$ then
  \begin{align*}
    \int_{y=0}^\infty e^{-y^2}\int_{x=0}^\infty e^{-y^2x^2}y \ \mathrm{d}x\mathrm{d}y = \int_{y=0}^\infty e^{-y^2}\ \mathrm{d}y \int_{w=0}^\infty e^{-w^2} \ \mathrm{d}w
  \end{align*}
  Considering LHS
  \begin{align*}
    \int F\ \mathrm{d}m = \int_{x=0}^\infty \int_y y \frac{\sqrt{1+x^2}}{\sqrt{1+x^2}} e^{-y^2(1+x^2)} \ \mathrm{d}y\mathrm{d}x 
  \end{align*}
  let $w = y \sqrt{1+x^2}$, $\mathrm{d}w = \sqrt{1+x^2} \ \mathrm{d}y$ then
  \begin{align*}
    \int F \ \mathrm{d}m &= \int_0^\infty \frac{1}{1+x^2} \int_0^\infty \frac{2w}2 e^{-w^2} \ \mathrm{d}w \mathrm{d}x \\
    &= \frac12 \int_0^\infty \frac1{1+x^2}\ \mathrm{d}x = \frac\pi4
  \end{align*}
  Considering the first integration and the second integration, we get that
  \begin{align*}
    \left(\int_0^\infty e^{-s^2} \ \mathrm{d}s\right)^2 &= \frac\pi4 \\
    \int_0^\infty e^{-s^2} \ \mathrm{d}s &= \frac{\sqrt{\pi}}2 \\
  \end{align*}
\end{proof}

\begin{example}
  \[\int_0^\infty e^{-ax} (\sin x)^2 \frac{\mathrm{d}x}x\]
  Hint: To apply Fubini-Tonelli, 
  \[\iint_{[0, \infty]\times[0, 1]} e^{-ax} \sin(2xy) \ \mathrm{d}m(x, y)\]
  (value $1/4 \ln(1+4/a^2)$)
\end{example}

\section{Linear Change of Variable}
\begin{theorem}
  Let $f \in L^1(\RR^n)$ or ($L^+(\RR^n)$) and let $A: \RR^n \to \RR^n$ be an invertible $n \times n$ matrix. Then 
  \[|\det A| \int f(Ax) \ \mathrm{d}m(x) = \int f(x) \ \mathrm{d}m(x)\]
  Moreover, $m(AE) = |\det A| m(E)$
  for every Lebesgue measureable set $E$.
\end{theorem}
\begin{proof}
  We first assume that $f$ is Borel measurable, then $x \mapsto f(Ax)$ is Borel measurable.
  (since $x \mapsto Ax$ is continuous) We consider a number of special cases.
  In what follows write $\RR^n \ni (x_1, x')$ where $x_1 \in \RR, x' \in \RR^{n-1}$.

  \textbf{Case 1:}
  \[ A = \begin{bmatrix}
    \lambda & 0 & \dotsm & 0 \\
    0 & 1 & \dotsm & 0 \\
    \vdots & \vdots & \ddots & \vdots \\
    0 & 0 & \dotsm & 1
  \end{bmatrix}\]
  $\lambda  \neq 0$ then $\det A = \lambda$. then 
  \begin{align*}
    |\lambda| \int f(Ax)\ \mathrm{d}m(x) &= |\lambda|\int_{\RR^{d-1}} \left[\int_\RR f(\lambda x, x') \ \mathrm{d}m_1(x_1)\right]\mathrm{d}m_{n-1}(x') \\
    \int_{\RR^{n-1}} \left[\int_\RR f(cx, x') \ \mathrm{d}m_1(x_1)\right]\mathrm{d}m_{n-1}(x') &= \int f(x) \ \mathrm{d}m(x) 
  \end{align*}

  \textbf{Case 2:}
  \[A = \begin{bmatrix}
    1 & \lambda  & \dotsm & 0\\
    0 & 1   & \dotsm & 0\\
    % 0 & 0 & 1 & \dotsm & 0\\
    \vdots & \vdots & \ddots & \vdots \\
    0 & 0  & \dotsm & 1
  \end{bmatrix}\]
  Then $\det A = 1$ 
  \[1 \int f(Ax) \ \mathrm{d}m(x) = \int_{\RR^{d-1}}\left[\int_\RR f(x_1 + \lambda x_2, x')\ \mathrm{d}m_1(x_1)\right] \mathrm{d}m_{n-1}(x') \]
  \[\int_{\RR^{n-1}}\left[\int_\RR f(x_1 + \lambda x_2, x')\ \mathrm{d}m_1(x_1)\right] \mathrm{d}m_{n-1}(x') = \int f(x) \ \mathrm{d}m(x)\]

  \textbf{Case 3:} A permutes two coordinates: there exist $j \neq k$ 
  where $A(e_i) = e_i$ if $i \not\in \{j, k\}$, $A(e_j) = e_k$, $A(e_k) = e_j$.
  Then $\det A = -1$.
  \[1 \int f(Ax) \ \mathrm{d}m(x) = \int f(x) \ \mathrm{d}m(x)\]

  \textbf{General case:} In general we can write $A = E_1\dotsc E_n$ where each $E_j$ is in Case 1, 2, or 3.
  \begin{align*}
    |\det A| \int f(Ax) \ \mathrm{d}x &= |\det (E_1 \dotsm E_{n-1})||\det E_n| \int f(f\circ E_1\dotsc E_{n-1})(E_n x) \ \mathrm{d}x  \\
    &= |\det (E_1 \dotsm E_{n-1})| \int f(E_1 \dotsm E_{n-1} x) \ \mathrm{d}x \\
    &= \dotsm \int f(x)\ \mathrm{d}x
  \end{align*}
  This proves the theorem when $f$ is Borel measurable. In particular, if $E$ is Borel measurable then
  \begin{align*}
    \divideontimes m(AE) &= \int \mathbbm{1}_{AE}(x) \ \mathrm{d}m(x)  \\
    &= |\det A| \int \mathbbm{1}_{AE}(Ax) \ \mathrm{d}m(x) \\
    &= |\det A| \int \mathbbm{1}_E(x) \ \mathrm{d}m(x) \\
    &= |\det A| m(E)
  \end{align*}

  Finally consider the case when $f$ is Lebesgue measurable.
  Then there is a Borel measurable function $g$ and a Borel measurable set $E$ with $m(E) = 0$
  and $f = g$ on $\RR^d \setminus E$ (so $f = g$ almost everywhere).
  Then $f \circ A = g\circ A$ on $\RR^d \setminus A^{-1}(E)$. By $\divideontimes$ $m(A^{-1}E) = 0$
  So $f \circ A$ is Lebesbue measurable and $f\circ A = g \circ A$ almost everywhere. So
  \begin{align*}
    |\det A| \int f(Ax) \ \mathrm{d}m(x) &= |\det A| \int g(Ax) \ \mathrm{d}m(x) \\
    \int g(x) \ \mathrm{d}m(x) &= \int f(x) \ \mathrm{d}m(x)
  \end{align*}
  Also, arguing exactly as in $\divideontimes$ we have $m(AE) = |\det A| m(E)$ for every Lebesgue measurable set $E$.
\end{proof}

\begin{theorem}
  Assume $f, g : [a, b] \to \RR$ are Lebesgue integrable. 
  Define
  \[F(x) = \int_{[a, x]} f \ \mathrm{d}m\]
  \[G(x) = \int_{[a, x]} g \ \mathrm{d}m\]
  then 
  \[\int_{[a, b]} Fg \ \mathrm{d}m = F(b)G(b) - \int_{[a, b]} fG \ \mathrm{d}m\]
\end{theorem}
\begin{remark}
  if $u, v: [a, b] \to \RR$ are continuously differentiable with $u(a) = v(a) = 0$, then 
  \[u(x) =\int_{[a, x]} u' \ \mathrm{d}m\]
  \[v(x) =\int_{[a, x]} v' \ \mathrm{d}m\]
  So, the above theorem implies 
  \[\int_{[a, b]}uv' \ \mathrm{d}m = u(b)v(b) - \int_{[a, b]} u'v \ \mathrm{d}m\]
\end{remark}

\begin{proof}
  Let $T = \{(x, t) : a \le x \le b, a \le t \le x\} = \{(x, t) : a \le t \le b, t \le x \le b\}$
  Then $T$ is a Lebesgue measurable set in $\RR^2$ and $(x, t) \mapsto f(t)g(x) \mathbbm{1}_T(x, t)$ is Lebesgue measurable. 
  Also by Tonelli's Theorem, 
  \begin{align*}
    \int |f(t)g(x) \mathbbm{1}_T(x, t)| \ \mathrm{d}m_2(x, t) &= \int_{[a, b]} |g(x)|\left[\int_{[a, x]}|f(t)|\ \mathrm{d}t\right] \mathrm{d}x \\
    &\le \int_{[a, b]} |g| \ \mathrm{d}m \int_{[a, b]} |f| \ \mathrm{d}m < \infty
  \end{align*}
  So, by Fubini's Theorem,
  \begin{align*}
    \int f(t) g(x) \mathbbm{1}_T(x, t) \ \mathrm{d}m_2(x, t) &= \int_{[a, b]}g(x) \left[\int_{[a, x]} f(t) \ \mathrm{d}t\right] \mathrm{d}x \\
    &= \int_{[a, b]} gF \ \mathrm{d}m
  \end{align*}
  and 
  \begin{align*}
    \int f(t)g(x)\mathbbm{1}_T(x, t) \ \mathrm{d}m_2(x, t) &= \int_{[a, b]} f(t) \left[\int_{[t, b]} g(x) \ \mathrm{d}x\right] \mathrm{d}t \\
    &= \int_{[a, b]} f(t) \left[\int_{[a, b]} g(x) \ \mathrm{d}x - \int_{[a, t]} g(x) \ \mathrm{d}x\right]\mathrm{d}t \\
    &= \int_{[a, b]} f(t) \left[ \int_{[a, b]} g(x) \ \mathrm{d}x \right]\mathrm{d}t \\
    &- \int_{[a, b]} f(t) \left[ \int_{[a, t]} g(x) \ \mathrm{d}x \right]\mathrm{d}t \\
    &= F(b)G(a) - \int_{[a, b]} fG \ \mathrm{d}m 
  \end{align*}
  So,
  \[\int_{[a, b]} Fg \ \mathrm{d}m = F(b)G(b) - \int_{[a, b]} fG \ \mathrm{d}m\]
\end{proof}

\section{Change of Variable}
We konw $A \in GL(n, \RR)$ $A$ is $n \times n$ matrix, $A$ is invertible $\iff$ $\det A = 0$.
If $f \in \mathcal{L}^+$ or $f \in \mathcal{L}^1$ then
\[\int f(x) \ \mathrm{d}m(x) = \int f(Ax) |\det A| \ \mathrm{d}m(x)\]
\[A \Omega = \{Ax : x \in \Omega\}\]

The step is to replace linear transformation to non-linear transformation.

$\phi(x) = \phi(x_1, \dotsc, x_n)$, $\phi$, $\RR^n$-valued, $\phi = (\phi_1, \dotsc, \phi_n)$.
we know that $\phi$ is differentiable on $\Omega$. if $\frac{\partial \phi_j}{\partial x_k}$ exist and are continuous. 
\[\phi'(x) = \begin{pmatrix}
  \frac{\partial \phi_1}{\partial x_1} & \dotsm & \frac{\partial \phi_1}{\partial x_n} \\
  \vdots & \ddots & \vdots \\
  \frac{\partial \phi_n}{\partial x_1} & \dotsm & \frac{\partial \phi_n}{\partial x_n} \\
\end{pmatrix}\]
$\phi(x) = Ax, \phi'(x) = A$


\begin{definition}
  Let $\Omega_1 \subseteq \RR^n$, $\Omega_2 \subseteq \RR^n$. 
  We say that $\phi: \Omega_1 \to \Omega_2$ is a $C^1$-diffeomorphism.
  if \begin{enumerate}[(i)]
    \item $\phi$ is bijective
    \item $\phi \in C^1$ (or $\frac{\partial \phi_j}{\partial x_k}$ exist and are continuous)
    \item $\det \phi'(x) \neq 0$ (i.e., $\phi'(x)$ is invertible)
  \end{enumerate}
\end{definition}

\begin{remark}
  We say that $\phi_1$ and $\phi_2$ are diffeomorphially equivalent.
  \begin{enumerate}
    \item $\Omega_1 \sim \Omega_1$, $\phi(x) = x$, $\phi'(x) = I$
    \item $\phi: \Omega_1 \to \Omega_2$, $C^1$-diffeomorphism, $\phi^{-1}(x)$ is also a $C^1$-diffeomorphism. 
    (use the inverse function theorem which proves the differentiability of $\phi^{-1}$)
    $(\phi^{-1})'(y) = (\phi'(\phi^{-1}(y)))^{-1} = (\phi'(x))^{-1}$ where $x = \phi^{-1}(y)$
    \item  
  \end{enumerate}
\end{remark}

\begin{example}
  $\phi:\underbrace{(0, 2) \times (0, \pi/4)}_{\Omega_1} \to \Omega_2$ defined by
  $\phi(x_1, x_2) = (x_1 \sin x_2, x_1 \cos x_2)$
  $\phi(r, \theta) = (r\sin \theta, r\cos \theta)$
  \[\phi'(x) = \begin{pmatrix}
    \sin x_2 & x_1 \cos x_2 \\
    \cos x_2 & -x_1 \sin x_2
  \end{pmatrix}\]
  $\det \phi'(x) = -x_1 = -r$
\end{example}

\begin{theorem}
  Let $\Omega_1, \Omega_2$ are open sets in $\RR^n$, $\phi: \Omega_1 \to \Omega_2$ is a $C_1$-diffeomorphism.
  $f \in \mathcal{L}^1(\Omega_2)$ or $f \in \mathcal{L}^+(\Omega_2)$ then 
  \[\int_{\Omega_2} f(y) \ \mathrm{d}m(y) = \int_{\Omega_1} f(\phi(x)) |\det \phi'(x)|\ \mathrm{d}m(x)\]
\end{theorem}

\begin{remark}
  In calculus, we have substitution rule
  \[\int_{\phi(a)}^{\phi(b)} f(y)\ \mathrm{d}y = \int_a^b f(\phi(x))\phi'(x)\ \mathrm{d}x\]
\end{remark}

\begin{proof}
  It is enough to show that 
  \[\int_{\Omega_2} f(y) \ \mathrm{d}m(y) \le \int_{\Omega_1} f(\phi(x)) |\det \phi'(x)| \ \mathrm{d}m(x)\]
  Then we may apply this inequality for the diffeomorphism $\phi^{-1}: \Omega_2 \to \Omega_1$ to get the reverse inequality.
  \[\int_{\Omega_1} g(x) \ \mathrm{d}m(x) \le \int_{\Omega_2} g(\phi^{-1}(y)) |\det(\phi^{-1})'(y)| \ \mathrm{d}m(y)\]
  Apply
  \[g(x) = f(\phi(x)) |\det \phi'(x)|\]
  \[\int_{\Omega_1} g(x) \ \mathrm{d}m(x) \le \int_{\Omega_2} f(\phi(\phi^{-1}(y))) |\det \phi'(\phi^{-1}(y))| |\det(\phi^{-1})'(y)| \ \mathrm{d}m(y)\]
  we know that $\phi(\phi^{-1}(y)) = y$, $\phi^{-1}(\phi'(x)) = x$ then take the differentiable, we get
  \[(\phi^{-1})'(\phi(x)) \phi'(x) = I\]
  \[\phi'(\phi^{-1}(y)) (\phi^{-1})'(y) = I\]

  If we can prove $f \in L^+(\Omega_2)$ we can also prove it for general $f \in L^1$

  Noted that it suffices to prove the $\le$

  \textbf{First step:} $f = \mathbbm{1}_{\phi(E)}$ we need to prove 
  \[m(\phi(E))\footnote{Note that $\mathbbm{1}_{\phi(E)}(\phi(x)) = \mathbbm{1}_E(x)$} \le \int_E |\det \phi'(x)|\ \mathrm{d}x\]
  Assume that $E$ is a cube $Q$ away from the boundary $\mathrm{dist}(Q, \Omega_1^\complement) > 0$
  For any $\eps > 0$, let $\delta > 0$, 
  \[\|\phi'(\widetilde{x})^{-1}\phi'(x) - I\|_{\infty-\infty} < \eps,\ \ \text{where }|x - \widetilde{x}|_{\infty}  < \delta\]
  where $|x - \widetilde{x}|_{\infty} = \max_{1\le j \le n} |x_j - \widetilde{x}_j|$ and $\|A\|_{\infty-\infty} = \max_{i} \sum_{k=1}^n|a_{ik}|$
  we decompose $Q = \biguplus Q^\nu$ where $\mathrm{diam} Q^\nu < \delta$ then
  $|\det \phi'(x) - \det\phi'(\widetilde{x})| < \eps$ if $|x - \widetilde{x}| < \delta$ 
  Now, we get that $\phi(Q_\nu)$ are all disjoint then 
  \[m(\phi(Q)) = \sum_{\nu} m(\phi(Q_\nu))\]
  \[\phi(Q_\nu) = \phi'(x_\nu) \phi'(x_\nu)^{-1}\phi\]
  then 
  \[\Psi^{\nu} = x \mapsto \phi'(x_{\nu})^{-1}\phi(x) \]
  for $x \in Q^\nu$, $\| (\Psi^\nu)' - I\|_{\infty - \infty} < \eps$

  \textbf{Claim:} $\Psi^\nu(Q^\nu)$ contain in a cube centered at $\Psi^\nu(x^\nu)$
  with side length = $(1 + \eps)$ side length($Q_\nu$).
  Thus $m(\Psi^nu(Q^\nu)) \le (1 + \eps)^n m(Q_\nu)$.
  (Need to estimate $\Psi^\nu(x) - \Psi^\nu(x^\nu)$ for $x \in Q_\nu$)

  Assuming the claim.
  \[m(\phi(Q_\nu)) = |\det \phi'(x_\nu)|m(\Psi^\nu(Q))\]
  by the Corollary for linear differentiation then
  \begin{align*}
    m(\phi(Q_\nu)) &\le |\det \phi'(x_\nu)|(1+\eps)^n\int_{Q_\nu} \ \mathrm{d}m(x) \\
    &= (1+\eps)^n \sum_{\nu}\int_{Q_\nu} \underbrace{|\det \phi'(x) - (\det \phi'(x) - \det \phi'(x^\nu))|}_{\le |\det \phi'(x)| + \eps} \ \mathrm{d}m(x) \\
    &\le (1 + \eps)^n \sum_{\nu}\int_{Q^\nu} |\det\phi'(x)| + (1 + \eps)^n \sum_\nu \int_{Q_\nu} \eps \\
    &\le (1 + \eps)^n \int_Q |\det \phi'(x)| \ \mathrm{d}m(x) + (1+\eps)^n\eps m(Q)
  \end{align*}

  For the \textbf{Claim}: we need to check
  \[|\Psi^\nu_j(x) - \Psi_j^\nu(x^\nu)| \le (1 + \eps)\max_{1 \le k \le n} |x_k - x_k^\nu|\]
  % Thus $m(\Psi^\nu(Q^\nu)) \le (1 + \eps)^n m(Q^\nu)$
  \begin{align*}
    |\Psi^\nu_j(x) - \Psi_j^\nu(x^\nu)| &= \left|\int_{s=0}^1 \sum_{k=1}^n (x_k - x_k^\nu) \frac{\partial \Psi_j^\nu}{\partial x_k} (x^\nu + s(x - x^\nu)) \ \mathrm{d}s\right| \\
    &=\left|(x_j - x_j^\nu)\int_0^1\sum_{k=1}^n (x_k - x_k^\nu) \left(\frac{\partial\Psi_j^\nu}{\partial{x_k}} - I_{jk}\right)(x^\nu + s(x - x^\nu)) \ \mathrm{d}s\right| \\
    &\le |x_j - x_j^\nu| + \int_0^1 \|x - x^\nu\|_{\infty} \underbrace{\sum_{k} \left|\frac{\partial\Psi_j^\nu}{\partial{x_k}} - I_{jk}\right| (x^\nu + s(x - x^\nu))}_{\le \eps}  \ \mathrm{d}s \\
    &\le |x_j - x_j^\nu| + \eps \|x - x^\nu\|_\infty 
    % &= \sum_{k=1}^n(x_k - x_k^\nu) \int_0^1 \frac{\partial \Psi_j^\nu}{\partial x_k} (x^\nu + s(x - x^\nu)) \ \mathrm{d}s
  \end{align*}
  So, $\|\Psi^\nu(x) - \Psi^\nu(x^\nu)\|_\infty\|_\infty \le (1 + \eps)\|x - x^\nu\|_\infty$

  If $E$ is an open set $\mathcal O \subseteq \Omega_1$ then by Whitney theorem, we can decompose 
  \[\mathcal O = \biguplus_{l=1}^\infty Q_l\]
  where the cubes $Q_l$ are disjoint and aways stay from the boundary of $\Omega_1$.
  then
  \[m(\phi(\mathcal O)) = \sum m(\phi(\mathcal O_l)) \le \sum_l\int_{Q_l} |\det \phi'(x)| \ \mathrm{d}m = \int_{\mathcal O}|\det \phi'(x)| \ \mathrm{d}x\]
  Let $R \in \NN$ then 
  \[\Omega_{1, R} \{x : |x| \le R, \mathrm{dist}(x_1, \Omega_1^\complement) \ge \frac1R\}\]
  and we get that $\overline{\Omega_{1, R}}$ is compact and
  \[\int_{\Omega_{1, R}} |\det \phi'(x)| < \infty\]
  Let $E \subseteq \Omega_{1, R}$ then find an $\mathcal{O}_k, m(\mathcal{O}_k) \le m(E) + 2^{-k}$.
  We can construct such that $\mathcal{O}_1 \supseteq \mathcal{O}_2 \supseteq \dotsm$.
  we canuse continuity from abouve or dominated convergence theorem to get
  \[m(E) \le \int_{\mathcal{O}_k} |\det \phi'(x)| \ \mathrm{d}m(x) \to \int_{E} |\det \phi'(x)| \ \mathrm{d}m(x)\]

  \textbf{Main Step:} (special case of $f = \mathbbm{1}_E$)
  \[m(\phi(E)) \le \int_E |\det \phi'| \ \mathrm{d}m\]
  for $E \subseteq \Omega_1$ open set.
  replace $\phi(E)$ by $A$ and $E$ by $\phi^{-1}(A)$
  $A_1, \dotsc, A_j$ are given, $c_1, \dotsc, c_j$, $c_j > 0$,
  let 
  \[s(x) = \sum_{i=1}^j c_i \mathbbm{1}_{A_i}\]
  then
  \begin{align*}
    \sum_{i=1}^j c_k \int \mathbbm{1}_{A_i} \ \mathrm{d}m \le \sum_{i=1}^j c_i \int_{\phi^{-1}(A_i)} |\det \phi'(x)| \ \mathrm{d}m
  \end{align*}
  \[x \in \phi^{-1}(A_i) \iff \phi(x) \in A_i\]
  then $\mathbbm{1}_{\phi^{-1}(A_i)} = \mathbbm{1}_{A_i} (\phi(x))$
  so
  \begin{align*}
    \int s(y) \ \mathrm{d}m(y) &\le \int \sum c_i \mathbbm{1}_{\phi^{-1}(A_i)}(x) |\det \phi'(x)| \ \mathrm{d}m(x)
    &\le \int s(\phi(x)) |\det \phi'(x)| \ \mathrm{d}m(x)
  \end{align*}
  then $s_n \nearrow f$ apply MCT given the formula for $f \in \mathcal{L}^+$
  \[f \in L^1 \implies f = (\Re f)_+ - (\Re f)_- + i((\Im f)_+ - (\Im f)_-)\]
\end{proof}

\section{Lebesgue Differentiation Theorem}

In Calculus, $f \in C[a, b]$
we have theorem that 
\[F(x) = \int_a^x f(t) \ \mathrm{d}t\]
then $F$ is differentiable at all $x \in [a, b]$ and $F' = f$
we use the prove that
\[\lim_{h \to 0^+} \frac{F(x+h) - F(x)}{h} = f(x), x < b\]
\[\lim_{h \to 0^-} \frac{F(x+h) - F(x)}{h} = f(x), x > a\]
rewrite the limit as 
\[\frac{F(x+h) - f(x)}h = \frac1h \int_x^{x+h} (f(t) - f(x) + f(x)) \ \mathrm{d}t = f(x) + \frac1h\int_x^{x+h}f(t) - f(x) \ \mathrm{d}t\]
For $t$ close to $x$, use $|f(t) - f(x)| < \eps$, $|t| < \delta$ for given
similarly for negative

\begin{theorem}[Lebesgue Differentiation Theorem]
  If $f \in L^1([a, b])$,
  \[F(x) = \int_{[a, x]} f \ \mathrm{d}m\]
  then $F$ is differentiable almost everywhere and $F' = f$ almost everywhere.
  %  then for almost every $x \in \RR^n$
  % \[\lim_{r \to 0} \frac1{m(B_r(x))} \int_{B_r(x)} f(y) \ \mathrm{d}m(y) = f(x)\]
\end{theorem}
\begin{definition}
  In $\RR^n$ averages over ball $B(x, r) = \{y : |y-x| < r\}$
  \[\mathcal{A}_r f(x) = \frac{1}{m(B(x, r))} \int_{B(x, r)} |f| \ \mathrm{d}m\]
\end{definition}
\begin{definition}
  \[M_{\mathrm HL}f(x) = \sup_{r > 0} \mathcal{A}_r |f|\] 
\end{definition}

\begin{remark}
  Note the function
  $x \mapsto \mathcal{A}_r f(x)$ is continuous
  \[\left|\int_{B(x, r)} |f| - \int_{B(\widetilde{x}, r)} |f| \right| \le \int_{B(x, r) \bigtriangleup B(\widetilde{x}, r)} |f| \to 0\]
  as $x \to \widetilde{x}$
  Observe that
  \[\Omega_\alpha = \left\{x : \sup_{r > 0} \mathcal{A}_r f(x) > \alpha\right\}\]
  is an open set because union of open sets are open set
\end{remark}

\begin{theorem}[Hardy and Littlewood]
  $M$ satisfies the inequality
  \begin{enumerate}[(i)]
    \item $m\left(\left\{x : M_{\mathrm HL}f(x) > \alpha\right\}\right) \le C_n \frac{\|f\|_1}{\alpha}$
    \item If $p > 1$, $\|M_{\mathrm HL}f\|_p \le C_{p, n} \|f\|_p$
    \item is clearly true for $p = \infty$ (for simplicity for continuous $f$)
  \end{enumerate}
  In general part (ii) is an instance of an interpolation.
\end{theorem}

We say that a constant $A$ essentially bounds a function $f$ if 
$|f(x)| \le A$ for almost every $x$.
%TODO: define operator esssup
\[\esssup |f(x)| = \inf \{A : A \text{ essentially bounds } |f|\}\]
\[|f(x)| \le B + \frac1n \text{ for } x\in X \setminus N_n \implies |f(x)| < B \text{ for } x \in X \setminus \bigcup N_n\]

\begin{proof}
  $\Omega_\alpha = \{x : M_{\mathrm HL}f(x) > \alpha\}$,
  $m(\Omega_\alpha) = \sup_{K \subseteq \Omega_\alpha} m(K)$.
  It suffices to show that for every compact set $K$ contained in $\Omega_\alpha$, 
  $m(K) \le 3^n \frac{\|f\|_1}\alpha$
  fix $x \in K$ then there exists a ball $B(x, r(x))$ such that 
  \[\frac1{m(B(x, r(x)))} \int_{B(x, r(x))} |f| \ \mathrm{d}m> \alpha \implies m(B(x, r(x))) < \frac1\alpha\int_{{B(x, r(x))}} |f| \ \mathrm dm\]
  there is a finite family of ball $B(x_i, r(x_i))$, for $i = 1, \dotsc, L$ such that 
  \[K \subseteq \bigcup_{B \in \kB} B\]
  $K$ is compact

  \textbf{Covering Lemma} Given ball $\kB = \{B_1, \dotsc, B_L\}$ there exists subcollection $\widetilde{\kB} \subseteq \kB$
  of ball $\{B : B \in \widetilde{B}\}$ which is pairwise disjoint such that the three fold balls
  $\{3B : B \in \widetilde{B}\}$ contain 
  \[\bigcup_{B \in \kB} B\]
  (union of the original ball) then
  Notice that $B^* \equiv 3B\equiv$ ball of radius 3 radius of $B$ centered at the center of $B$.
  ($x_B + 3(B - x_B)$) 

  \textbf{Proof by induction} on the cardinality of $\kB$. $|\kB| = N > 1$,
  find ball $B_1 \in \kB$ with the maximal radius. Observe that for any $B \in \kB$, if $B \cap B_1 \neq \emptyset$
  then $B \subseteq B_1^*$.
  Consider the collection $\kB_{2} = \{\text{all ball in } \kB \text{ which are disjoint from }B_1\}$
  By the induction hypothesis (and $|\kB_2| < |\kB| = N$)
  there are finitely many balls $B_2, \dotsc, B_M$ disjoint and 
  \[\bigcup_{j=2}^M B_j^* \supseteq \bigcup_{B \in \kB_2}B\]
  All balls in $\kB$, that intersect $B_1$ are contained in $B_1^*$.
  \[m(B(x, r(x_j))) \le \frac1\alpha \int_{B(x_j, r(x_j))} |f|\]
  Apply the covering lemma, we get balls
  \[B_1 = B(x_1, r(x_1)), \dotsc, B_M = B(x_M, r(x_M))\]
  disjoint, and 
  \[\bigcup B_i^* \supseteq K\]
  then 
  \begin{align*}
    m(K) &\le m\left(\bigcup_{i=1}^M B_i^*\right) 
    \le \sum_{i=1}^M m(B_i^*)  = 3^n \sum_{i=1}^Mm(B_i) \\
    &\le 3^n\sum_{i=1}^n\frac1\alpha \int_{B_i} |f| \ \mathrm{d}m\le \frac{3^n}\alpha \int|f|\ \mathrm{d}m = 3^n\alpha^{-1}\|f\|_{1}
  \end{align*}
\end{proof}



Claim in the Lebesgue differentiation theorem
Let $f \in L^1$ then for almost every where $x \in \RR^n$ 
\[\frac1{m(B(x, r))}\int_{B(x, r)} f(y) \ \mathrm{d}m(y) = f(x)\]
slightly strong statement
\[\frac1{m(B(x, r))}\int_{B(x, r)} |f(y) - f(x)| \ \mathrm{d}m(y) \to 0\]
for almost everywhere $x$ (If this is true for a specific $x \in \RR^n$ then we call $x$ a Lebesgue point)

\begin{theorem}
  for $f \in L^1$ then for almost every $x \in \RR^n$
  \[\lim_{r \to 0} \frac1{m(B(x, r))}\int_{B(x, r)} |f(y) - f(x)| \ \mathrm{d}m(y) = 0\]
\end{theorem}

\begin{proof}
  Define 
  \[E_\alpha = \left\{x : \limsup_{r \to 0} \frac1{m(B(x, r))}\int_{B(x, r)} |f(y) - f(x)| \ \mathrm{d}m(y) > \alpha\right\}\]
  The goal is to show that for any $\alpha > 0$,
  \[m\left(E_\alpha\right) = 0\]

  We know that for every $\delta > 0$ there exists $g$ continuous with compact support that
  \[\|f-g\|_1 < \delta\]
  So, we can get
  \[\frac{1}{m(B(x, r))}\int_{m(B(x, r))} |f(y) - g(y)| \ \mathrm{d}m(y) \to 0, \ r\to0\]
  since $f(y)-f(x) = (f(y) - g(y)) + (g(y) - g(x)) + (g(x) - f(x))$ then 
  \begin{align*}
    &\limsup_{r\to0} \frac{1}{m(B(x, r))}\int_{m(B(x, r))} |f(y)-f(x)| \ \mathrm{d}m(y) \\
    &=\limsup_{r\to0} \frac{1}{m(B(x, r))}\int_{m(B(x, r))} |(f-g)(y) - (f-g)(x)| \ \mathrm{d}m(y) \\
    &\le \limsup_{r\to0} \frac{1}{m(B(x, r))}\int_{m(B(x, r))} |(f-g)(y)|  \ \mathrm{d}m(y) + |f(x)-g(x)| \\
    &\le M_{\mathrm HL}(f-g)(x) + |f(x) - g(x)| 
  \end{align*}
  then 
  \begin{align*}
    m(E_\alpha) &\le m\left(\left\{x : M_{\mathrm HL}(f-g)(x) > \alpha/2\right\}\right) + m\left(\left\{x : |f(x) - g(x)| > \alpha/2\right\}\right) \\
    &\le \frac{C}{\alpha/2} \|f-g\|_1 + \underbrace{\frac{1}{\alpha/2} \|f-g\|_1}_{\text{from Chebyshev}} \\
    &\le \frac{2C+2}{\alpha} \|f-g\|_1 \\
    &\le \frac{2C+2}{\alpha} \delta \\
    &< \eps
  \end{align*}
\end{proof}

\[\int_K |f| \ \mathrm{d}m < C_K\]
for all compact set.
Fix any ball $K$ of radius 1. show that
\[\int_{B(x, r)}||f(y) - f(x)| \ \mathrm dm \to 0 \text{ as } r \to 0\]
$f = f \mathbbm{1}_{K^*} + f \mathbbm{1}_{\RR^n \setminus K^*}$
If $x \in K$ and $r$ is sufficiently small then
\[\int_{B(x, r)} |f(x) - f(y)| \mathrm{d}m(y) = \int |f \mathbbm{1}_{K^*}(x) - f \mathbbm{1}_{K^*}(y)| \ \mathrm{d}m(y)\]

Extension, if $E_r(x)$ is a sequence of sets such that
$m(E_r(x)) \ge \gamma m(B(x, r))$
then we shall have 
\[\int_{E_r(x)} |f(x) - f(y)| \ \mathrm{d}m(y) \to 0\]
when $r\to 0$ a.e.

We can replace $B(x, r)$ b


``Trivial inequality''
\[\|Mf\|_\infty \le \|f\|_\infty\]
where $\|\cdot\| = \esssup$

\begin{theorem}
  For $p > 1$ 
  \[\|Mf\|_p \le C_p \|f\|_p\]
\end{theorem}
More generally, let $T$ defined on $L^1, L^\infty$ and therefore $L^1 + L^\infty$.
and let $T$ be sublinear (subadditive) $T(f + g){(x)} \le T(f)(x) + T(g)(x)$
Assume
\begin{enumerate}[(i)]
  \item weak type (1, 1) inequality
  \[m(\{x : |Tf(x)| > \alpha\}) \le A \frac{\|f\|_1}{\alpha}\]
  \item $\|Tf\|_\infty \le B\|f\|_\infty$
\end{enumerate}
for all $f \in L^1 \cup L^\infty$
Then there is $C_p$ such that 
\[\|Tf\|_p \le C_p A^{1/p} B^{1-1/p}\|f\|_p\]
UseLayer-cake formula for $L^p$-norm of $g$.
\[\int|g|^p \ \mathrm{d}\mu = \int_0^\infty p\alpha^{p-1} \mu(\{x : |g| > \alpha\}) \ \mathrm{d}\alpha\]
We apply this for $Tf, \beta = \beta(\alpha)$
\[f_\beta(x) = \begin{cases}
  f(x) & |f(x)| \le \beta \\
  0 & |Tf(x)| > \beta 
\end{cases}\]
\[f^\beta(x) = \begin{cases}
  0 & |f(x)| \le \beta \\
  f(x) & |f(x)| > \beta
\end{cases}\]
then $f = f_\beta + f^\beta$ 
\[|Tf| \le |Tf_{\beta(\alpha)} + |Tf^{\beta(\alpha)}|\]
we know that
\[\{x : |Tf| > \alpha \} \subseteq \{x : |Tf_\beta| > \frac\alpha2\} \cup\{x : |Tf^\beta| > \frac\alpha2\} \]
then
\[m(\{x : |Tf| > \alpha \}) \le m(\{x : |Tf_\beta| > \frac\alpha2\})+ m(\{x : |Tf^\beta| > \frac\alpha2\})\]
$|f_\beta| \le \beta$ a.e $\implies$ $|Tf_\beta| \le B\beta$ a.e.

Define $\beta = \beta(\alpha) = \frac{\alpha}{2B}$ then the first term $m(\{x : |Tf_\beta| > \frac\alpha2\}) = 0$. 
\begin{align*}
  \|Tf\|_p^p &\le p\int_0^\infty \alpha^{p-1} m(\{x : |Tf^{\beta(\alpha)}(x)| \ge \frac\alpha2\}) \ \mathrm{d}\alpha \\
  &\le p\int_0^\infty \alpha^{p-1} \frac{A}\alpha \int_{\{|f(x)| > \beta(\alpha)\}} |f^{\beta(\alpha)}(x)| \ \mathrm{d}m \ \mathrm{d}\alpha \\
  &= \int_X\int_{\alpha=0}^{2B|f(x)|} p\alpha^{p-2} \ \mathrm{d}\alpha |f(x)| \ \mathrm{d}m(x) \\
  &= \int_X |f(x)| A\frac p{p-1}(2B |f(x)|)^{p-1} \ \mathrm{d}m(x) \\
  &= \|f\|_p^p AB^{p-1} \underbrace{\frac{2^{p-1}p}{p-1}}_{C_p^p}
\end{align*}

Weak $L^1$ also $L^{1, \infty}$

\[\|f\|_{L^{1, \infty}} = \sup_{\alpha > 0} \alpha \mu({x : |f(x)| > \alpha})\]
then 
\begin{itemize}
  \item $\mu_f(\alpha) \le \frac1\alpha\|f\|_{L^{1, \infty}}$
  \item $\|cf\|_{L^{1, \infty}} = |c|\|f\|_{L^{1, \infty}}$
  \item $\|f + g\|_{L^{1, \infty}} \le 2(\|f\|_{L^{1, \infty}} + \|g\|_{L^{1, \infty}})$
\end{itemize}

Q: Is there a norm $\|\cdot\|$ such that $\|f\| \approx \|f\|_{L^{1, \infty}}$?

A: No

Argue by contradiction, let $\|f\|_*$ be a norm on $L^{1, \infty}(\RR, m)$
Goal is to show that there is a sequence $f_{N} \in L^{1, \infty}$ such that
\[ \frac{\|f_N\|_{L^{1, \infty}}}{\|f_N\|_*}\to \infty\]
Simple building block is $\frac1{|x|}$, $m\left(\{x : \frac1{|x|} > \alpha\}\right) = \frac2\alpha$
$\frac1{|x-k|} \in L^{1, \infty}$ and $g_k(x)= \frac{1}{|x-k|}$ then
$\|g_k\|_{L^{1, \infty}} = c$ then
\begin{align*}
  f_N = \sum_{k=1}^N g_k(x) &= \sum_{k=1}^N \frac1{|x-k|} \\
  \|f_N\|_* &\le \sum_{k=1}^N \|g_k\|_* \le cN
\end{align*}
Consider this $f_N$ in $[0, N]$, for $x \in [\frac N4, \frac N2]$
\[\sum_{k=1}^N \frac1{|x-k|} > c\sum_{1 \le l \le N/2} \frac 1l \ge c^1 \log N\]
Conclusion on a set of measure $N/4$, $|f_N(x)| > c^1 \log N$.
Let $\alpha = c^1 \log N$ then 
\begin{align*}
  \|f_N\|_{L^{1, \infty}} &\ge c^1 \log N \mu(\{x : |f_N(x)| > c^1 \log N\}) \\
  &> \frac{N}4 c^1 \log N
\end{align*}

\section{Convolution}
or approximation of the identity.
Let take $\phi \in L^1$ (or much nicer)
\[\int \phi = 1,\ \int \frac1{t^n}\phi\left(\frac yt\right) \ \mathrm{d}m(y) = \int \phi = 1\]
\[f * \phi_t = \int f(x-y) \frac1{t^n}\phi\left(\frac yt\right) \mathrm{d}y,\ \phi_t = \frac1{t^n} \phi\left(\frac nt\right)\]

If $f$ is continuous, $\lim\limits_{|x|\to\infty} f(x) = 0$
\begin{align*}
  \int f(x-y) \frac{1}{t^n} \phi\left(\frac yt\right) \ \mathrm{d}t - f(x) &= \int f(x-y) - f(x) \frac1{t^n} \phi\left(\frac yt\right) \ \mathrm{d}m(y) \\
\end{align*}

Claim: $f$ is uniformly continuous on $\RR$

Given $\eps$, there is $\delta$ such that $|f(x-y) - f(x)| < \eps$ if $|y| < \delta$
\begin{align*}
  \left|\int_\RR f(x-y) - f(x) \frac1{t^n} \phi\left(\frac yt\right) \ \mathrm{d}t\right| &= \\
  \int_{|y| < \delta} \underbrace{|f(x-y) - f(x)|}_{< \eps} \left|\frac1{t^n} \phi\left(\frac yt\right)\right| \ \mathrm{d}t& + \int_{|y| > \delta} \underbrace{|f(x-y) - f(x)|}_{< \eps} \left|\frac1{t^n} \phi\left(\frac yt\right)\right| \ \mathrm{d}t\\
  &\le \eps \int \left|\frac1{t^n} \phi(\frac yt)\right| \mathrm{d}t + \int_{|y| \ge \delta} 2 \|f\|_{\infty} \frac{1}{t^n} |\phi(\frac yt)| \ \mathrm{d}t \\
\end{align*}

then $\int_{|y| \ge \delta} \frac1{t^n} |\phi(\frac yt)| \ \mathrm{d}t$ change variable
$\int_{|w| > \delta /t} |\phi(w)| \mathrm{d}w \to 0$ on $t \to \infty$

$f \in L^1$, $\|f(\cdot - y) - f(\cdot)\|_{L^1} \to 0$ as $|y| \to 0$

There exists $g$ with compact support such that $\|f-g\|_{L^1} < \eps$
then 
\[\|f(\cdot - g) - f(\cdot) -(g(\cdot y) - g(\cdot)) + g(\cdot -y) - g(\cdot)\|_{L^1}\|_1 + \|g(\cdot-y)-g(\cdot)\|_1 \to 0\]
as $y\to0$

$f \in L^1$, $\phi \in L^1$, $\frac 1{t^n} \phi(\frac\cdot{t}) = \phi_t$, $\int \phi = 1$
then 
\begin{align*}
  % \left\|\int f(\cdot - y) \phi_t(y) \ \mathrm{d}y - f\right\|_{L^1} \to 0 \\
  &= \int_x\int_y |f(x-y) - f(x)||\phi_t(y)| \ \mathrm{d}y \ \mathrm{d}x \\
  &= \int_{|y| \le \delta} |\phi_t(y)| \int_x |f(x-y) - f(x)| \ \mathrm{d}x \ \mathrm{d}y - \int_{|y| > \delta} |\phi_t(y)| \int |f(x-y) - f(x)| \ \mathrm{d}x \ \mathrm{d}y \\
  &\le \eps \|\phi\|_{L^1} + \|f\|_12\int_{|y| > \delta} |\phi_t(y)| \ \mathrm{d}y
\end{align*}

For pointwise convergence, we would have to prove an estimate 
\[\sup_t |\phi_t * f(x)|\le M_{\mathrm{HL}}f(x)\]
