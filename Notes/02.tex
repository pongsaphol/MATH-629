\chapter{Measures}
\section{Introduction}

We define the $\ell([c, d]) = d-c$ and If $E = [c_1, d_1] \cup[c_2, d_2]$ 
where $d_1 < c_2$ then $\ell(E) = d_1 - c_1 + d_2 - c_2$.
This is consistent with the definition
\[\ell(E) = \int \mathbbm{1}_E(x) \ \mathrm{d}x\]
where the integral denotes the Riemann integral.

if $E \subseteq[a, b]$ reference interval is 
\[\int_a^b \mathbbm{1}_E\ \mathrm{d}x\]

\begin{remark}
  The consistency of the definition also works with the set $(c, d)$, $[c, d)$, and $(c, d]$, 
  where the length of all of them is $d-c$.
\end{remark}

\begin{remark}
  we defnote $\mathbbm{1}_E$ to be such that
  \[
    \mathbbm{1}_E = 
  \begin{cases}
    1 & \text{if } x \in E\\
    0 & \text{if } x \notin E 
  \end{cases}
  \]
\end{remark}

\begin{example}
  Let $f(x) = \mathbbm{1}_{\QQ}(x)$ defined on the $[0, 1]$. 
  Then $U(f, P) = 1$ and $L(f, P) = 0$ for any partition $P$. 
\end{example}



Fix the reference interval $[a, b]$ and consider subset of $[a, b]$

Let $\mathcal{A}=$ collection of sets for which $\int_{[a, b]} \mathbbm{1}_E\ \mathrm{d}x$ exists.

If $A_1, \dotsc, A_n \in \mathcal{A}$, we can make the set to be mutually disjoint by taking 
$E_1 = A_1$, $E_2 = A_2 \setminus A_1$, $E_3 = A_3 \setminus (A_1 \cup A_2)$, and so on.

\begin{example}
  For $E_1, E_2 \in \mathcal{A}$, we have
  \[\mathbbm{1}_{E_1 \cap E_2}(x) = \mathbbm{1}_{E_1}(x)\mathbbm{1}_{E_2}(x)\]
\end{example}

\begin{example}
  For the Riemann integral, we have
  \[\int_a^b f(y) = \int_{a-v}^{b-v} f(v+y)\]
  and we want
  \[\int \mathbbm{1}_E(x)\ \mathrm{d}x = \int \mathbbm{1}_{v+E}\]
  where $v+E = \{v+x: x\in E\}$
\end{example}

Let $E = \QQ \cap [0, 1]$ countable set, we can enumerate
$r_1, r_2, r_3, \dotsc$ such that 
\[E = \bigcup_{n=1}^{\infty}\{r_n\}\] and 
\[\int \mathbbm{1}_{\{r_k\}} = 0\]
$E$ should have length zero but according $\mathbbm{1}_E$ is not 
Riemann integrable.

\section{Construction of Measure}

Suppose that $\mathcal{C}$ be a collection of sets.

Can we define on suitable large collection of subset of $\RR$?

a set function $\mu: \mathcal{C} \to [0, \infty) \cup \{\infty\}$ such that
if $\{E_j\}_{j=1}^\infty$ is a sequence of disjoint set in $\mathcal{C}$ then
\[\bigcup E_j = \mathcal{C}\]

\[\mu\left(\bigcup_{i=1}^\infty E_j\right) = \sum_{j=1}^\infty \mu(E_j)\]

$\mu([a, b]) = b-a$, $\mu([0, 1]) = 1$

Can we do this for the collection of all subset of $\RR$?

Answer: No, Vitali set.

\begin{theorem}
  We cannot define a measure on the collection of all subset of $\RR$.
\end{theorem}

Before we prove that theorem, we need to define something and prove the following lemma.

\begin{definition}
  We define a Vitali set $V$ from picking an element $x \in [0, 1)$ from each equivalence class of the relation $x \sim y$ if $x - y \in \QQ$.
  (e.g, pick $x \in O_x$ for $O_x \in \RR/\QQ$)
\end{definition}

\begin{lemma}\label{lem:vitali}
  Suppose that $V$ is a Vitali set then 
  \[V \cap V+q = \emptyset\] 
  For all $q \in \QQ \setminus \{0\}$
\end{lemma}

\begin{proof}
  Suppose not, there exists $a \in V$ such that $a \in V + q\implies a-q \in V$ 
  but we only pick 1 element in each equivalence class. contradiction.  
\end{proof}

\begin{lemma}\label{lem:VW}
  Let $V$ be a Vitali set and let $W = \{q \in [-1, 1] : q \in \QQ\}$ and 
  \[ E = \bigcup_{w \in W}V + w\]
   then 
  \[[0, 1] \subseteq E \subseteq [-1, 2]\]
\end{lemma}

\begin{proof}
  Consider $E \subseteq [-1, 2]$. Since $V \subseteq [0, 1)$, then
  for any $v \in V$, $v \in [0, 1) \implies v + w \in [-1, 2]$.

  For the $[0, 1] \subseteq E$,
  for any $x \in [0, 1]$ there exists $O_x \in \RR/\QQ$ such that $x \in O_x$.
  then there exists $v \in C_x$ such that $v \in [0, 1)$ and $v \in V$, since both
  are from the same equivalence class, then $x - v \in \QQ$ and $|x - v| < 1 \implies x - v \in (-1, 1)$.
  Hence, there exists $w \in W$ such that $w = x-v$ so $v+w = x$.
\end{proof}


\begin{proof}[Proof of the theorem]
  Suppose that $\mu$ exists then using the result from Lemma~\ref{lem:VW} we get that  
  \[ \mu([0, 1]) \leq \mu\left(E\right) \leq \mu([-1, 2])\]
  from Lemma~\ref{lem:vitali} we know that each $V+w$ is disjoint, so
  \begin{align*}
    \mu([0, 1]) &\leq \sum_{w \in W}\mu(V) \leq \mu([-1, 2])\\
    1 &\leq \sum_{w \in W}\mu(V) \leq 3
  \end{align*}
  if $\mu(V) = 0$ then $\mu(E) = 0$ and if $\mu(V) > 0$ then $\mu(E) = \infty$. Both are contradiction. 
  % Assume $\mu$ exists then for $E \subseteq F$ then $\mu(E) \le \mu(F)$.
  % From the disjoint, we get $\mu(F) = \mu(E) + \mu(F\setminus E)$. 
  % Now, we define a special set $E$. We consider an equivalence relation 
  % on $\RR$, saying $x\sim y$, if $x-y \in \QQ$. Then $\RR$ is a disjoint union
  % of equivalence classes. We can form a set $E$ with the property that eachequivalence class has exactly one member in $E$
  % (and that member belongs to $[0, 1)$).
  % Let $r_1, r_2, \dotsc$ be an enumeration of the rational numbers in $[-1, 1]$.
  % let $A = \cup_{k=1}^\infty(r_k + E)$, then $A \subseteq [-1, 2] = [-1, 0] \cup [0, 1] \cup [1, 2] \implies \mu(A) \le 3$.
  % Claim: $A \supset [0, 1] \implies \mu(A) \ge 1$. Pick $x \in [0, 1)$, $x \sim w$, 
  % $w \in E \cap [0, 1)$ so $x - w \in [-1, 1]$ is rational.
  % $A = \cup_+ (R_k + E)$ If $y = r_k + w_k = r_k + w_l\in E \implies w_k \sim w_l \implies k=l$
  % (because $E$ has exactly one member in eqch equivalence class). then 
  % \[\mu\left(\bigcup_+ (r_k + E)\right) = \sum_{k = 1}^\infty \mu(r_k + E) = \sum_{k = 1}^\infty \mu(E)\]
  % If $\mu(E) = 0$ then $\mu(A) = 0$.
  % If $\mu(E) > 0$ then $\mu(A) = \infty$.
\end{proof}

\section{$\sigma$-algebra}

\begin{definition}
  Given a reference $X$. An \textbf{algebra} is a collection of subsets of $X$, $\mathcal{A}$, such that
  \begin{enumerate}[(i)]
    \item $X \in \mathcal{A}$
    \item If $A \in \mathcal{A}$ then the complement $A^c = X \setminus A \in \mathcal{A}$
    \item If $A, B \in \mathcal{A}$ then $A \cup B \in \mathcal{A}$
  \end{enumerate}
\end{definition}

\begin{remark}
  \begin{itemize}
    \item $\emptyset \in \mathcal{A}$ because $\emptyset = X^c$
    \item $A_1, A_2 \in \mathcal{A}$, $A_1 \setminus A_2 = A_1 \cap A_2^c \in \mathcal{A}$
    \item Observe that if $A_1, A_2 \in \mathcal{A}$ then $A_1 \cap A_2 \in \mathcal{A}$ because $(A_1 \cap A_2)^c = A_1^c \cup A_2^c$
  \end{itemize}
\end{remark}

\begin{example}
  $X = [a, b]$ and $\mathcal{A}$ is the collection of all sets $E \subseteq [a, b]$  
  such that the Riemann integral 
  $\int x_E(t) \ \mathrm{d}t$
 exists
\end{example}


\begin{definition}
  A $\sigma$-algebra $\mathcal{M}$ on $X$ is 
  \begin{enumerate}[(i)]
    \item an algebra of subsets of $X$
    \item If $A_1, A_2, A_3, \dotsc$ is a sequence of set in $\mathcal{M}$ then
    \[\bigcup_{j = 1}^\infty A_j \in \mathcal{M}\]
  \end{enumerate}
  $(X, \mathcal{M})$ is called a ``\textbf{measurable space}''.
\end{definition}

\begin{remark}
  $\mathcal{M}$ is a $\sigma$-algebra then it satisfies
  \begin{enumerate}[(i)] 
    \item $X \in \mathcal{M}$
    \item If $A \in \mathcal{M}$ then $A^c \in \mathcal{M}$
    \item countable union of sets in $\mathcal{M}$ is in $\mathcal{M}$ 
  \end{enumerate}
\end{remark}

\begin{definition}
Let $(X, \mathcal{M})$ be a measurable set. Then a measure $\mu$ is a set function 
$\mu: \mathcal{M} \to [0, \infty], E \mapsto \mu(E)$ such that
\begin{enumerate}[(i)]
  \item $\mu(\emptyset) = 0$
  \item If $E_1, E_2, E_3, \dotsc$ is a sequence of disjoint set in $\mathcal{M}$ then
  \[\mu\left(\bigcup_{j=1}^\infty E_j\right) = \sum_{j = 1}^\infty \mu(E_j)\]
  called $\sigma$-additivity.
\end{enumerate}
$(X, \mathcal{M}, \mu)$ is called a ``\textbf{measure space}''.
\end{definition}

\begin{remark}
  \begin{align*}
    \left(\bigcap_{j=1}^\infty A_j\right) &= \left(\bigcup_{j=1}^\infty A_j^C\right)^C \in \mathcal{M}
  \end{align*} 
\end{remark}
\begin{example}
  examples of $\sigma$-algebra
  \begin{enumerate}[(i)]
    \item $\mathcal{M} = \{\emptyset, X\}$ 
    \item $\mathcal{M} = \mathcal{P}(X) = $ collection of all subsets of $X$
    \item[] $\NN = \{1, 2, 3, \dotsc\}$ and $\mu(E) = |E|$ (the cardinality of $E$) if $E$ is finite and 
    $\mu(E) = \infty$ if $E$ is infinite.
    \item $X$ write $X$ as a disjoint (countable) union of sets $A_j$. Then
    $\mathcal{M} = $ all countable unions of $A_j$. 
    \item Let $X$ be a set. Let $\mathcal{M}$ be the collection of all sets $A$, $A \subseteq X$ such that
    $A$ is countable or $A^C$ is countable.
    \item $X = \RR$ (or $\RR^n$), $\mathcal{B}_\RR$ is the smallest $\sigma$-algebra containing all open sets.
  \end{enumerate}
\end{example}

More generally if $\mathcal{E}$ is a collection of subsets of $X$ then $\mathfrak{M}(\mathcal{E})$ is the smallest
$\sigma$-algebra that contains all sets in $\mathcal{E}$.

If $\mathcal{M}_1, \mathcal{M}_2$ are two $\sigma$-algebras, then $\mathcal{M}_1 \cap \mathcal{M}_2$ is also a 
$\sigma$-algebra.

If $\{\mathcal{M}_\alpha\}_{\alpha}$ is a collection of $\sigma$-algebras, their intersection is also a $\sigma$-algebra.

\begin{definition}
$\mathfrak{M}(\mathcal{E})$ = intersection of all $\sigma$-algebra that contain the collection $\mathcal{E}$
We call it the $\sigma$-algebra generated by $\mathcal{E}$.
\end{definition}

\begin{remark}
If $\mathcal{E}\subset \mathcal{F} \implies \mathfrak{M}(\mathcal{E}) \subset \mathfrak{M}(\mathcal{F})$
\end{remark}
\begin{lemma}
  If $\mathcal{E} \subseteq \mathfrak{M}(\mathcal{F})$ then $\mathfrak{M}(\mathcal{E}) \subseteq \mathfrak{M}(\mathcal{F})$
\end{lemma}
\begin{proof}
  $\mathfrak{M}(\mathcal{F})$ is a $\sigma$-algebra that contains $\mathcal{E}$
  It contains the intersection of all $\sigma$-algebras which contain $\mathcal{E}$
\end{proof}

\begin{example}
  $\mathcal{B}_\RR$ = $\sigma$-algebra on $\RR$ containing all open sets 
  $\mathcal{E}$ a collection of all open intervals, 
  $\mathcal{E}\subseteq\mathcal{O}$ = collection of all open sets in $\RR$,
  $\mathcal{B}_\RR = \mathfrak{M}(\mathcal{O})$. $\mathfrak{M}(\mathcal{E}) \subseteq \mathcal{B}_\RR$.
  Each open set is a countable union of open intervals. Each open set is contained in $\mathfrak{M}(\mathcal{E})$.
  
  Since $\mathcal{O} \subseteq \mathfrak{M}(\mathcal{E}) \implies \mathfrak{M}(\mathcal{O}) \subseteq \mathfrak{M}(\mathcal{E})$.
  get $\mathfrak{M}(\mathcal{O}) = \mathfrak{M}(\mathcal{E})$.
\end{example}

Lebesgue-Borel measure,
$\mu((a, b)) = b-a$, $\mu$ is defined on $\mathcal{B}_\RR$.