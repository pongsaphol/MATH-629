\chapter{Measures}
\section{Introduction}

We define the $\ell([c, d]) = d-c$ and If $E = [c_1, d_1] \cup[c_2, d_2]$ 
where $d_1 < c_2$ then $\ell(E) = d_1 - c_1 + d_2 - c_2$.
This is consistent with the definition
\[\ell(E) = \int \mathbbm{1}_E(x) \ \mathrm{d}x\]
where the integral denotes the Riemann integral.

if $E \subseteq[a, b]$ reference interval is 
\[\int_a^b \mathbbm{1}_E\ \mathrm{d}x\]

\begin{remark}
  The consistency of the definition also works with the set $(c, d)$, $[c, d)$, and $(c, d]$, 
  where the length of all of them is $d-c$.
\end{remark}

\begin{remark}
  we defnote $\mathbbm{1}_E$ to be 
  \[
    \mathbbm{1}_E(x) = 
  \begin{cases}
    1 & \text{if } x \in E\\
    0 & \text{if } x \notin E 
  \end{cases}
  \]
\end{remark}

\begin{example}
  Let $f(x) = \mathbbm{1}_{\QQ}(x)$ defined on the $[0, 1]$. 
  Then $U(f, P) = 1$ and $L(f, P) = 0$ for any partition $P$. 
\end{example}



Fix the reference interval $[a, b]$ and consider subset of $[a, b]$

Let $\mathcal{A}=$ collection of sets for which $\int_{[a, b]} \mathbbm{1}_E\ \mathrm{d}x$ exists.

If $A_1, \dotsc, A_n \in \mathcal{A}$, we can make the set to be mutually disjoint by taking 
$E_1 = A_1$, $E_2 = A_2 \setminus A_1$, $E_3 = A_3 \setminus (A_1 \cup A_2)$, and so on.

\begin{example}
  For $E_1, E_2 \in \mathcal{A}$, we have
  \[\mathbbm{1}_{E_1 \cap E_2}(x) = \mathbbm{1}_{E_1}(x)\mathbbm{1}_{E_2}(x)\]
\end{example}

\begin{example}
  For the Riemann integral, we have
  \[\int_a^b f(y) = \int_{a-v}^{b-v} f(v+y)\]
  and we want
  \[\int \mathbbm{1}_E(x)\ \mathrm{d}x = \int \mathbbm{1}_{v+E}\]
  where $v+E = \{v+x: x\in E\}$
\end{example}

Let $E = \QQ \cap [0, 1]$ countable set, we can enumerate
$r_1, r_2, r_3, \dotsc$ such that 
\[E = \bigcup_{n=1}^{\infty}\{r_n\}\] and 
\[\int \mathbbm{1}_{\{r_k\}} = 0\]
$E$ should have length zero but according $\mathbbm{1}_E$ is not 
Riemann integrable.

\section{Construction of Measure}

Suppose that $\mathcal{C}$ be a collection of sets.

Can we define on suitable large collection of subset of $\RR$?

a set function $\mu: \mathcal{C} \to [0, \infty]$ such that
if $\{E_j\}_{j=1}^\infty$ is a sequence of disjoint set in $\mathcal{C}$ then
% \[\bigcup E_j = \mathcal{C}\]
\[\mu\left(\bigcup_{i=1}^\infty E_j\right) = \sum_{j=1}^\infty \mu(E_j)\]

$\mu([a, b]) = b-a$, $\mu([0, 1)) = 1$

Can we do this for the collection of all subset of $\RR$?

Answer: No, Vitali set.

\begin{theorem}
  We cannot define a measure on the collection of all subset of $\RR$. i.e.,
  there does not exist a set function $\mu: \kP(\RR) \to [0, \infty]$ such that
  \begin{enumerate}[(i)]
    \item $\mu(v + E) = \mu(E)$ for all $E \subseteq \RR$ and $v \in \RR$
    \item $\mu([0, 1]) = 1$
    \item $\mu\left(\bigcup_{j=1}^\infty A_j\right) = \sum_{j=1}^\infty \mu(A_j)$ for all disjoint $A_j \subseteq \RR$
  \end{enumerate}
\end{theorem}

Before we prove that theorem, we need to define something and prove the following lemma.

\begin{definition}
  We define a Vitali set $V$ from picking an element $x \in [0, 1)$ from each equivalence class of the relation $x \sim y$ if $x - y \in \QQ$.
  (e.g, pick $x \in O_x$ for $O_x \in \RR/\QQ$)
\end{definition}

\begin{lemma}\label{lem:vitali}
  Suppose that $V$ is a Vitali set then 
  \[V \cap V+q = \emptyset\] 
  For all $q \in \QQ \setminus \{0\}$
\end{lemma}

\begin{proof}
  Suppose not, there exists $a \in V$ such that $a \in V + q\implies a-q \in V$ 
  but we only pick 1 element in each equivalence class. contradiction.  
\end{proof}

\begin{lemma}\label{lem:VW}
  Let $V$ be a Vitali set and let $W = \{q \in [-1, 1] : q \in \QQ\}$ and 
  \[ E = \bigcup_{w \in W}V + w\]
   then 
  \[[0, 1] \subseteq E \subseteq [-1, 2]\]
\end{lemma}

\begin{proof}
  Consider $E \subseteq [-1, 2]$. Since $V \subseteq [0, 1)$, then
  for any $v \in V$, $v \in [0, 1) \implies v + w \in [-1, 2]$.

  For the $[0, 1] \subseteq E$,
  for any $x \in [0, 1]$ there exists $O_x \in \RR/\QQ$ such that $x \in O_x$.
  then there exists $v \in C_x$ such that $v \in [0, 1)$ and $v \in V$, since both
  are from the same equivalence class, then $x - v \in \QQ$ and $|x - v| < 1 \implies x - v \in (-1, 1)$.
  Hence, there exists $w \in W$ such that $w = x-v$ so $v+w = x$.
\end{proof}


\begin{proof}[Proof of the theorem]
  Suppose that $\mu$ exists then using the result from Lemma~\ref{lem:VW} we get that  
  \[ \mu([0, 1]) \leq \mu\left(E\right) \leq \mu([-1, 2])\]
  from Lemma~\ref{lem:vitali} we know that each $V+w$ is disjoint, so
  \begin{align*}
    \mu([0, 1]) &\leq \sum_{w \in W}\mu(V) \leq \mu([-1, 2])\\
    1 &\leq \sum_{w \in W}\mu(V) \leq 3
  \end{align*}
  if $\mu(V) = 0$ then $\mu(E) = 0$ and if $\mu(V) > 0$ then $\mu(E) = \infty$. Both are contradiction. 
  % Assume $\mu$ exists then for $E \subseteq F$ then $\mu(E) \le \mu(F)$.
  % From the disjoint, we get $\mu(F) = \mu(E) + \mu(F\setminus E)$. 
  % Now, we define a special set $E$. We consider an equivalence relation 
  % on $\RR$, saying $x\sim y$, if $x-y \in \QQ$. Then $\RR$ is a disjoint union
  % of equivalence classes. We can form a set $E$ with the property that eachequivalence class has exactly one member in $E$
  % (and that member belongs to $[0, 1)$).
  % Let $r_1, r_2, \dotsc$ be an enumeration of the rational numbers in $[-1, 1]$.
  % let $A = \cup_{k=1}^\infty(r_k + E)$, then $A \subseteq [-1, 2] = [-1, 0] \cup [0, 1] \cup [1, 2] \implies \mu(A) \le 3$.
  % Claim: $A \supset [0, 1] \implies \mu(A) \ge 1$. Pick $x \in [0, 1)$, $x \sim w$, 
  % $w \in E \cap [0, 1)$ so $x - w \in [-1, 1]$ is rational.
  % $A = \cup_+ (R_k + E)$ If $y = r_k + w_k = r_k + w_l\in E \implies w_k \sim w_l \implies k=l$
  % (because $E$ has exactly one member in eqch equivalence class). then 
  % \[\mu\left(\bigcup_+ (r_k + E)\right) = \sum_{k = 1}^\infty \mu(r_k + E) = \sum_{k = 1}^\infty \mu(E)\]
  % If $\mu(E) = 0$ then $\mu(A) = 0$.
  % If $\mu(E) > 0$ then $\mu(A) = \infty$.
\end{proof}

\section{$\sigma$-algebra}

\begin{definition}
  Given a reference $X$. An \textbf{algebra} is a collection of subsets of $X$, $\mathcal{A}$, such that
  \begin{enumerate}[(i)]
    \item $X \in \mathcal{A}$
    \item If $A \in \mathcal{A}$ then the complement $A^\complement = X \setminus A \in \mathcal{A}$
    \item If $A, B \in \mathcal{A}$ then $A \cup B \in \mathcal{A}$
  \end{enumerate}
\end{definition}

\begin{remark}
  \begin{itemize}
    \item $\emptyset \in \mathcal{A}$ because $\emptyset = X^\complement$
    \item $A_1, A_2 \in \mathcal{A}$, $A_1 \setminus A_2 = A_1 \cap A_2^\complement \in \mathcal{A}$
    \item Observe that if $A_1, A_2 \in \mathcal{A}$ then $A_1 \cap A_2 \in \mathcal{A}$ because $(A_1 \cap A_2)^\complement = A_1^\complement \cup A_2^\complement$
  \end{itemize}
\end{remark}

\begin{example}
  $X = [a, b]$ and $\mathcal{A}$ is the collection of all sets $E \subseteq [a, b]$  
  such that the Riemann integral 
  $\int \mathbbm{1}_E(t) \ \mathrm{d}t$
 exists
\end{example}


\begin{definition}
  A $\sigma$-algebra $\SM$ on $X$ is 
  \begin{enumerate}[(i)]
    \item an algebra of subsets of $X$
    \item If $A_1, A_2, A_3, \dotsc$ is a sequence of set in $\SM$ then
    \[\bigcup_{j = 1}^\infty A_j \in \SM\]
  \end{enumerate}
  $(X, \SM)$ is called a ``\textbf{measurable space}''.
\end{definition}

\begin{remark}
  $\SM$ is a $\sigma$-algebra on $X$ then it satisfies
  \begin{enumerate}[(i)] 
    \item $X \in \SM$
    \item If $A \in \SM$ then $A^\complement \in \SM$
    \item countable union of sets in $\SM$ is in $\SM$ 
  \end{enumerate}
\end{remark}

\begin{definition}
Let $(X, \SM)$ be a measurable set. Then a measure $\mu$ is a set function 
$\mu: \SM \to [0, \infty], E \mapsto \mu(E)$ such that
\begin{enumerate}[(i)]
  \item $\mu(\emptyset) = 0$
  \item If $E_1, E_2, E_3, \dotsc$ is a sequence of disjoint set in $\SM$ then
  \[\mu\left(\bigcup_{j=1}^\infty E_j\right) = \sum_{j = 1}^\infty \mu(E_j)\]
  called $\sigma$-additivity.
\end{enumerate}
$(X, \SM, \mu)$ is called a ``\textbf{measure space}''.
\end{definition}

\begin{remark}
  \begin{align*}
    \left(\bigcap_{j=1}^\infty A_j\right) &= \left(\bigcup_{j=1}^\infty A_j^\complement\right)^\complement \in \SM
  \end{align*} 
\end{remark}
\begin{example}
  examples of $\sigma$-algebra
  \begin{enumerate}[(i)]
    \item $\SM = \{\emptyset, X\}$ 
    \item $\SM = \kP(X) = $ collection of all subsets of $X$
    \item[] $\NN = \{1, 2, 3, \dotsc\}$ and $\mu(E) = |E|$ (the cardinality of $E$) if $E$ is finite and 
    $\mu(E) = \infty$ if $E$ is infinite.
    \item $X$ write $X$ as a disjoint (countable) union of sets $A_j$. Then
    $\SM = $ all countable unions of $A_j$. 
    \item Let $X$ be a set. Let $\SM$ be the collection of all sets $A$, $A \subseteq X$ such that
    $A$ is countable or $A^\complement$ is countable.
    \item $X = \RR$ (or $\RR^n$), $\SB_\RR$ is the smallest $\sigma$-algebra containing all open sets.
  \end{enumerate}
\end{example}

More generally if $\SE$ is a collection of subsets of $X$ then $\kM(\SE)$ is the smallest
$\sigma$-algebra that contains all sets in $\SE$.

If $\SM_1, \SM_2$ are two $\sigma$-algebras, then $\SM_1 \cap \SM_2$ is also a 
$\sigma$-algebra.

If $\{\SM_\alpha\}_{\alpha \in \mathcal{I}}$ is a collection of $\sigma$-algebras, their intersection is also a $\sigma$-algebra.

\subsection{Generating $\sigma$-algebra}

\begin{definition}
$\kM(\SE)$ := intersection of all $\sigma$-algebra that contain the collection $\SE$
We call it the $\sigma$-algebra generated by $\SE$.
\end{definition}

\begin{remark}
If $\SE\subset \SF \implies \kM(\SE) \subset \kM(\SF)$
\end{remark}
\begin{lemma}
  If $\SE \subseteq \kM(\SF)$ then $\kM(\SE) \subseteq \kM(\SF)$
\end{lemma}
\begin{proof}
  $\kM(\SF)$ is a $\sigma$-algebra that contains $\SE$
  It contains the intersection of all $\sigma$-algebras which contain $\SE$
\end{proof}

\begin{example}
  $\SB_\RR$ = $\sigma$-algebra on $\RR$ containing all open sets 
  $\SE$ a collection of all open intervals, 
  $\SE\subseteq\SO$ = collection of all open sets in $\RR$,
  $\SB_\RR = \kM(\SO)$. $\kM(\SE) \subseteq \SB_\RR$.
  Each open set is a countable union of open intervals. Each open set is contained in $\kM(\SE)$.
  
  Since $\SO \subseteq \kM(\SE) \implies \kM(\SO) \subseteq \kM(\SE)$.
  get $\kM(\SO) = \kM(\SE)$.
\end{example}

% TODO: maybe a definition of Lebesgue-Borel measure
% Lebesgue-Borel measure,
% $\mu((a, b)) = b-a$, $\mu$ is defined on $\SB_\RR$.

\begin{definition}
  Given $(X_1, \SM_1), (X_2, \SM_2), \dotsc, (X_n, \SM_n)$ measurable spaces.
  Define a ``product $\sigma$-algebra'' on $X_1 \times X_2 \times \dotsm \times X_n$ denoted by
  \[\SM_1 \oplus \SM_2\oplus \dotsm \oplus \SM_n = \bigoplus_{j=1}^n \SM_j\]
  defined as the $\sigma$-algebra generated by the sets $E_1 \times E_2 \times \dotsm \times E_n$ where $E_j \in \SM_j$.
  
  i.e., define $\SE := \{(E_1 \times E_2 \times \dotsm \times E_n) : E_j \in \SM_j\}$ then 
  \[ \bigoplus_{j=1}^n \SM_j := \kM(\SE)\]
\end{definition}

\begin{remark}
  Folland defines it the $\sigma$-algebra generated by 
  \[(X_1 \times X_2 \times \dotsm \times X_{n-1} \times E_n)\]
  where $E_n \in \SM_n$, 
  \[(X_1 \times X_2 \times \dotsm E_{n-1} \times X_n)\]
  where $E_{n-1} \in \SM_{n-1}$.
  and so on. To be clear, let 
  \[\SE' := \bigcup_{j=1}^n \{ (X_1 \times \dotsm\times  X_{j-1} \times E_j \times X_{j+1} \times \dotsm \times X_n) : E_j \in \SM_j\}\]
  then 
  \[\bigoplus_{j=1}^n \SM_j := \kM(\SE') \]
\end{remark}

\begin{claim}
  Both defintions on product of $\sigma$-algebra are equivalent. 
\end{claim}

\begin{proof}
  The goal is to show that $\kM(\SE) = \kM(\SE')$.
  \begin{itemize}
    \item[($\supseteq$)] Obviously, $\SE' \subseteq \SE$ so $\kM(\SE') \subseteq \kM(\SE)$.
    \item[($\subseteq$)] We want to show that $\SE \subseteq \kM(\SE')$. Fix $(E_1 \times E_2 \times \dotsm \times E_n) \in \SE$ then 
    from the definition of $\sigma$-algebra generated by a collection, which is closed under intersection, so we can pick an element from the 
    construction of $\SE'$ and do the intersection, so $(E_1 \times E_2 \dotsm \times E_n) \in \kM(\SE)$.

  \end{itemize}
\end{proof}

\begin{theorem}
  Given $(X_1, \SM_1), (X_2, \SM_2)$ measurable spaces. Assume that  
  $\SM_1$ is generated by a collection $\SE_1$ and $\SM_2$ is generated by a collection $\SE_2$.
  Then $\SM_1 \oplus \SM_2$ is generated by the sets $E_1 \times X_2$, $X_1 \times E_2$, where $E_1 \in \SE_1$ and $E_2 \in \SE_2$.
\end{theorem}
\begin{proof}
  Let $\SP := \{E_1 \times E_2 : E_i \in \SE_i\}$, obviously $\kM(\SP) = \kM(\{E_1\times X_2 : E_1 \in \SE_1\}\cup\{X_1 \times E_2 : E_2 \in \SE_2\})$ and $\kM(\SP) \subseteq \SM_1 \oplus \SM_2$.
  We need to show that $\SM_1 \oplus \SM_2 \subseteq \kM(\SP)$. Define 
  $$\SG_1 = \{E_1 \subseteq X_1 : E_1 \times X_2 \in \kM(\SP)\}$$
  $$\SG_2 = \{E_2 \subseteq X_2 : X_1 \times E_2 \in \kM(\SP)\}$$
  then $\SG_1$ is a $\sigma$-algebra consisting of subset of $X_1$ which contains $\SE_1$, $\SE_1\subseteq \SG_1$.
  $\SE_1$ generates $\SM_1$ so $\kM(\SE_1) = \SM_1 \subseteq \SG_1$.
  So, we have $E_1 \times X_2 \in \kM(\SP)$ for all $E_1 \in \SM_1$ and $X_1 \times E_2 \in \kM(\SP)$ for all $E_2 \in \SM_2$.
  The $\sigma$-algebra generated by the sets $E_1 \times X_2$, $X_1 \times E_2$ is contained $\SM_1 \oplus \SM_2 \in \kM(\SP)$.
\end{proof}

\begin{claim}
  $\SB_\RR \oplus \SB_\RR = \SB_{\RR^2}$.

  where $\SB_\RR \oplus \SB_\RR$ is generated by $E_1 \times E_2$, where $E_1, E_2 \in \SB_\RR$.
  and $\SB_{\RR^2}$ is generated by the open sets in $\RR^2$. 
\end{claim}

\begin{proof}
  $\SB_\RR \oplus \SB_\RR \subseteq \SB_{\RR^2}$. Want $\SB_{\RR^2} \subseteq \SB_\RR \oplus \SB_\RR$.
  Consider the collection of all open rectangle of the form $(a_1, b_1) \times (a_2, b_2)$ such $a_i, b_i \in \QQ$.
  which are contained in $O \subseteq \RR^2$
\end{proof}

\begin{definition}[The Borel $\sigma$ algebra on the extended real line]
We use the notion $\ol{\RR} = \RR \cup \{-\infty, \infty\} = [-\infty, \infty]$.
One possibility to define ``$\SB_{\ol{\RR}}$'' is the $\sigma$-algebra generated by open sets in $\RR, \{\infty\}, \{-\infty\}$
open intervals should be $(a, b), (a, \infty], [-\infty, b)$ for $-\infty \leq a < b \leq \infty$. Then define $d(x, y) = |\arctan(x) - \arctan(y)|$ and 
$\arctan(\infty) = \pi/2$, $\arctan(-\infty) = -\pi/2$.
\end{definition}

\section{Measures}

\begin{definition}
Measures are $\sigma$-additive set functions, $\mu(\emptyset) = 0$ and
\[\mu\left(\biguplus_{j=1}^\infty E_j\right) = \sum_{j=1}^\infty \mu(E_j)\]
where $E_1, E_2, \dotsc$ is a sequence of disjoint sets.
\end{definition}

\begin{remark}
There is some property   

$E \subseteq F \implies \mu(E) \leq \mu(F)$ 

$F = E \uplus (F \setminus E) \implies \mu(F) = \mu(E) + \mu(F \setminus E)$

$\mu(\bigcup A_j) \leq \sum \mu(A_j)$
we can write $\bigcup A_j$ as a disjoint union, i.e.,
$E_1 = A_1$, $E_2 = A_2 \setminus A_1$, $E_3 = A_3 \setminus (A_1 \cup A_2)$, and so on then
$\mu(\bigcup A_j) = \mu(\bigcup E_j) = \sum \mu(E_j) \leq \mu(A_j)$
\end{remark}

The monotone convergence theorem for sets (continuity from below)

\begin{theorem}\label{lem:continuity-from-below}
  
If $E_1 \subseteq E_2 \subseteq E_3 \subseteq \dotsm$ then
\[\mu\left(\bigcup_{j=1}^\infty E_j\right) = \lim_{j\to\infty} \mu(E_j)\]
\end{theorem}

\begin{proof}
 % TODO: add the proof 
$$\bigcup_{j=1}^\infty E_j = E_1 \cup (E_2 \setminus E_1) \cup (E_3 \setminus (E_1 \cup E_2)) \cup \dotsm$$
So, we define $B_1 = E_1, B_n = E_n \setminus E_{n-1}$ for $n \ge 2$ then all $B_j$ are disjoint.
\begin{align*}
  \bigcup_{j=1}^\infty E_j &= \bigcup_{j=1}^\infty B_j \\
  \mu\left(\bigcup_{j=1}^\infty E_j\right)&= \mu\left(\bigcup_{j=1}^\infty B_j\right) \\
  &= \sum_{j=1}^\infty \mu(B_j) \\
  &= \mu(E_1) + \sum_{j=2}^\infty \mu(E_j \setminus E_{j-1}) \\
  &= \mu(E_1) + \sum_{j=2}^\infty \mu(E_j) - \mu(E_{j-1}) \\
  &= \lim_{n\to\infty} \mu(E_n)
\end{align*}
\end{proof}
\begin{remark}
  If we prove something for the set then we can prove it for the complement.
  \[\mu(A) + \mu(A^\complement) = \mu(X)\]
\end{remark}

\begin{theorem}\label{lem:continuity-from-above}
  If $\mu(X) < \infty$ then if $E_1 \supseteq E_2 \supseteq E_3 \supseteq \dotsm$, $E_n \supseteq E_{n+1}$ for all $n$ then
  \[\mu\left(\bigcap_{j=1}^\infty E_j\right) = \lim_{j \to \infty} \mu(E_j)\] 
\end{theorem}

\begin{proof}
  Assume $E_j$ are decreasing, i.e.,
  \[E_1 \supseteq E_2 \supseteq E_3 \supseteq \dotsm \]
  then $E_1^\complement \subseteq E_2^\complement \subseteq \dotsm$ then
  \begin{align*}
  \mu\left(\bigcup_{j=1}^\infty E_j^\complement\right) &= \lim_{j \to \infty} \mu(E_j^\complement) \\
  \mu(X) - \mu\left(\left(\bigcup_{j=1}^\infty E_j^\complement\right)^\complement\right) &= \lim_{j \to \infty}( \mu(X) - \mu(E_j) )\\
  \mu(X) - \mu\left(\bigcap_{j=1}^\infty E_j\right) &= \lim_{j \to \infty} (\mu(X) - \mu(E_j)) \\
  \end{align*}
\end{proof}

\begin{example}
  $\NN$ with counting measure, $E_{j} = \{j, j+1, j+2, \dotsc\}$, $\mu(E_j) = \infty$,
  $\bigcap{E_j} = \emptyset$ has measure 0.
\end{example}

\begin{definition}
  If $A_1, A_2, A_3, \dotsc$ is an arbitrary sequence of measurable sets. 
  We can define 
  \[\limsup A_j := \bigcap_{n=1}^\infty \bigcup_{j \ge n} A_j = \{x : x \in A_n \text{ for infinitely many }n\}\]
  \[\liminf A_j := \bigcup_{n=1}^\infty \bigcap_{j \ge n} A_j = \{x : x \text{ belong to all but finitely many}\}\]
\end{definition}

\begin{lemma}[Borel-Cantelli Lemma]
  If $\{A_j\}$ is a sequence of measurable sets such that 
  \[\sum_{j=1}^\infty \mu(A_j) < \infty\]
  then almost every $x$ (meaning all $x$ except in a null set) belong to on $A_n$ for only finitely many $n$.
  Or equivalently, 
  \[\mu\left(\limsup A_n\right) = 0\]
\end{lemma}

\begin{proof}
  $\bigcup_{j \ge n} A_j$ are decreasing. In Borel Cantelli, we have $\sum \mu(A_j) < \infty$, so $\mu\left(\bigcup A_n\right) = 0$. 
  use ``continuity from above'' $$\mu(\limsup A_n) = \lim_{n \to \infty} \mu\left(\bigcup_{j \ge n} A_j\right)$$
  \[\mu\left(\bigcup_{j \ge n} A_j\right) \le \sum_{j \ge n} \mu(A_j) \to 0\]
  as $n \to \infty$.
\end{proof}

% some def, neet to relocate???

Completion of a $\sigma$-algebra (when a measure $\mu$ is given), $(X, \SM, \mu)$ $\overline{\SM}$ consists
of all unions $E \cup F$, where $E \in \SM$ and $F \subseteq N \in \SM$ for some null set $N$, $\mu(N) = 0$.

Define $\overline{\mu}$ by $\overline{\mu}(E \cup F) = \mu(E)$.

\section{Measurable Functions}


\begin{definition}
  $f: X \to Y$ where $(X, \SM)$ and $(Y, \SN)$ are measurable spaces.
  $f$ is $(\SM, \SN)$-measurable if for every $E \in \SN$, $f^{-1}(E) \in \SM$.
  where $f^{-1}(E) = \{x \in X : f(x) \in E\}$.
\end{definition}

\begin{lemma}
  Let $\SE$ generate $\SN$ (i.e., $\SN = \kM(\SE)$). 
  Then $f$ is $(\SM, \SN)$-measurable if and only if $f^{-1}(E) \in \SM$ for all $E \in \SE$. 
\end{lemma}

\begin{proof}
  Define $\SC = \{E : f^{-1}(E) \in \SM\}$, observe that $\SC$ is a $\sigma$-algebra. then
  \[f(x) = \bigcup E_j \iff x \in f^{-1}\left(\bigcup E_j\right) \iff x \in \bigcup f^{-1}(E_j) \iff \bigcup\{x  : f(x) \in E_j\}\]
\end{proof}

\begin{claim}
$f: X \to Y$ is $(\SM, \SN)$-measurable, $g: Y \to Z$ is $(\SN, \SR)$-measurable then $g \circ f: X \to Z$ is $(\SM, \SR)$-measurable.
\end{claim}
\begin{proof}
  $(g \circ f)^{-1}(E) = \{x \in X : g(f(x)) \in E\} = f^{-1}(g^{-1}(E)) = \{x \in X: f(x) \in g^{-1}(E)\}$
\end{proof}

vector-valued-function
$f: X \to (Y_1 \times Y_2 \times \dotsm \times Y_n)$ and defined by 
$x \mapsto (f_1(x), f_2(x), \dotsc, f_n(x))$ where $f_j: X \to Y_j$ is $(\SM, \SN_j)$-measurable.

Then $f$ is $(\SM, \SN_1, \SN_2, \dotsc,\SN_n)$ if and only if $f_i(\SM_i, \SN_i)$-measurable.
\begin{align*}
  f^{-1}(E_1 \times E_2 \times \dotsm \times E_n) &= f_1^{-1}(E_1) \cap f_2^{-1}(E_2) \cap \dotsm \cap f_n^{-1}(E_n) \\
  &= \bigcap_{j=1}^n f_j^{-1}(E_j)
\end{align*}

Let $f: X \to \RR$, $g: X \to \RR$ $\SM$ $\sigma$-algebra on $X$, $f, g$ are $\SM$-measurable
What about $f(x) + g(x)$

$M(x) = \max\{f(x), g(x)\}$, $f,g: X \to \RR$, $\SM$-measurable.
The collection $(a, \infty)$ generates the Borel $\sigma$-algebra on $\RR$.
$(a, \infty) \setminus (b, \infty) = (a, b]$ for $a < b$ and $(a, c) = \bigcup_n (a, c - \frac1n]$
$M^{-1}(a, \infty) = \{x : f(x) > a, g(x) > a\} = f^{-1}(a, \infty) \cap g^{-1}(a, \infty)$

$f_n: X \to \RR$, $\SM$-measurable, tehn $S = \sup_n f_n$ is $\SM$-measurable.
$S^{-1}(a, \infty) = \{x : \sup f_n(x) > a\} = \bigcup_n \{x : f_n(x) > a\} = \bigcup f_n^{-1}(a, \infty)$

Similar for $\min$ and $\inf$.

\begin{definition}
If $f_n: X \to \RR$, $\SM$-measurable then \[\limsup f_n = \inf_k \sup_{n \ge k} f_n\]
\[\liminf f_n = \sup_k \inf_{n \ge k} f_n\]
\end{definition}

\begin{claim}
  $\limsup f_n$ and $\liminf f_n$ are $\SM$-measurable.
\end{claim}

\begin{proof}
  $f_n$ has a pointwise limit if $\lim\limits_{n\to\infty}f_n(x)$ exists for all $x \in X$.
  $\{x \subseteq X : \lim f_n(x) \text{ exists }\} = \{x \subseteq X : \underbrace{\limsup f_n(x) - \liminf f_n(x)}_{D(x)} = 0\} = D^{-1}(\{0\})$
\end{proof}
