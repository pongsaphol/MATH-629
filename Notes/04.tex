\chapter{$L^p$ Spaces}

\section{normed spaces}
\begin{remark}
  If $f_n \to f$ almost everywhere, do we have $\int |f_n -f| \ \mathrm{d}\mu \to 0$?
  \begin{itemize}
    \item No, if $f_n = \mathbbm{1}_{[n, n+1]}$ then $f_n \to 0$ almost everywhere but $\int |f_n - 0| \ \mathrm{d}\mu = 1$ and $\int |f_n - f_m|\ \mathrm{d}\mu = 2$.
    \item No, if $f_n = \mathbbm{1}_{[0, \frac1n]}$
  \end{itemize}
\end{remark}

\begin{remark}
  Convergence in $L^1$ implies convergence almost everywhere?
  No, If $2^k \le n \le 2^{k+1}$ where $n = 2^k +j$, $j = 0, \dotsc, 2^k-1$
  $f_{2^k+1} = \mathbbm{1}_{[i 2^{-k}, (i+1)2^{-k}]}$ for $i = 0, \dotsc, 2^k-1$.
  For $2^k \le n \le 2^{k+1}$, $\|f_n\|_{L^1} = 2^{-k}$
\end{remark}

\begin{claim}
  If $f_n \to f$ in $L^1$ ($\int |f_n - f| \ \mathrm{d}\mu \to 0$) 
  then there is a subsequence $f_{n_k} \to f$ almost everywhere.
\end{claim}

\begin{proof}
  Consider the normed space $L^1$ space of semi-nromed space $\mathcal{L}^1$. (define as a equivalence class of almost everywhere where $f \underset{\text{a.e.}}\sim g$ if $f = g$ almost everywhere)
  Construct a convergence subsequence (a.e. and also in Norm)
  Choose $\eps = \frac1{2^k}$ there exists number $N(k)$ such that $\|f_l - f_m\| < \frac1{2^k}$ for $l, m \ge N(k)$ for $l, m \ge N(k)$
  then $\|f_{N(k)} - f_{N(k+1)}\| \le \frac1{2^k}\|$
  Define
  \[G(x) = |f_{N(1)}(x)| + \sum_{k=1}^\infty |f_{N(k+1)}(x) - f_{N(k)}(x)|\]
  then 
  \begin{align*}
    \int G(x)\ \mathrm{d}\mu &= \int|f_{N(1)}(x)|\ \mathrm{d}\mu + \sum_{k=1}^\infty \int |f_{N(k+1)}(x) - f_{N(k)}(x)|\ \mathrm{d}\mu
    &\le \|f_{N(1)}\|_1 + \sum_{k=1}^\infty \frac1{2^k} 
  \end{align*}
  So, $G$ is integrable, $\int |G(x)|\ \mathrm{d}\mu < \infty$ then $G(x) < \infty$ almost everywhere.
  We se that for almost everywhere, 
  \[f_{N_1}(x) + \sum_{k=1}^\infty f_{N(k+1)} - f_{N(k)}(x)\]
  converges for almost everywhere $x$, define
  \[s_M(x) = f_{N(1)}(x) + f_{N(2)}(x) - f_{N(1)}(x) + \dotsc + f_{N(M+1)}(x) - f_{N(M)}(x) = f_{N(M+1)}(x)\]
  then $s_{M-1}(x) = f_{N(M)}(x)$ and as $M \to \infty$, this is converges for almost everywhere $x$.

  $f(x) = \lim_{M\to\infty} f_{N(M)}$ 
  \begin{align*}
    f(x) &= f_{N_1}(x) + \sum_{M=1}^\infty f_{N(M+1)}(x) - f_{N(M)}(x) \\
    \int |f(x) - f_{N_1}(x)| \ \mathrm{d}\mu &= \int \sum_{M=1}^\infty f_{N(M+1)}(x) - f_{N(M)}(x) \ \mathrm{d}\mu \\
    &\le \int |f_{N(M+1)}(x) - f_{N(M)}(x)| + \int |f_{N(M+2)} - f_{N(M+1)}| + \dotsc \\
    &= \sum_{k=M}^\infty \int |f_{N(k+1)}(x) - f_{N(k)}(x)| \ \mathrm{d}\mu \\
    &\le 2^{1-M}
  \end{align*}
  This shows convergence of $f_{N(M)} \to f$ in $L^1$.
  What happens with $l \ge N(k)$, 
  \[\|f_l - f\| \le \|f_l - f_{N(k)}\| + \|f_{N(k)} - f\| \le \frac1{2^k}, \to 0\]
\end{proof}


$L^1$ or ($\mathcal{L}^1$) are complete, in the sense that every Cauchy sequence converges.
$\{f_n\}$ cxauchy, For every $\eps > 0$, there exists $N(\eps)$ such that for $l, m \ge N(\eps)$ then $\|f_l - f_m\| < \eps$

% \begin{proof}
% \end{proof}

% \begin{theorem}
  
% \end{theorem}

\section{Function space}

\begin{definition}
  \[\|f\|_p = \left(\int |f|^p \ \mathrm{d}\mu\right)^{\frac1p}\]
  where $L^P$ is space of equivalence class and $\mathcal{L}^p$ is space of functions,
  $f \in \mathcal{L}^p$ if $\|f\|_p < \infty$
\end{definition}

\begin{theorem}
  $\|f\|_p$ is a norm on $L^p$, if $p \ge 1$ (not a nrom if $p < 1$ because triangle inquality fails)
\end{theorem}
\begin{proof}
  for any $f, g \in L^p$, 
  \begin{align*}
    \int |f+g|^p \ \mathrm{d}\mu &\le \int (2 \max{|f|,|g|})^p \ \mathrm{d}\mu \\
    &= 2\left(\int \max |f|^p, |g|^p\right) \ \mathrm{d}\mu \\
    &\le 2\int |f|^p + |g|^p\ \mathrm{d}\mu 
  \end{align*}
\end{proof}

\begin{remark}
  $\|f+g\|_p \le 2^{\frac1p}(\|f\|_p + \|g\|_p)$
\end{remark}

\begin{theorem}
  For $p < 1$ we have inequality
  \[\|f+g\|_p^p \le \|f\|_p^p + \|g\|_p^p\]
\end{theorem}

\begin{proof}
  we claim that
  \[\int |f+g|^p \ \mathrm{d}\mu \le \int |f|^p \ \mathrm{d}\mu + \int |g|^p\ \mathrm{d}\mu\]
  for $a, b \in [0, \infty)$, $(a+b)^p \le a^p + b^p$
  WLOG $b \le a$
  $f(x) = 1+x^p - (1-x)^p$, $f'(x) \ge 0 \implies (1+x)^p \le 1 + x^p$ for $0 \le x \le 1$ 
\end{proof}

\begin{remark}
  For $p < 1$ we do not get$\|f+g\|_p \le \|f\|_p + \|g\|_p$
  for $x, y \in \RR^2$ want to disprove $\|x+y\|_p \le \|x\|_p + \|y\|_p, p < 1$
  $$2^{\frac1p} = (1^p + 1^p)^{\frac1p}$$ (it is because failure of convexity of the norm $p < 1$)
\end{remark}




\begin{claim}
  For $0 < \theta < 1$, $a, b \ge 0$, then $a^{1-\theta}b^\theta \le (1-\theta)a + \theta b$
\end{claim}

\begin{proof}
  Generalized AM-GM inequality ($\sqrt{ab}\le \frac{a+b}{2}$)
  then put for $0 < \theta < 1$ then $a^{1 -\theta}b^\theta \le (1-\theta)a + \theta b$ 
  WLOG $b \le a$ then \[\left(\frac ba\right)^\theta \le 1 - \theta + \theta\frac ba\]
  let $x = \frac ba$ for $0 \le x \le 1$ we need to show that $g(x) = 1- \theta + \theta x - x^\theta \ge 0$
  then $g'(x) = -1 + \theta -\theta x^{\theta - 1} \le 0$ (because $0 \le \theta \le 1$)
\end{proof}

\begin{claim}[Holder's inequality]
  Given $p > 1$, $q$ to be such that 
  \[\frac1{p}+\frac1q = 1\ \ \left(q = \frac p{p-1}\right)\]
  for $f \in L^p, g \in L^{q}$, then $fg \in L^1$ and
  \[\int |fg| \ \mathrm{d}\mu \le \|f\|_p\|g\|_{q}\]
\end{claim}

\begin{proof}
  Rewrite AM-GM (generalized) as ``Young's inquality'' substitute $a = u^p, 1-\theta = \frac1p, b = v^{q}, \theta = \frac1{q}$
  then we get 
  \[uv \le \frac1pu^p + \frac1{q}v^{q}\]
  apply $f(x)g(x)$
  \[\int |f(x)||g(x)|\ \mathrm{d}\mu \le \int \frac{|f(x)|^p}{p} \ \mathrm{d}\mu + \int \frac{|g(x)|^{q}}{q}\ \mathrm{d}\mu = \frac{\|f\|_p^p}{p} + \frac{\|g\|_{q}^{q}}{q}\]
  (This is Holder when two norms are normalized $\|f\|_p = 1 = \|g\|_{q}$)

  Then $\frac{f(x)}{\|f\|_p}$ has ``p-norm'' equal to 1 because
  \[\left(\int \left|\frac{f(x)}{\|f\|_p}\right|^p \ \mathrm{d}\mu\right)^{\frac1p} = \frac{1}{\|f\|_p}\left(\int |f(x)|^p \ \mathrm{d}\mu\right)^{\frac1p}\]
  Substitute $f = \frac{f}{\|f\|_p}$ and $g = \frac{g}{\|g\|_{q}}$ then we get
  \begin{align*}
    \int \frac{|f|}{\|f\|_p}\frac{|g|}{\|g\|_{q}} \ \mathrm{d}\mu  &\le \frac1p + \frac1q = 1 \\
    \int |fg| \ \mathrm{d}\mu &\le \|f\|_p\|g\|_{q} 
  \end{align*}
  % So
  % \[\int \frac{|f|}{\|f\|_p}\frac{|g|}{\|g\|_{q}}  \le 1\]

\end{proof}
  
\begin{theorem}[Minkowski's inquality]
  $p \ge 1$
  We do have a triangle inquality $\|f + g\|_p \le \|f\|_p + \|g\|_p$
  \[\left(\int |f+g|^p\ \mathrm{d}\mu \right)^{\frac1p} \le \left(\int |f|^p\ \mathrm{d}\mu\right)^{\frac1p} + \left(\int |g|^p \ \mathrm{d}\mu\right)^{\frac1p}\]
\end{theorem}

\begin{proof}
  It is enough to show that
  \[\|f+g\|_p^p \le (\|f\|_p + \|g\|_p)\|f+g\|_p^{p-1}\]
  \begin{align*}
    \int |f+&g|^{p-1+1} \ \mathrm{d}\mu = \int |f + g|^{p-1}|f|\ \mathrm{d}\mu + \int |f+g|^{p-1}|g|\ \mathrm{d}\mu \\
    &\le \left(\int |f|^{p}\ \mathrm{d}\mu\right)^{\frac{1}{p}}\left(\int |f+g|^p\ \mathrm{d}\mu\right)^{\frac{p-1}p} + \left(\int |g|^{p}\ \mathrm{d}\mu\right)^{\frac{1}{p}}\left(\int |f+g|^p\ \mathrm{d}\mu\right)^{\frac{p-1}p} \\
    &= (\|f\|_p + \|g\|_p)\|f+g\|_p^{p-1}
  \end{align*}
\end{proof}

\begin{remark}
  Holder's inequality holds
  \[\int fg\ \mathrm{d}\mu \le \|f\|_{p}\|g\|_{q}\]
  where $\frac1p + \frac1{q} = 1$ can be generalized to several factors
  \[\int f_1f_2\dotsm f_n\ \mathrm{d}\mu \le \prod_{j=1}^n \|f_j\|_{p_j}\]
  where $\sum_{j=1}^n \frac1{p_j} = 1$
\end{remark}

\section{Application}
\begin{lemma}[Shebyshev's inequality]
  This is an inequality for the distribution function (given a measure space $(X, \SM, \mu)$) 
  $\mu_f(\alpha) = \mu(\{x : |f(x)| \ge \alpha\})$
  The Shebyshev's inequality is 
  \[\mu_f(\alpha) \le \frac{\|f\|_p^p}{\alpha^p}\]
\end{lemma}
\begin{proof}
  \begin{align*}
    \mu_f(\alpha) &= \int_{E_\alpha} \mathbbm{1} \ \mathrm{d}\mu \\
    &\le \int_{E_\alpha} \frac{|f(x)|^p}{\alpha^p}\ \mathrm{d}\mu \\
    &\le \frac1{\alpha^p}\int |f(x)|^p \ \mathrm{d}\mu \\
  \end{align*} 
  (Probabilitiy may call this Markov's inequality)
\end{proof}

$\mu_{cf}(\alpha) = \mu(\{x : |cf(x)| \ge \alpha\}) = \mu(\{x : |f(x)| \ge \frac{\alpha}{|c|}\})$

$\alpha\mu_{cf}(\alpha)^{\frac1p} = |c|\frac\alpha{|c|}\mu_f\left(\frac\alpha{|c|}\right)^{\frac1p} 
% \le |c|\|f\|_p
$


\begin{claim}
  If $\delta_0 + \delta_1 = 1$, $\delta_0, \delta_1 \ge 0$, then
  \[E_\alpha(f + g) \subseteq E_{\alpha\delta_0}(f) \cup E_{\alpha\delta_1}(f)\]
\end{claim}

\begin{theorem}
  If $\mu(X) < \infty$ then $L^q \subseteq L^p$ for $p \le q$ 
\end{theorem}

\begin{proof}
  We need an inequality $\|f\|_p \le C \|f\|_q$

  \begin{align*}
    \left(\int 1 |f(x)|^p \ \mathrm{d}\mu\right)^{\frac1p} &\le C\left(\int |f(x)|^q \ \mathrm{d}\mu\right)^{\frac1q} \\
  \end{align*}
  Apply Holder with exponent $\frac{q}{p} > 1, \left(\frac{q}{p}\right)'$ where
  \[\frac1{\left(\frac qp\right)} + \frac1{\left(\frac qp\right)'} = 1\]
  \begin{align*}
    \int |f|^p \cdot 1\ \mathrm{d}\mu &\le \left(\int (|f|^p)^\frac{q}{p} \ \mathrm{d}\mu\right)^{\frac{p}{q}}\left(\int_X 1^{\left(\frac qp\right)'} \ \mathrm{d}\mu\right)^{\frac1{\left(\frac{q}p\right)'}} \\
    &= \|f\|_q^p\mu(X)^{\frac1{\left(\frac{q}p\right)'}} \\
    &= \left[\|f\|_q \mu(X)^{\frac1{\left(\frac{q}p\right)'}}\right]^p
  \end{align*}
\end{proof}

Another extreme case would be $\NN$ with counting measure 
In this $L^p(\NN, \mu)$ is denoted by $\ell^p(\NN)$

\begin{theorem}
  For $p \ge 1$, $\ell^p \subseteq \ell^q$ for $p \le q$
\end{theorem}

\begin{proof}
  We wan to prove $\|f\|_{\ell^q} \le C\|f\|_{\ell^p}$
  If $\|f\|_{\ell^p} < 1$ , this means 
  \[\sum_{n=1}^\infty |f(n)|^p \le 1\]
  $\implies$ $|f(n)|^p \le 1$ for all $n$
  \[\sum_{n=1}^\infty |f(n)|^q \le \sum_{n=1}^\infty |f(n)|^p\]
  provided that $|f(n)|^q \le |f(n)|^p$

  For $f \in \ell^p$, $\frac{f}{\|f\|_p}$ has $\ell^p$ norm equal to 1
  therefore $\left\|\frac f{\|f\|_p}\right\|_q \le 1$ then $\|f\|_q \le \|f\|_p$
\end{proof}

\begin{theorem}[Littlewood Theorem]
  Every measurable function is nearly continuous. i.e.,
  $f \in L^1(A)$ there exists $g$ continuous such that
  \[\int |f(x) - g(x)| \ \mathrm{d}m < \eps\]
\end{theorem}

\begin{theorem}[Lusin's Theorem]
  $f:[a, b] \to \CC$ (or $\RR$) almost everywhere, 
  then there is a compact set $K \subseteq [a, b]$ such that $f|_K$ is continuous and $\mu([a, b] \setminus K) = 0$
\end{theorem}

\begin{example}
  $f(x) = \mathbbm{1}_{\QQ'}(x)$
  let $\{r_k\}$ be enumeration of rational number in $[a, b]$. Define
  \[O = \bigcup_{k=1}^\infty \left(r_k - \frac\eps{2^{k+2}}, r_k + \frac\eps{2^{k+2}}\right)\]
  then $m(O) < \eps$, $K = [a, b] \setminus O$, $f|_K = 1$ is continuous.
  % $f$ is not continuous at any point, but $f$ is nearly continuous.
\end{example}

\begin{example}
  \[\sum_{k=1}^\infty \frac1{|x - r_k|^{10}}2^{-k} \mathbbm{1}_{[a, b]}\]
  (this function is in $L^p$, if $p < \frac1{10}$).
  The Challenging part is to find a $K$ as in Lusin's theorem such that
  $f = f|_K$ is continuous.
\end{example}

\begin{proof}
$f: A \to \RR$ (or $\CC$), $m(A) < \infty$
The for every $\eps > 0$, there is a compact set $K \subseteq A$, such that 
$m(A \setminus K) < \eps$ and $f|_K$ is continuous.

\begin{enumerate}
  \item We can find a set $E_1$ such that $m(A\setminus E_1) < \frac\eps3$ and $f|_{E_1}$ is bounded.
  \[S_\alpha = \{x \in A : |f(x)| > \alpha\}\]
  Then $\bigcup S_\alpha = \bigcup S_{2^M}$ has measure zero by the assumption. From continuity from above, we get
  \[m(S_\alpha) = m\left(\bigcup S_{2^M}\right) = \lim_{M\to\infty}m(S_{2^M})\]
  because $m(A) < \infty$. For large $M$, $m(S_M) < \frac\eps3$, let $A \setminus E_1 = S_{2^M}$ ($M$ large)
  \item We know that $f$ is bounded on $E_1$
  Can find a sequence $g_n$ of continuous functions $g_n \to f$ almost everywhere on $E_1$, $m(E_1) < \infty$.
  \item Using Egorov's theorem: $g_n \to f$ almost uniformly on $E_1$.
  Find $E_2 \subseteq E_1$ such that $m(E_1 \setminus E_2) < \frac\eps3$ and $g_n \to f$ uniformly on $E_2$.
  $f|_{E_2}$ is continuous on $E_2$, we can find $E_3 = K$ compact, $E_3 \subseteq E_2$, 
  $m(E_2 \setminus E_3) < \frac\eps3$ and $f|_K$ is continuous.
  \[m(A\setminus K) < \frac\eps3 + \frac\eps3 + \frac\eps3 = \eps\]
\end{enumerate}
\end{proof}
