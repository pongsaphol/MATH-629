\chapter{Practice Exam}
\section{Practice Exam 1}
\begin{problem}
  Let $E_n$ be Lebesgue measurable subsets of $[0, 1]$ such that $E_{n+1} \subseteq E_n$.
  What can you say about the Lebesgue measure of $\bigcap_n E_n$? Does your answer
  necessarily hold when $[0, 1]$ is replaced by $[0, \infty)$?
\end{problem}
\begin{proof}[solution]
  We can use continuity from above because $\mu([0, 1]) < \infty$.
  We can say that $$\mu\left(\bigcap_n E_n\right) = \lim_{n \to \infty} \mu(E_n)$$
  In case of $[0, \infty)$, we can't use continuity from below becuase if $E_n = [n, n+1)$
  then $\mu(E_n) = 1$ but $\bigcap_n E_n = \emptyset$, so, $\lim_{n\to \infty}\mu(E_n) = 1$ but 
  $\mu\left(\bigcap_n E_n\right) = 0$.
\end{proof}

% q2
\begin{problem}
  Let 
  \begin{align*}
    E = \{x &= (x_1, x_2) \in \RR^2 : \\
    &\frac1{1 + (x_1 - x_2)^3} \le e^{\sin x_1} \text{ if } x_1^{23} < 3|\cos(x_1 + x_2)|, \\
    &\sqrt{1 + e^{|x_2| + |x_1|}} > e^{x_1^2} \text{ if } x_1^{23} > 3|\cos(x_1 + x_2)|\text{ and }x_2 \in \RR\setminus\QQ, \\
    &\sqrt{\cos(|x_1x_2|)}\sin(x_1x_2) > 0 \text{ if } x_1^{23} > 3|\cos(x_1 + x_2)|\text{ and }x_2 \in \QQ\}
  \end{align*}
  \begin{enumerate}[(i)]
    \item Is the characteristic function of $E$ Borel measurable?
    \item If $\SM$ denote the $\sigma$-algebra of Lebesgue measurable subsets of $\RR$ does $E$ belong to $\SM \oplus \SM$?
  \end{enumerate}
\end{problem}

\begin{proof}[solution]
  \begin{enumerate}[(i)]
    \item Let $f_1(x_1, x_2) = \frac1{1 + (x_1 - x_2)^3} - e^{\sin x_1}$, $f_2(x_1, x_2) = \sqrt{1 + e^{|x_2| + |x_1|}} - e^{x_1^2}$, $f_3(x_1, x_2) = \sqrt{\cos(|x_1x_2|)}\sin(x_1x_2)$ 
    and $g(x_1, x_2) = x_1^{23} - 3|\cos(x_1 + x_2)|$ 
    Then $E = \{x : f_1(x) \le 0 \text{ if } g(x) < 0, f_2(x) > 0 \text{ if } g(x) > 0 \text{ and } x_2 \in \RR \setminus \QQ \text{ ...}\}$
    $E = (f_1^{-1}(-\infty, 0] \cap g^{-1}(-\infty, 0)) \cup (f_2^{-1}(0, \infty) \cap g^{-1}((0, \infty)) \cap \RR \times (\RR \setminus \QQ)) \cup \dots$
    Then $f_1, f_2, f_3, g$ are Borel measurable (because it is continuous) functions then $E$ is Borel measurable.
    \item 
  \end{enumerate}
\end{proof}

%q3
\begin{problem}
  For each of the statements give a proof of find a counterexample.
  \begin{enumerate}[(i)]
    \item For each $r \in \RR$ let $f_r : \RR \to \RR$ be a continuous function. Define 
    $G(x) = \sup_{r \in \RR} f_r(x)$. Is $G$ a Borel measurable function?
    \item For each $r \in \RR$ let $f_r : \RR \to \RR$ be a Borel measurable function. 
    Define $G(x) = \sup_{r \in \RR} f_r(x)$. Is $G$ a Borel measurable function?
    \item For each $r \in \RR$ let $f_r : \RR \to \RR$ be a Borel measurable function.
    Define $G(x) = \sup_{r \in \QQ} f_r(x)$. Is $G$ a Borel measurable function?
    \item For each $r \in \QQ$ let $f_r : \RR \to \RR$ be a Borel measurable function.
    Define $S(x) = \sum_{r \in \QQ} f_r(x)$. Is $S$ a Borel measurable function?
    \item What happens if you replace Borel measurable by Lebesgue measurable in the above statements?
  \end{enumerate}
\end{problem}

\begin{proof}[solution]
  \begin{enumerate}[(i)]
    \item $G^{-1}((-\infty, x]) = \bigcap_{r \in \RR} f_r^{-1}((-\infty, x])$ and $f_r$ is continuous so, $G$ is Borel measurable.
    \item No, $f_r = \mathbbm{1}_{\{c_r\}}$ where $c_r \in \QQ + r$ and $c_r \in [0, 1)$ then $G^{-1}(\{1\})$ is Vitali set.
    \item For $E \in \SM$, $G^{-1}(E) = \bigcap_{r \in \QQ} f_r^{-1}(E)$ and $f_r$ is Borel measurable so, $G$ is Borel measurable.
    \item Suppose that $\{q_r\}$ is an enumeration of $\QQ$ then define $g_n = \sum_{r=1}^n f_{q_r}$ then suppose that limit exists then
    $\limsup g_n$ and $\liminf g_n$ are Borel measurable then $S$ is Borel measurable.
  \end{enumerate} 
\end{proof}

%q4
\begin{problem}
  Assume that $f_n$ is a sequence of $L^1(\mu)$ functions.
  \begin{enumerate}[(i)]
    \item If in addition $\mu(X) < \infty$ and 
    \[\lim_{n\to\infty}\sup_{x \in X} |f_n(x) - f(x)| = 0\]
    then you have learned that $f \in L^1(\mu)$ and that $\lim_{n\to\infty}\int f_n \ \mathrm{d}\mu = \int f\ \mathrm{d}\mu$.
    Review the proof.
    \item Show that if $\mu(X) = \infty$ both conclusions may fail in (i).
    \item If $\mu(X) < \infty$ if $f_n \to f$ converges just pointwise show that for every $\eps > 0$
    there exists a set $A$ of measure $< \eps$ such that 
    \[\int_{A^\complement} |f_n - f| \ \mathrm{d}\mu < \eps\]
  \end{enumerate}
\end{problem}
\begin{proof}[solution]
  \begin{enumerate}[(i)]
    \item Suppose that $\lim_{n\to\infty} \|f_n - f\|_{\sup} \to 0$ then 
    \[\lim_{n\to\infty}\int |f_n - f| \ \mathrm{d}\mu \le \lim_{n\to\infty}\mu(X)\|f_n - f\|_{\sup} = 0\]
    and we know that $f = (f - f_n) + (f_n)$ then
    \[\int |f| \ \mathrm{d}\mu = \int |f - f_n + f_n| \ \mathrm{d}\mu \le \int |f - f_n| \ \mathrm{d}\mu + \int |f_n| \ \mathrm{d}\mu\]
    so, $f \in L^1$ and 
    \[\left| \int f_n - f \ \mathrm{d}\mu\right|\le \int |f_n - f| \ \mathrm{d}\mu\]
    So, 
    \[\ \lim_{n\to\infty}\int f_n\ \mathrm{d}\mu = \int f \ \mathrm{d}\mu\]
    \item Let $f_n(x) = \frac1x\mathbbm{1}_{[1, n]}$ then $f_n \in L^1$ but $f_n \to f = \frac{1}x\mathbbm{1}_{[1, \infty)}$ and $f \not\in L^1$.
    \item By Egorov's theorem, for any $\eps > 0$ there exists $A \in \SM$ such that $\mu(A) < \eps$ and $f_n \to f$ uniformly on $A^\complement$.
    Then select $n$ such that $\|f_n - f\| < \frac\eps{\mu(X)}$ then
    \[\int_{A^\complement} |f_n - f| \mathrm{d}\mu < \eps\]
    % \item Define $E_n(k) = \{x : \exists m \ge n, |f_m(x) - f(x)| > 1/k\}$ fix $k$, then 
    % \[E_n(k) \supseteq E_{n+1}(k) \text{ and } \bigcup_{n \in \NN} E_n(k) = \emptyset\]
    % From continuity from above, there exists $n_k$ such that $\mu(E_{n_k}(k)) < \frac\eps{2^k}$.
    % Define 
    % \[A = \bigcup_{k \in \NN} E_{n_k}(k), \mu(A) < \eps\]
    % and for all $x \in A^\complement$, there exists $k$ such that $\frac1k < \eps$ for $m \ge n_k$,
    % \[|f_m(x) - f(x)| < \eps\]
    % Then substitute $\eps = \eps / \mu(X)$ and we get the result.
  \end{enumerate}
\end{proof}

%q5
\begin{problem}
  Let $\SM$ be a $\sigma$-algebra on $X$.
  \begin{enumerate}[(i)]
    \item Consider a function $f = \begin{pmatrix}
      f_1\\
      f_2
    \end{pmatrix} : X \to \RR^2$ such that for all rational number $r_1, r_2$ the sets
    \[\{x \in X : r_1 < f_1(x), f_2(x) < r_2\}\]
    belong to $\SM$. Is $f$ ($\SM, \mathcal{B}(\RR^2)$)-measurable?
    \item Let $f_n : X \to \overline{\RR}$ a sequence of measurable functions. Show that the set
    \[E = \{x \in X : \{f_n(x)\}_{n=1}^\infty \text{ is a nondecreasing sequence}\}\]
    is measurable.
  \end{enumerate} 
\end{problem}

\begin{proof}[solution]
  \begin{enumerate}[(i)]
    \item $\{x : r_1 < f_1(x), f_2(x) < r_2\} = f_1^{-1}((r_1, \infty)) \cap f_2^{-1}((-\infty, r_2))$
    It is enough to show that for any $p \in \RR$, $f_1^{-1}(p, \infty) \in \SM$ and $f_2^{-1}(-\infty, p) \in \SM$.
    \begin{itemize}
      \item Fix $p$, there exists a sequence $\{x_n\}, \{y_n\}$ of $\QQ$ such that $x_n \to p^+$ and $y_n \to \infty$
      Then \[f_1^{-1}(p, \infty) = \bigcup_{n=1}^\infty f_1^{-1}((x_n, \infty)) \cap f_2^{-1}((-\infty, y_n)) \in \SM\]
      \item Similarly, there exists a sequence $\{x_n\}, \{y_n\}$ of $\QQ$ such that $x_n \to -\infty$ and $y_n \to p^-$ 
      Then \[f_2^{-1}(-\infty, p) = \bigcup_{n=1}^\infty f_1^{-1}((x_n, \infty)) \cap f_2^{-1}((-\infty, y_n)) \in \SM\]
    \end{itemize}
    \item Let $g_k = f_k - f_{k+1}$ then $g_k$ is measurable. So,
    \[E = \bigcap_{k=1}^\infty \{x \in X : g_k(x) \ge 0 \}\]
  \end{enumerate}
\end{proof}

%q6
\begin{problem}
  Let $f$ be a Lebesgue measurable function on $\RR^n$. Let $m$ denote Lebesgue measure on $\RR^n$.

  Show that the following  three statements are equivalent:
  \begin{enumerate}[(a)]
    \item $f$ is integrable (i.e. belong to $L^1(\RR^n)$).
    \item $\sum_{k \in \ZZ} 2^km(\{x : |f(x)| > 2^k\}) < \infty$.
    \item $\sum_{k \in \ZZ} 2^km(\{x : 2^k \le |f(x)| < 2^{k+1}\}) < \infty$.
  \end{enumerate}
\end{problem}

\begin{proof}[solution]
  \begin{itemize}
    \item (a) $\Rightarrow$ (c): 
      \[\sum_{k \in \ZZ} 2^km(\{x : 2^k \le |f(x)| < 2^{k+1}\}) \le \int |f| \ \mathrm{d}m < \infty\]
    \item (b) $\Rightarrow$ (c): 
    \[\sum_{k \in \ZZ} 2^km(\{x : 2^k \le |f(x)| < 2^{k+1}\}) < \sum_{k \in \ZZ} 2^km(\{x : |f(x)| > 2^{k-1}\}) < \infty\]
    \item (c) $\Rightarrow$ (a): 
    \[\int |f| \ \mathrm{d}m \le \sum_{k\in \ZZ} 2^{k+1}m(\{x : 2^k \le |f(x)| < 2^{k+1}\}) < \infty\]
    \item (c) $\Rightarrow$ (b): 
    Let $E_k = \{x : 2^k \le |f(x)| < 2^{k+1}\}$ then
    \begin{align*}
      \sum_{k \in \ZZ} 2^km(\{x : |f(x)| > 2^k\}) &\le \sum_{k \in \ZZ} 2^k \sum_{j=k-1}^\infty E_j\\
      &= \left(\sum_{k \in \ZZ} 2^k E_{k-1} + E_k\right) + \left(\sum_{k \in \ZZ}2^k\sum_{j=k+1}^\infty E_j\right) \\
      &= 3\sum_{k \in \ZZ} 2^kE_k + \left(\sum_{j \in \ZZ} \sum_{k=-\infty}^{j-1} 2^k E_j\right) \\
      &\le 3\sum_{k \in \ZZ} 2^kE_k + \left(\sum_{j \in \ZZ} 2^j E_j\right) \\
      &= 4\sum_{k \in \ZZ} 2^kE_k
      % &\le \sum_{k \in \ZZ} 2\cdot2^k m(\{x : 2^k \le |f(x)| < 2^{k+1}\})\\
    \end{align*}
    \[ \]
  \end{itemize}
\end{proof}

%q7
\begin{problem}
  Let $m$ be Lebesgue measure on $\RR$ and $f$ be a Lebesgue measurable function with $\int |f|\ \mathrm{d}m < \infty$. 
  Define $G(x) = \int_{-\infty}^x f\ \mathrm{d}m$. Prove that $G$ is uniformly continuous on $\RR$.
\end{problem}

\begin{proof}[solution]
  I want to show that for any $\eps > 0$, there exists $\delta > 0$ such that for any $E \in \SM$ with $m(E) < \delta$,
  implies that $\int_E |f|\ \mathrm{d}m < \eps$. Define $E_n = \{x : |f(x)| > n\}$.
  Define $f_n = 1_{E_n}|f|$ then $|f_n| \le |f|$. Then by Dominated Convergence Theorem,
  \[\lim_{n\to\infty}\int f_n \ \mathrm{d}m = 0\]
  There exists $k$ such that $\int f_k \ \mathrm{d}m < \eps/2$, select $\delta = \frac\eps{2k}$.
  Given any $E \in \SM$ then 
  \begin{align*}
    \int_E |f|\ \mathrm{d}m &= \int_{E \cap E_n} |f|\ \mathrm{d}m + \int_{E \cap E_n^\complement} |f|\ \mathrm{d}m\\
    &\le \frac\eps2+ n\mu(E) \\
    &\le \eps
  \end{align*}
  For any $\eps > 0$, pick $\delta$ to be in this theorem, then 
  if $|x - y| < \delta$ then $|G(x) - G(y)| < \eps$. 
\end{proof}

\begin{problem}
  Determine the limits 
  \begin{enumerate}[(i)]
    \item $\lim_{n\to\infty} \int_0^{n^{99/100}} (x/n)^n\ \mathrm{d}x$
    \item $\lim_{n\to\infty} \int_0^n (1 - \frac xn)^n e^{-2x}\ \mathrm{d}x$
    % \item $\lim_{n\to\infty} \int_0^n (1 + \frac xn)^n e^{x/2}\ \mathrm{d}x$
  \end{enumerate}
  and, in both cases carefully justify your computation.
\end{problem}
\begin{proof}[solution]
   \begin{enumerate}[(i)]
    \item define $f_n = \mathbbm{1}_{[0, n^{99/100}]} (x/n)^n$ 
    and $f = $ then $|f_n| \le |f| \in L^1([0, \infty))$ then by Dominated Convergence Theorem,
    % \item define \[f_n = \mathbbm{1}_{[0, n]} \left(1 - \frac xn\right)^n e^{x/2}\]
    % and $f = e^{-x}e^{x/2}$
    % then $|f_n| \le |f| \in L^1([0, \infty))$ then by Dominated Convergence Theorem,
    % \[\lim_{n\to\infty}\int_{0}^\infty f_n\ \mathrm{d}x = \int_{0}^\infty\lim_{n\to\infty} f_n\ \mathrm{d}x = \int_0^\infty e^{-x/2}\ \mathrm{d}x = 2\]
    % \item Similarly, define \[f_n = \mathbbm{1}_{[0, n]} \left(1 + \frac xn\right)^n e^{x/2}\]
    % and $f = e^{x}e^{x/2}$
    % then $|f_n| \le |f| \in L^1$ then by Dominated convergence theorem,
    % \[\lim_{n\to\infty}\int_{0}^\infty f_n\ \mathrm{d}x = \int_{0}^\infty\lim_{n\to\infty} f_n\ \mathrm{d}x = \int_0^\infty e^{3x/2}\ \mathrm{d}x = 2\]
   \end{enumerate}
\end{proof}

\begin{problem}
  Let $f \in \mathcal{L}^1(\RR^n)$. Let m be Lebesgue measure in $\RR^n$. Prove that for $t > 0$
  \[t^n \int f(tx)\ \mathrm{d}m = \int f(x)\ \mathrm{d}x\]
Hint: First prove this for indicator functions of cubes, then for indicator
functions of sets of finite measure.
\end{problem}

\begin{proof}[solution]
  Suppose that $f = \mathbbm{1}_E$, then
  \[\mathbbm{1}_E(tx) = \begin{cases}
    1 & \text{if } tx \in E\\
    0 & \text{otherwise}
  \end{cases} = \begin{cases}
    1 & \text{if } x \in \frac1tE\\
    0 & \text{otherwise}
  \end{cases} = \mathbbm{1}_{\frac1tE}(x)\] 
  and $m(\frac1tE) = t^{-n}m(E)$1
\end{proof}

\begin{problem}
  Let $f \in \mathcal{L}^1(\RR^n)$. Then
  \begin{enumerate}[(i)]
    \item $\lim_{|h| \to 0} \int |f(x+h) - f(x)|\ \mathrm{d}m = 0$.
    \item $\lim_{t \to 1} \int |f(tx) - f(x)|\ \mathrm{d}m = 0$.
    \item Can the Lebesgue dominated convergence be used for the proof of (i) or (ii)?
  \end{enumerate}
\end{problem}

\begin{proof}[solution]
\begin{enumerate}[(i)]
  \item there exists $g$ continuous function with compact support such that $\int |f-g| < \frac\eps3$ then
  \[|f(x+h) - f(x)| \le |f(x+h) - g(x+h)| + |g(x+h) - g(x)| + |g(x) - f(x)|\]
  Obviously, $\int |f(x+h) - g(x+h)| \ \mathrm{d}m < \frac\eps3$ and $\int |g(x) - f(x)| \ \mathrm{d}m < \frac\eps3$ then
  \begin{align*}
    \int |g(x+h) - g(x)| \ \mathrm{d}m &\le \eps \cdot 2\mu(K)
  \end{align*}
  \item 
\end{enumerate}
\end{proof}

\begin{problem}
  Let $I = [a, b], f \in \mathcal{L}^1(I)$. Show that
  \[\lim_{n\to\infty} \int_I f(x)\sin(nx)\ \mathrm{d}m(x) = 0\]
\end{problem}

\begin{proof}[solution]
  There exists $g$ step such that $\int |f-g| \ \mathrm{d}m < \eps$ then 
  \begin{align*}
    \left|\int f(x)\sin(nx) - g(x)\sin(nx) \ \mathrm{d}m\right| &\le \int |f(x) - g(x)||\sin(nx)|\ \mathrm{d}m\\ 
    \left|\int f(x)\sin(nx)\ \mathrm{d}m\right|- \left|\int g(x)\sin(nx) \ \mathrm{d}m\right|&\le \eps
  \end{align*}
  then fix some interval $c$ then 
  \begin{align*}
    \left|\int c\sin(nx) \ \mathrm{d}m\right| &= \frac1n\left|\int c\sin(x) \ \mathrm{d}m\right|\\
    &\to 0
  \end{align*}
  as $n \to \infty$.
\end{proof}

%q12 
\begin{problem}
  Recall the monotone convergence theorem and Fatou's lemma.
  \begin{enumerate}[(i)]
    \item Show that Fatou's lemma implies the monotone convergence theorem.
    \item Show that the monotone convergence theorem implies Fatou's lemma.
  \end{enumerate}
\end{problem}

\begin{proof}[solution]
  \begin{enumerate}[(i)]
    \item $f_n \le f \implies \int f_n \le \int f$ So, $\lim_{n\to\infty} \int f_n \le \int f$.
    Then from Fatou's lemma, we have
      \[\int \liminf_{k\to\infty} f_k \ \mathrm{d}\mu \le \liminf_{k\to\infty}\int f_k \ \mathrm{d}\mu\]
    Since $f_n$ is non-decreasing, so, $\int f_n$ is also non-decreasing. So, we have $\displaystyle\liminf_{k\to\infty} f_k = \lim_{k\to\infty} f_k$ then
    \[\int \lim_{k\to\infty} f_k \ \mathrm{d}\mu \le \lim_{k\to\infty}\int f_k \ \mathrm{d}\mu\]
    \item we want to show that 
    \[\int \liminf_{k\to\infty} f_k \ \mathrm{d}\mu \le \liminf_{k\to\infty}\int f_k \ \mathrm{d}\mu\]
    It is enough to show that
    \[\int \lim_{n\to\infty} \inf_{m \ge n} f_m \ \mathrm{d}\mu \le \lim_{n\to\infty} \inf_{m \ge n}\int f_m \ \mathrm{d}\mu\]
    From the monotone convergence theorem, we have
    \[\int \lim_{n\to\infty} \inf_{m \ge n} f_m \ \mathrm{d}\mu \underset{\text{MCT}}= \lim_{n\to\infty}\int \inf_{m \ge n} f_m \ \mathrm{d}\mu\]
    Then for any $k \ge n$, we have $\inf_{m \ge n} f_m \le f_k$ then
    \[\int \inf_{m \ge n} f_m \ \mathrm{d}\mu\le \int f_k \ \mathrm{d}\mu\]
    So, we have 
    \[\int \lim_{n\to\infty} \inf_{m \ge n} f_m \ \mathrm{d}\mu \le \lim_{n\to\infty} \inf_{m \ge n}\int f_m \ \mathrm{d}\mu\]
  \end{enumerate} 
\end{proof}

\begin{problem}
  Determine
  \[\lim_{n\to\infty} \int_0^\infty\frac{2n \sin(x/n)}{x(1+x^2)}\ \mathrm{d}x\]
  Provide justifications.
\end{problem}

\begin{proof}[solution]
  We know that $|n/x \sin(x/n)| \le 1$ then define 
  \[f_n = \frac{2n \sin(x/n)}{x(1+x^2)}\]
  Then $|f_n| \le \frac{2n}{x(1+x^2)}$ and $\int \frac{2n}{x(1+x^2)} \ \mathrm{d}x < \infty$ then by Dominated Convergence Theorem,
  \begin{align*}
  \lim_{n\to\infty} \int_0^\infty\frac{2n \sin(x/n)}{x(1+x^2)}\ \mathrm{d}x &= \int_0^\infty \lim_{n\to\infty} \frac{2n \sin(x/n)}{x(1+x^2)}\ \mathrm{d}x \\
  &= \int_0^\infty \frac{2}{1+x^2}\ \mathrm{d}x\\
  &= \pi
  \end{align*}

\end{proof}

\begin{problem}
  Let $f(x) = \sin(x^2)$ on the measure space $X = [1, \infty)$ (with Lebesgue measure $m$). Prove:
  \begin{enumerate}[(i)]
    \item $\int_{[1, \infty)} |f| \ \mathrm{d}m = \infty$
    \item $\lim_{R \to \infty}\int_{[0, R]} f\ \mathrm{d}m$ exists (and is finite).
  \end{enumerate}
  Hint: For part (ii) use that $2x\sin(x^2)$ is the derivative of $-\cos(x^2)$.
\end{problem}

\begin{proof}[solution]
  \begin{enumerate}[(i)]
    \item Constructing triangles under the curve, we have 
  \begin{align*}
    \int_{[0, \infty)} |f| \ \mathrm{d}m &\ge \sum_{k=1}^\infty \frac12 \sqrt{(k+1)\pi} - \sqrt{k\pi}\\
    &= \frac12\sqrt{\pi}\sum_{k=1}^\infty \sqrt{k+1} - \sqrt{k}\\
    &= \frac12\sqrt{\pi} \left(\lim_{k \to \infty} \sqrt{k} - \sqrt{1}\right) \\
    &= \infty
  \end{align*}
  \item Define $u = 1/2x$ then $dv = 2x \sin(x^2)\ \mathrm{d}x$ then $v = -\cos(x^2)$ and $du = -1/2x^2\ \mathrm{d}x$ then
  \begin{align*}
    \int_0^R f \ \mathrm{d}x &= uv\Big|_0^R - \int_0^R v\ \mathrm{d}u\\
    &= -\frac12\cos(R^2) + \frac12\cos(0) - \int_0^R -\cos(x^2)\ \mathrm{d}u\\
  \end{align*}

  \end{enumerate}
\end{proof}

% q15
\begin{problem}
  Let $f : X \to \overline{\RR}$ be a nonnegative measurable function on the measure space $(X, \SM, \mu)$
  and assume $\mu(X) < \infty$.
  \begin{enumerate}[(i)]
    \item Let $E_R = \{x \in X : |f(x)| > R\}$. Prove: If $|f(x)| < \infty$ for almost every $x \in X$ then 
    $\lim_{R \to \infty}\mu(E_R) = 0$.
    \item Is the conclusion in (i) still valid if we drop the assumption of finite measure space?
    Give a proof or counterexample.
  \end{enumerate}
\end{problem}

\begin{proof}[solution]
  \begin{enumerate}[(i)]
    \item Using continuity from above, we have
    \[\lim_{n\to\infty} \mu(E_n) = \mu\left(\bigcap_{n=1}^\infty E_n\right)\]
    and $|f(x)| < \infty$ for almost every $x \in X$ then $\mu\left(\bigcap_{n=1}^\infty E_n\right) = 0$, so $\lim_{R\to\infty} \mu(E_R) = 0$.
    \item $f(x) = x$
  \end{enumerate}
\end{proof}

% q16
\begin{problem}
  Let $p > 0$. For $x \in \RR^n$ let $|x|_p = (\sum_{i=1}^n |x_i|^p)^{1/p}$. Let $\Omega = \{x \in \RR^n : |x|_p > 3\}$.
  Show that 
  \[\int_{\Omega} |x|_p^{-\alpha}\ \mathrm{d}m < \infty\]
  if and only if $\alpha > n$. What is the result if you replace $\Omega$ by $\Omega^\complement$?
\end{problem}

\begin{proof}[solution]
  Define $E_k = \{x \in \RR^n : 3^k \le |x|_p < 3^{k+1}\}$ then
  \begin{align*}
    \int_{\Omega} |x|_p^{-\alpha}\ \mathrm{d}m &= \sum_{k=1}^\infty \int_{E_k} |x|_p^{-\alpha}\ \mathrm{d}m\\
    &\le \sum_{k=1}^\infty c_1 3^{-\alpha k}\mu(E_k)\\
    &\le \sum_{k=1}^\infty c_1 3^{-\alpha k}c_23^{kn}\\
    &= c_1c_2\sum_{k=1}^\infty 3^{k(n - \alpha)}\\
  \end{align*}
  So, $n - \alpha < 0$ then $\alpha > n$. For the converse use the same bound (but lower bound)
  
  For $\Omega^\complement$, define $E_k = \{x \in \RR^n : 3^{-k} \le |x|_p < 3^{-k+1}\}$ then
  \begin{align*}
    \int_{\Omega^\complement} |x|_p^{-\alpha}\ \mathrm{d}m &= \sum_{k=1}^\infty \int_{E_k} |x|_p^{-\alpha}\ \mathrm{d}m\\
    &\le \sum_{k=1}^\infty c_1 3^{\alpha k}\mu(E_k)\\
    &\le \sum_{k=1}^\infty c_1 3^{\alpha k}c_23^{-kn}\\
    &= c_1c_2\sum_{k=1}^\infty 3^{k(\alpha - n)}\\
  \end{align*}
  So, $\alpha - n < 0$ then $\alpha < n$.
\end{proof}
