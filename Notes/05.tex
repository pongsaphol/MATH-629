\chapter{Outer Measure}
\section{Abstract Outer Measure}

\begin{definition}
  Abstract outer measure, 
  \[\varrho: \mathfrak{P}(X) \to [0, \infty]\]
  $\varrho(\emptyset) = 0$, which have
  \begin{itemize}
    \item Monoticity $A \subseteq B, \varrho(A) \le \varrho(B)$
    \item $\sigma$-Subadditivity $\varrho\left(\bigcup_{j=1}^\infty E_j\right) \le \sum_{j=1}^\infty \varrho(E_j)$
  \end{itemize}
\end{definition}


\begin{definition}
  Given $X$, $\SE \subseteq \mathfrak{P}(X)$ where $\emptyset \in \SE$ and $X \in \SE$, $u: \SE \to [0, \infty]$.
  For any set 
  \[u^*(E) = \inf_{\bigcup E_j \supseteq E}\sum_{j=1}^\infty u(E_j)\]
  where $E_j \in \SE$, called the (concrete) outer measure induced by $\SE$.
  ($u^*(E) = \infty$ if we cannot cover $E$ with a countable collection of sets in $\SE$
\end{definition}

\begin{remark}
  It is not necessary to assume that $X \in \SE$.
  In this case, $u^*(E) = \infty$ if we cannot cover $E$ with a countable collection of sets in $\SE$.
\end{remark}

\begin{lemma}
  An outer measure $u^*$ incuced by the collection $\SE$ satisfies
  \begin{enumerate}[(i)]
    \item $u^*(\emptyset) = 0$
    \item For $A \subseteq B$, $u^*(A) \le u^*(B)$ Monotonicity
    \item $u^*(\bigcup E_j) \le \sum u^*(E_j)$ Subaddictivity
  \end{enumerate}
  E.g., A (concrete) outer measure is an abstract outer measure
\end{lemma}

\begin{proof}
  $u*(\bigcup A_j) \subseteq \sum u^*(A_j)$
  WLOG, $LHS < \infty$, for each $j$ find a collection $\{F_k^\gamma\}_{k \in \NN}$
  \[\sum u(E_k^j) < u*(A_j) + \eps 2^{-j-1}\]
  $\{E_k^j\}_{j, k=1}^\infty$ covers $\bigcup A_j$
  \[u^*\left(\bigcup A_j\right)\le \sum_j\sum_k u(E_k^j) \le \sum_j u^*(A_j) + \eps \sum_j {2^{-j-1}}\]
\end{proof}

\section{Caratheodory's Construction}

\begin{definition}
  Given $\varrho$ abstract outer measure,
  a set $A$ is Caratheodory measure or $\varrho$-measurable (in the sence of) 
  if \textbf{for all} $E \in \mathfrak{P}(X)$, 
  \[\varrho(E) = \varrho(E \cap A) + \varrho(E\cap A^\complement)\]
\end{definition}

\begin{remark}
  It is trivial that 
  \[\varrho(E) \le \varrho(E \cap A) + \varrho(E \cap A^\complement)\] 
  So, we only need 
  \[\varrho(E) \ge \varrho(E \cap A) + \varrho(E \cap A^\complement)\] 
  to show the equality of the outer measure.
\end{remark}

\begin{theorem}[Caratheodory's Theorem]
  The collection of $\varrho$-measurable sets is a $\sigma$-algebra $\SM$,
  $\varrho|_\SM$
  is a measure (in fact, a complete measure)
\end{theorem}

\begin{proof}
  for any set $\varrho(\emptyset) = 0$. For $A \neq \emptyset$, $\varrho(A) = c \neq 0$,
  $\varrho(E) = \varrho(E \cap A) + \varrho(E\cap A^\complement)$
  \begin{enumerate}
    \item want to show that for any $A, B \in \SM$, $A \cap B \in \SM$
      \begin{align*}
        \varrho(E) &\ge \varrho(E \cap A) + \varrho(E \cap A^\complement) \\
        &= \varrho(E\cap A\cap B) + \varrho(E \cap A \cap B^\complement) + 
          \varrho(E \cap A^\complement \cap B) + \varrho(E \cap A^\complement \cap B^\complement) \\
        &\ge \varrho(E \cap (A \cap B)) 
      \end{align*}
      This shos that $\SM$ is an algebra.
      A finite union of sets in $\SM$ can be written as a finite disjoint union.
      We also get the addicitivity of $\varrho$ when applied to sets in $\SM$
      let $E = A \uplus B$ then 
      \[\varrho(A \uplus B) = \varrho((A \uplus B) \cap A) + \varrho((A \uplus B) \cap A^\complement) = \varrho(A) + \varrho(B)\]
    \item $A_1, \dotsc, A_n$ parwise \textbf{disjoint}, let $u_n = \biguplus_{j=1}^n A_j$. 
      We claim that $$\varrho(u_n \cap E) = \sum_{j=1}^n \varrho(A_j \cap E)$$ (for all $E$)
      Observe that $u_n = A_n \uplus u_{n-1}$, $A_n, u_{n-1} \in \SM$ then
      \begin{align*}
        \varrho(u_n \cap E) &= \varrho(A_n \cap u_n \cap E) + \varrho(A_n^\complement \cap u_n \cap E) \\
        &= \varrho(A_n \cap E) + \varrho(u_{n-1} \cap E) \\
      \end{align*}
      Iterate and we get the claim.

      Set $U = \bigcup_{j=1}^\infty A_j$, GOAL: is to show $\varrho(E) = \varrho(E \cap U) + \varrho(E \cap U^\complement)$.
      \begin{align*}
        \varrho(E) &\ge \varrho(u_n \cap E) + \varrho(u_n^\complement \cap E) \\
        &\ge \varrho(u_n \cap E) + \varrho(U^\complement \cap E) \\
        &= \sum_{j=1}^n\varrho(A_j \cap E) + \varrho(U^\complement \cap E) \\
        &\ge \varrho(U \cap E) + \varrho(U^\complement \cap E)
      \end{align*}
      So $\varrho$ is $\sigma$-addictive on $\SM$.
  \end{enumerate} 
\end{proof}

\begin{definition}
  $\varrho$ is complete if for $A \in \SM$, if $\varrho(A) = 0$ then $\varrho(N) = 0$ for $N \subseteq A$.
\end{definition}

``Intervals'' or ``I-cells'' half open intervals of the form $(a, b]$
\[(a, b] \uplus (b, c] = (a, c]\]
n-dimensional analogy n-cells
\[(a_1, b_1] \times (a_2, b_2] \times \dotsm \times (a_n, b_n]\]

\begin{definition}[Semiring]
  Given $X$, a collection of subsets $\mathcal{S}$ of $X$ is a semiring if 
  \begin{enumerate}[(i)]
    \item $\emptyset \in \mathcal{S}$
    \item $A, B \in \mathcal{S} \implies A \cap B \in \mathcal{S}$
    \item $A, B \in \mathcal{S} \implies A \setminus B = \biguplus_{j=1}^n C_j$ where $C_j \in \mathcal{S}$
  \end{enumerate}
\end{definition}

\begin{definition}[Ring]
  A collection of subsets $R$ is a ring if 
  \begin{enumerate}[(i)]
    \item $\emptyset \in R$
    \item $A, B \in R \implies A \setminus B \in R$
    \item $A, B \in R \implies A \cup B \in R$
  \end{enumerate}
\end{definition}

\begin{remark}
  $A \cap B = A \setminus (A \setminus B)$, $A \bigtriangleup B = (A \setminus B) \uplus (B \setminus A)$
  
  The ring is equivalent to for $A, B \in R$
  \begin{enumerate}[(i)]
    \item $\emptyset \in R$
    \item $A \cap B \in R$
    \item $A \bigtriangleup B \in R$
  \end{enumerate}
  where $\cap$ is multiplication and $\bigtriangleup$ is addition.

  Finite $\implies$ unions of disjoint sets in $R$ are in $R$.
  \[A \cup B = (A \cap B) \uplus (A \bigtriangleup B)\]
  \[(A \bigtriangleup B)\cap A = (A \setminus B \cup B \setminus A) \cup A = A \setminus B\]
\end{remark}
  
\begin{definition}
  $\SE$ a collection. $\mathcal{R}(\SE)$ is the smallset ring that contains the collection. 
  \[\mathcal{R}(\SE) = \bigcap_{\underset{\text{contains }\SE}{R \text{ ring}}} R\]
\end{definition}

\begin{lemma}
  If $\SE$ any collection. if $\SE \subseteq R$ then $\mathcal{R}(\SE) \subseteq R$
\end{lemma}

\begin{theorem}
  Let $\mathcal{S}$ be a semiring (of subsets of $X$). Then $\mathcal{R}(\mathcal{S})$ is a ring generated by $\mathcal{S}$,
  is the collection of finite disjoint unions of sets in $\mathcal{S}$.
\end{theorem}

\begin{proof}
  GOAL: $A \setminus B \in \mathcal{R}(\mathcal{S})$.
  \[\biguplus A_j \setminus \biguplus B_k = \biguplus_j (A_j \setminus \biguplus B_k)\]
  Neet to check that for each $j$, $(A_j \setminus \biguplus B_k)$ is a disjoint union of $\mathcal{S}$. 
  Take $A \in \mathcal{S}$, $B_1, \dotsc, B_n \in \mathcal{S}$, $B_i$ are disjoint

  Claim$_n$: $A \setminus \biguplus B_k \in$ disjoint union of sets in $\mathcal{S}$.
  By induction $n = 1$, by definition (iii)

  then Claim$_{n-1} \implies$ Claim$_n$. Assume 
  \begin{align*}
    A \setminus \biguplus_{k=1}^{n-1}B_k &= \biguplus_{l=1}^M C_l \\
    A \setminus \left(\biguplus_{k=1}^{n-1}B_k\right) \setminus B_n &= \bigcup_{l=1}^M (C_l\setminus B_n) \\
    A \setminus \left(\biguplus_{k=1}^{n}B_k\right) &= \biguplus_{l=1}^M\biguplus_{j=1}^{M(l, n)}C_{l,n,j} \\ 
  \end{align*}

  The hard part is to show that $\mathcal{R}(\mathcal{S})$ is a ring.
  $A = \biguplus A_j, B = \biguplus B_k$ then
  \[A \cup B = \biguplus_{j,k} \underbrace{(A_j \cap B_k)}_{\in \mathcal{S}}\]
\end{proof}

\begin{lemma}
  If $\mathcal{S}_1, \mathcal{S}_2, \dotsc, \mathcal{S}_n$ are semirings, then 
  the collection $(A_1 \times A_2 \times \dotsm \times A_n), A_i \in \mathcal{S}_i$
  form a semiring
\end{lemma}

\begin{proof}
  By induction, it suffices to check this for $n = 2$.
  \begin{itemize}
    \item $\emptyset \times \emptyset \in \mathcal{S}_1 \times \mathcal{S}_2$
    \item $A_1 \times A_2 \cap B_1 \times B_2 = (A_1 \cap B_1) \times (A_2 \cap B_2) \in \mathcal{S}_1 \times \mathcal{S}_2$
    \item \begin{align*}
      (A_1 \times A_2) \setminus (B_1 \times B_2) &= (A_1 \setminus B_1) \times A_2 \uplus (A_1 \cap B_1) \times (A_2 \setminus B_2) \\
      &= \biguplus_{k=1}^{M_1} (C_{1, k} \times A_2) \uplus \biguplus_{l=1}^{M_2} (A_1 \cap B_1) \times (C_{2, l})
    \end{align*}
  \end{itemize}
\end{proof}
Content on a semiring

\begin{definition}
  A content on a semiring $\mathcal{S}$ (ring) is 
  $\varrho : \mathcal{S} \to [0, \infty]$
  $\varphi(\emptyset) = 0$, for $A_1, A_2, \dotsc, A_n \in \mathcal{S}$, $A_1, A_2, \dotsc, A_n$ disjoint then
  $\varrho\left(\biguplus A_j\right) = \sum \varrho(A_j)$
\end{definition}
(in a ring $R$ same definition but we have $\biguplus A_j \in R$ if $A_j \in R$)

\begin{definition}
  A pre-measure on a semiring (ring) is $\nu: \mathcal{S} \to [0, \infty)$ such that
  $\nu(\emptyset) = 0$ and if $A_1, A_2, \dotsc, A_n \in \mathcal{S}$, disjoint and if 
  $\biguplus A_j \in \mathcal{S}$ then $\nu\left(\biguplus A_j\right) = \sum \nu(A_j)$
\end{definition}

\begin{example}
  $S_1$ = intervals of the form $(a, b]$ for $a, b\in \RR$
  $\varrho((a, b]) = b-a$ is a premeasure on $S$.
  First check $\varrho$ is a content on $S_1$.
  $(a, b] = \bigcup_{j=1}^M (a_j, a_{j+1}]$ ordered $a_1 < a_2 < \dotsm < a_{M+1} = b$
  $\sum a_{j+1} - a_j = b-a$
\end{example}

% \begin{example}
%   $S^2$ semiring   
% \end{example}

\begin{theorem}
  A conten on a semiring extends (uniquely) to the $\mathcal{R}(S)$. 
  \[\varrho\left(\biguplus A_j\right) = \sum \varrho(A_j)\]
\end{theorem}

\begin{proof}
  Hvae to check well-defined.  
  Let $A \in \mathcal{R}(S)$
  \[A = \biguplus_{j=1}^{M_1} A_j = \biguplus_{k=1}^{M_2} B_k \]
  Want to show that
  \[\sum_{j=1}^{M_1} \varrho(A_j) = \sum_{k=1}^{M_2} \varrho(B_k)\]
  \begin{align*}
    A_j &= \biguplus_{k=1}^{M_2} (A_j \cap B_k) \\
    B_k &= \biguplus_{j=1}^{M_1} (A_j \cap B_k) \\
    \sum_{j=1}^{M_1} \varrho(A_j) &= \sum_{j=1}^{M_2} \sum_{k=1}^{M_2} \varrho(A_j \cap B_k) \\
    \sum_{k=1}^{M_2} \varrho(B_k) &= \sum_{k=1}^{M_2} \sum_{j=1}^{M_1} \varrho(A_j \cap B_k) \\
  \end{align*}
  % Check additivity for the element 

Half oepn interval length is a premeasure.
Preliminary consideration: If $(a_j, b_j], (a, b) \subseteq \bigcup(a_j, b_j]$ then
$b-a \le \sum (b_j - a_j)$

Now $(a, b] = \biguplus_{j=1}^\infty (a_j, b_j]$, $a_j < b_j$ then 

Claim:
\[\sum_{j=1}^N (b_j - a_j) \le b-a\]
for all $N$. From the monotonicity property of a content.

Now show 
\[\sum (b_j - a_j) \ge b-a-C\eps\]
$[a+\eps, b]$ this is covered by $(a_j, b_j]$
and in fact by the open set $(a_j, b_j + \eps 2^{-j-1})$
There is a finite subcover of $[a + \eps, b]$ by $(a_{j_i}, b_{j_i} + \eps 2^{-j_i-1})$ for $i = 1, \dotsc, M$
$$\sum_{i=1}^M (b_{j_i} + \eps 2^{-j_i} - a_{j_i}) \le \sum_{j} (b_j - a_j) + \sum_{j=1}^\infty \eps2^{-j-1}$$

\end{proof}

Extend premeasures to measure on a $\sigma$-algebra.

Idea: use the outer measure $\nu^*$
\[\nu^*(E) = \inf \sum_{A_j \in S, \bigcup A_j \supseteq E} \nu(A_j)\]
use the Caratheodory construction to produce a measure $\nu^*$ on some $\sigma$-algebra $\SM^*$.

$\nu^{**}(E) = $ same except $S$ is replaced by $\mathcal{R}(S)$

\begin{claim}
  $\nu^{**}(E) = \nu^{*}(E)$
  if $\nu$ is a \underline{content}.
\end{claim}

\begin{proof}
  Show $\le$, WLOG $\nu^{**}(E) < \infty$. 
  Given $\eps > 0$ fin $A_j \in \mathcal{R}(S)$ such that $\sum \nu(A_j) < \nu^{**}(A) + \eps$
  \[A_j =\biguplus_{k=1}^{M(j)} C_{j, k}, \nu(A) = \sum_{n=1}^{M(j)} \nu(C_{j, k})\]
  then 
  \[\sum_j\sum_{k=1}^{M(j)} \nu(C_{j, k}) < \nu^{**}(A) + \eps \implies \nu^{*}(A) \le \nu^*(A) + \eps\]
\end{proof}

\begin{theorem}[Hahn-Kolomogorov-Caratheodory-Frechet] 
  \begin{enumerate}[(i)]
    \item If $\nu$ is a content on $S$ (semiring) over $X$, (and therefore on $\mathcal{R}(S)$).  
      Then the Caratheodory $\sigma$-algebra $\SM^*$ of $\nu^*$-measurable set contains $S$ {and $\mathcal{R}(S)$} and $\kM(S) \subseteq \overline{\kM}$)
    \item If $\nu$ is a premeasure then $\nu^*|_{S} = \nu$
    then $\nu^*$ is a measure (on $\SM^* \supseteq \overline{\kM(S)}$)
  \end{enumerate}
\end{theorem}

$\nu^*$-measurable if for all $E \subseteq X$, $\nu^*(E) \ge \nu^*(E \cap A) + \nu^*(E \cap A^\complement)$
(for another direction is obvious, so, we need to only check one direction)

\begin{proof}
  Show that $S \subseteq \SM^*$. Let $A \subseteq S$. 
  Fix $E \subseteq X$, work with an $\eps$-efficient cover, i.e.,
  find a collection $\{A_j\}_{j=1}^\infty$ such that $A_j \in S, \bigcup A_j \supseteq E$ : 
  $\sum \nu(A_j) \le \nu^*(E) + \eps$
  We get that $A_j = (A_j \cap A) \uplus (A_j \cap A^\complement)$
  $\nu(A_j) = \nu(A_j \cap A) + \nu(A_j \cap A^\complement)$
  \[\nu^*(E) + \eps \ge \sum \nu(A_j) = \sum \nu(A_j \cap A) + \sum \nu(A_j \cap A^\complement) \]
  We know that $E \cap A \subseteq \bigcup (A_j \cap A)$ and $E \cap A^\complement \subseteq \bigcup (A_j \cap A^\complement)$
  then 
  \begin{align*}
   \nu^*(E) + \eps \ge \nu^*(E \cap A) + \nu^{**}(E \cap A^\complement) = \nu^*(E \cap A) + \nu^*(E \cap A^\complement)
  \end{align*}

  For $A \in S$ show $\nu*(A) = \nu(A)$.

  It is easy to show that $\nu*(A) \le \nu(A)$ since $\{A\}$ is a cover of $A$.

  We need $\nu(A) \le \nu^*(A) + \eps$ (for all $\eps > 0$)

  Pick $\{A_j\}$, $\bigcup A_j \supseteq A$, $\sum \nu(A_j) < \nu^*(A) + \eps = \nu^{**}(A) + \eps$
  Define $B_1 = A_1, B_2 = A_2 \setminus A_1, B_3 = A_3 \setminus (A_1 \cup A_2), \dotsc$ $B_j$ is disjoint and $B_n \in \mathcal{R}(S)$ (but might not in the semiring)
  We claim that premeasure assumption
  We know that $\bigcup_n (A \cap B_n) = A$
  \[\nu(A) = \sum_n \nu(A\cap B_n) \le \sum_n \nu(A_n) \le \nu^{**}(E) + \eps\]
\end{proof}

\begin{example}
  $E \subseteq \NN$ 
  \[\varrho(E) = \limsup_{n\to\infty} \frac{|E \cap [1, n]|}n\]
  If $E$ is finite $\varrho(E) = 0$, if $E = \NN$ then $\varrho(E) = 1$
  But 
  \[\varrho(\NN) \not\le \sum_{n=1}^\infty \varrho(\{n\})\]
  So, Subaddictivity fails, not a premeasure
\end{example}