
\chapter{Integration}
\section{Simple Functions}

\begin{definition}
  nonnegative simple function are measurable function with finitely many values in $\RR$ (NOT on $\overline{\RR}$).
  $s: X \to \RR$, $s(x) = \sum z_j \mathbbm{1}_{x, s(z) = z_j}(x) = \sum z_j \mathbbm{1}_{f^{-1}({z_j})}$
  If values of $s$ are $\{z_1, \dotsc, z_n\}$
\end{definition}

% \pagebreak
% \begin{proof}
%   for any $v \in (\mathrm{span}(S_1) + \mathrm{span}(S_2))$, there exists $w_1 \in \mathrm{span}(S_1)$, $w_2 \in \mathrm{span}(S_2)$ such that
%   $v = w_1 + w_2$. Then, from the definiton of the span, there exists $\alpha_1, \alpha_2, \dotsc, \alpha_n \in \mathbb{F}$ and $a_1, a_2, \dotsc, a_n \in S_1$ such that 
%   \[w_1 = \sum_{i = 1}^n \alpha_ia_i\]
%   Similarly, there exists $\beta_1, \beta_2, \dotsc, \beta_m \in \mathbb{F}$ and $b_1, b_2, \dotsc, b_m \in S_2$ such that
%   \[ w_2 = \sum_{j = 1}^m \beta_jb_j\]
%   Then, since for all $a_i$, $a_i \in S_1 \cup S_2$ and for all $b_j$, $b_j \in S_1 \cup S_2$, then
%   \[v = w_1 + w_2 = \sum_{i = 1}^n \alpha_ia_i + \sum_{j = 1}^m \beta_jb_j \in \mathrm{span}(S_1 \cup S_2)\]
% \end{proof}

\begin{theorem}
  Consider nonnegative measurable function $f$.
  There exist a sequence of simple function $s_n$ such that 
  \begin{itemize}
    \item $0 \le s_n \le s_{n+1} \le f$ (i.e, $s_n(x) \le s_{n+1}(x)$)
    \item $\lim\limits_{n\to\infty}s_n(x) = f(x)$ for all $x$
    \item The convergence is uniform on all sets where $f$ is bounded.
    If $E$ is such that $|f(x)| \le M$ for all $x \in E$ then
    \[\sup_{x \in E} f(x) - s_n(x) \to 0\]
  \end{itemize}
\end{theorem}

\begin{proof}
  $s_n$ is defined so that if takes value in $[0, 2^n)$. Consider the segment $\frac{k}{2^n}$ on y-axis,
  then $$s_n(x) = \begin{cases}
    k\cdot2^{-n} & \text{if } k2^{-n} \le f(x) < (k+1)2^{-n}, 0 \le k \le 4^n - 1 \\
    2^n & \text{if } f(x) \ge 2^n
  \end{cases}$$

If $f(x) < 2^n$ then $0 \le f(x) - s_n(x) \le 2^{-n}$. We can see that $s_n(x) \le s_{n+1}(x)$ because
each step of $s_{n+1}$ is a refinement of $s_n$.
\end{proof}

We first define the integral for simple function (in analogy to the definition of Riemann-integral for stop functions)
\begin{definition}
  Define 
  $s(x) = \sum_j c_j \mathbbm{1}_{E_j}$
  where the $E_j$ are pairwise disjoint, $\biguplus E_j = X$, then
  \[\int s \ \mathrm{d}\mu = \sum_j c_j \mu(E_j)\]
\end{definition}

\begin{claim}
  $$s(x) = \sum_{j=1}^n c_j\mathbbm{1}_{E_j}(x) = \sum_{k = 1}^m d_k \mathbbm{1}_{E_k}(x)$$
  where $X = \biguplus E_j = \biguplus E_k$. If $x \in E_j \cap E_k$ then $c_j = d_k$.
\end{claim}

\begin{proof}
  We know that $\biguplus_{j, k} E_j \cap E_k = X$ and $E_j = \biguplus_{k} E_j \cap E_k$

  GOAL: $\sum_{j=1}^n c_j\mu(E_j) = \sum_{k=1}^m d_k \mu(F_k)$

  \begin{align*}
    \text{LHS}=\sum_{j=1}^n c_j \sum_{k=1}^\infty \mu(E_j \cap F_k) &= \sum_{k=1}^m \sum_{j=1}^n d_k \mu(E_j \cap E_k) \\
    &= \sum_{k=1}^m d_k \mu(F_k)
  \end{align*}
\end{proof}


\begin{lemma}
  Suppose $s, t$ are simple functions then
  \[\int (s + t)\ \mathrm{d}\mu = \int s \ \mathrm{d}\mu+ \int t\ \mathrm{d}\mu\] 
\end{lemma}

\begin{remark}
  Can shortly write 
  \[\int s + t = \int s + \int t\]
\end{remark}

\begin{proof}
  \begin{align*}
  s = \sum_{j=1}^n c_j \mathbbm{1}_{E_j} &= \sum_{j}\sum_k c_j \mathbbm{1}_{E_{j} \cap F_k} \\
  t = \sum_{k=1}^m d_k \mathbbm{1}_{F_k} &= \sum_{j}\sum_k d_k \mathbbm{1}_{E_{j} \cap F_k} \\
  s + t&= \sum_{j, k} (c_j + d_k) \mathbbm{1}_{E_j \cap F_k} \\
  \end{align*} 
  \begin{align*}
    \int s\ \mathrm{d}\mu &= \sum_{j, k}c_j \mu(E_j \cap F_k) \\
    \int t\ \mathrm{d}\mu &= \sum_{j, k}d_k \mu(E_j \cap F_k) \\
    \int (s + t)\ \mathrm{d}\mu &= \sum_{j, k}(c_j + d_k) \mu(E_j \cap F_k) \\
  \end{align*}
\end{proof}

\begin{lemma}\label{lem:measure-from-set}
  $\nu(E) = \int_E s \ \mathrm{d}\mu = \int s \mathbbm{1}_E\ \mathrm{d}\mu = \sum c_j \mu(E_j \cap E)$
  this defines a measure on $\SM$ (given $\sigma$-algebra)
\end{lemma}

\begin{proof}
  If $E^l$ is a sequence of pairwise disjoint measureable set, check
  \[\nu\left(\biguplus E^l\right) = \sum \nu(E^l)\] 
  \begin{align*}
    \nu\left(\biguplus E^l\right) &= \sum c_j \mu(E_j \cap \biguplus E^l) \\
    &= \sum_{j=1}^n c_j \sum_l \mu(E_j \cap E^l) \\
    &= \sum_l \sum_j c_j \mu(E_j \cap E^l) \\
    &= \sum_l \nu(E^l)
  \end{align*}
\end{proof}

\section{Non-negative Measurable Functions}

\begin{definition}
  For any non-negative $f$, a measurable function, define
  \[\int f \ \mathrm{d}\mu = \sup_{\underset{s \text{ simple}}{s \le f}} \int s \ \mathrm{d}\mu\]
\end{definition}

\begin{remark}
  If $0 \le f \leq g$ then $\int f \ \mathrm{d}\mu \leq \int g \ \mathrm{d}\mu$
\end{remark}

\begin{theorem}[Monotone Convergence Theorem]
  If $\{f_n\}$ is a sequence of measurable function, and $0 \le f_n \le f_{n+1}$ for all $n$. 
  (that means $f(x) = \lim\limits_{n\to\infty}f_n(x)$)Then
  \[\int f\ \mathrm{d}\mu = \lim_{n\to\infty}\int f_n\ \mathrm{d}\mu\]
\end{theorem}

\begin{proof}

  Since $f_n \le f_{n+1} \le f$ then 
  \begin{align*}
    \lim_{n\to\infty}\int f_n\ \mathrm{d}\mu \le \int f\ \mathrm{d}\mu
  \end{align*}
  % $\int f_n \le \int f \implies \lim_{n\to\infty} \int f_n \le \int f$.

  We need to show that $$\int f \ \mathrm{d}\mu\le \lim\limits_{n\to\infty} \int f_n\ \mathrm{d}\mu$$
  So, it suffices to show that for any $0 \le \underset{\text{simple}}s \le f$, that 

  $$\int s\ \mathrm{d}\mu \le \lim\limits_{n\to\infty} \int f_n\ \mathrm{d}\mu$$

  It suffices to show that for any $\eps > 0$, 
  $$(1-\eps)\int s\ \mathrm{d}\mu \le \lim_{n\to\infty} \int f_n\ \mathrm{d}\mu$$
  Define $E_n = \{x : (1 - \eps)s(x) \le f_n(x)\}$, any $x$ will be in one of the $E_n$.
  Then for any $x \in E_n$, 
  $$s(x) \le \frac{f_n(x)}{1-\eps}$$
  % $$(1-\eps) s(x)< \lim_{n \to \infty} f_n(x) = f(x)$$
  % This mean $(1-\eps)s(x) < f_n(x)$ for sufficiently large $n$ ($n \ge N(x)$).
  % We will show that $$\bigcup_{n=1}^\infty E_n = X$$
  Consider the measure defined by $$\nu(E) = \int_E s\ \mathrm{d}\mu$$ (we already show this is a measure in \ref{lem:measure-from-set}).
  We have $E_n \subseteq E_{n+1}$ and $E_n \to X$. By continuity from below~\ref{lem:continuity-from-below}, 
  $$\lim\limits_{n\to\infty}\nu(E_n) = \nu(X) = \int s\ \mathrm{d}\mu$$
  We get that
  $$\nu(E_n) = \int_{E_n} s\ \mathrm{d}\mu \le \int_{E_n} \frac{f_n(x)}{1-\eps}\ \mathrm{d}\mu \leq \int\frac{f_n(x)}{1-\eps}\ \mathrm{d}\mu = \frac1{1-\eps}\int f_n(x)\ \mathrm{d}\mu$$
  Finally, we take limit on both sides and we have 
  $$\lim_{n\to\infty}\nu(E_n) = \nu(\RR) = \int s\ \mathrm{d}\mu \le \lim_{n\to\infty} \frac1{1-\eps} \int f_n\ \mathrm{d}\mu$$
\end{proof}

\begin{lemma}
  If $f, g$ are non negative measurable function then 
  \[\int (f+g)\ \mathrm{d}\mu = \int f\ \mathrm{d}\mu + \int g\ \mathrm{d}\mu\]
\end{lemma}

\begin{proof}
  Now we have a tool \begin{itemize}
    \item Monotone Convergence Theorem
    \item Existence of $s_n\gg f, t_n \gg g$
  \end{itemize}
  \begin{align*}
    \int (s_n + t_n)\ \mathrm{d}\mu &= \int s_n\ \mathrm{d}\mu + \int t_n\ \mathrm{d}\mu \\
    \int (f + g)\ \mathrm{d}\mu &= \int f\ \mathrm{d}\mu + \int g\ \mathrm{d}\mu
  \end{align*}
\end{proof}

\begin{lemma}
$f_k \ge 0$, $f_k$ is measurable
$$\int \sum_{k=1}^\infty f_k(x)\ \mathrm{d}\mu = \sum_{k=1}^\infty\int f_k \ \mathrm{d}\mu$$
\end{lemma}

\begin{proof}
  Just apply the Monotone Convergence Theorem.
  \[s_n(x) = \sum_{k=1}^n f_k(x) \to \sum_{k=1}^\infty f_k(x)\]
\end{proof}

\begin{remark}
We cannot always interchange integrals and limits (monotonicity is key)
$f_n(x) = \frac1n\mathbbm{1}_{[0, n]}, \int f_n\ \mathrm{d}\mu = 1$ 
but $\lim\limits_{n\to\infty}f_n(x) = 0$.
\[0 = \int \lim_{n\to \infty} f_n(x)\ \mathrm{d}\mu < \lim_{n\to\infty}\int f_n\ \mathrm{d}\mu\]

Or on $[0, 1]$, $f_n(x) = n\mathbbm{1}_{[0, 1/n]}$, $\int f_n\ \mathrm{d}\mu = 1$ but $\lim\limits_{n\to\infty}f_n(x) = 0$. 
$$\lim\limits_{n\to\infty}f_n(x) = \begin{cases}
  \infty & \text{if } x = 0 \\ 
  0 & \text{if } x > 0
\end{cases}$$
\end{remark}
  
\begin{lemma}[Fatou's Lemma]
  If $\{f_j\}$ is a sequnce of measurable functions
  \[\int\liminf_{j\to\infty}f_j(x)\ \mathrm{d}\mu \le \liminf_{j\to\infty}\int f_j \ \mathrm{d}\mu\] 
  meaning 
  \[\int\lim_{k\to\infty} \underbrace{\inf_{j\ge k}f_j(x)\ \mathrm{d}\mu}_{\text{increasing on }k} \le \lim_{k\to\infty}\inf_{j\ge k}\int f_j \ \mathrm{d}\mu\] 
\end{lemma}
\begin{proof}
  \[\int\lim_{k\to\infty} \inf_{j\ge k}f_j(x)\ \mathrm{d}\mu \underset{\text{MCT}}= \lim_{k \to \infty} \int \inf_{j \ge k}f_j(x) \ \mathrm{d}\mu \]
  Take any $l \ge k$, then $\inf_{j \ge k}f_j(x) \le f_l(x)$, then for $l \ge k$
  \begin{align*}
    \int\inf_{j\ge k}f_j(x)\ \mathrm{d}\mu &\le \int f_l(x)\ \mathrm{d}\mu \\
    \int \inf_{j\ge k}f_j(x)\ \mathrm{d}\mu &\le \inf_{j\ge k}\int f_j(x)\ \mathrm{d}\mu
  \end{align*}
\end{proof}

\section{General Measurable Functions}

Integral for ``general'' measurable functions.

\begin{definition}
  Given a measurable function $f$, we define the \textbf{positive part} of $f$ as 
  \[f^+(x) = \max\{f(x), 0\}\]
  and the \textbf{negative part} of $f$ as
  \[f^-(x) = \max\{-f(x), 0\}\]
  Then we get that 
  \[f = f^+ - f^-\]
\end{definition}

\begin{definition}
  $f: X \to \RR$ (or $\overline{\RR}$)
  Suppose that $f$ is a measurable function, then we define
  \[\int f \ \mathrm{d}\mu = \int f^+\ \mathrm{d}\mu - \int f^-\ \mathrm{d}\mu\]
  provided that at least one of $\int f^\pm \ \mathrm{d}\mu$ is finite
\end{definition}
\begin{definition}
  $f: X \to \RR$ (or $\overline{\RR}$)
  $f$ is \textbf{integrable} if $\int f^+ \ \mathrm{d}\mu$, $\int f^- \ \mathrm{d}\mu$ is finite
  $\iff$ $\int |f| \ \mathrm{d}\mu$ is finite

  $\mathcal{L}^1$ is the class of integrable function
\end{definition}

\begin{definition}
$f: X \to \CC$ is measureable ($\iff$ $\Re(f)$ and $\Im(f)$ are measurable)
Assumeing that $\Re f \in \mathcal{L}^1$ and $\Im f \in \mathcal{L}^1$ then
\[\int f\ \mathrm{d}\mu = \int \Re f \ \mathrm{d}\mu + i \int \Im f \ \mathrm{d}\mu\]
\end{definition}

\begin{claim}
  Suppose that $f, g$ are measurable then
  \[\int f + g\ \mathrm{d}\mu  = \int f\ \mathrm{d}\mu + \int g\ \mathrm{d}\mu\]
  \[\int \alpha f \ \mathrm{d}\mu + \alpha \int f \ \mathrm{d}\mu\]
\end{claim}

\begin{lemma}
  $f: X \to \overline{\RR}$ is measurable, and $\int |f| \ \mathrm{d}\mu = 0$ then $f = 0$ almost everywhere.
\end{lemma}

\begin{proof}
  Define $E_n = \left\{x : |f(x)| > \frac1n\right\}$ then from continuity from below, we get that
  \[\lim_{n\to\infty}\mu(E_n) = \mu\left(\bigcup_{n=1}^\infty E_n\right)\]
  define $E = \bigcup E_n$ and we can write $E = \{x : |f(x)| > 0 \}$then we know that
  \[|f| \ge |f|\mathbbm{1}_{E_n} \ge \frac1n\mathbbm{1}_{E_n}\]
  then 
  \begin{align*}
    \int |f|\ \mathrm{d}\mu &\ge \int \frac1n\mathbbm{1}_{E_n}\ \mathrm{d}\mu \\
    &= \frac1n \mu(E_n) \\
  \end{align*}
  we get that $\mu(E_n) = 0$ for all $n$ then $\mu(E) = 0$. Therefore $f = 0$ almost everywhere.
\end{proof}

% \begin{proof}
%   Assume that $\int |f| \ \mathrm{d}\mu = 0$ Find simple function $s_n$ of $f$ then 
%   $\int s_n \ \mathrm{d}\mu = 0 \implies$ $s_n(x) = 0$ almost everywhere.
%   we claim that $s_n\to f$ everywhere then $s_n = 0$ almost everywhere then $f = 0$ alomost everywhere.
%   There is a set $N_n, \mu(N_n) = 0$ so that $s_n = 0$ in $N_n^\complement$. 
%   $N = \bigcup_{n=1}^\infty N_n \implies \mu(N) = 0$ and $N^\complement$ we have $s_n \to f$, $s_n = 0$
% \end{proof}

% We could redefine the notion of $f=0$ (being $f(x) = 0$ for all $x \in X$) to $f = 0$ almost everywhere (meaning $f(x) = 0$ except for all null set).


\begin{remark}
  $\|f\| = \int |f|\ \mathrm{d}\mu$ satisfies
  \begin{itemize}
    \item $\|f + g\| \le \|f\| + \|g\|$
    \item $\|c f\| = |c|\|f\|$
    \item $\|f\| = 0 \iff f = 0$ almost everywhere
  \end{itemize}
\end{remark}

\begin{remark}
  Almost everywhere equal is an equivalence relation.
  \[f \sim g \underset{\text{def}}\iff f(x) = g(x)\ \mu\text{-almost everywhere}\]
  $N = \{f \in\mathcal{L}^1 : f(x) = 0 \text{ almost everywhere}\}$ is a linear subspace of $\mathcal{L}^1$ vector.
  $\mathcal{L}^1/N$ is the set of equivalence classes of $\mathcal{L}^1$.
\end{remark}

$f_n \to f$ almost everywhere, $f_n \ge 0$, $f_n$ measurable, Can we define $\int f\ \mathrm{d}\mu$?
$f$ may not be measurable. This problem is fixed if $f$ we work in a complete measurable space
$(X, \SM, \mu) \to (X, \overline{\SM}, \overline{\mu})$
where 
\[\overline{\SM} = \{A \cup B : A \in \SM, B \text{ a subset of a set of measure }0\}\]

\begin{lemma}
$f \in \mathcal{L}^1$, $\int |f|\ \mathrm{d}\mu < \infty$. If $f$ is real valued $f = f^+ - f^-$,
\[\left|\int f\ \mathrm{d}\mu \right| \le \int |f| \ \mathrm{d}\mu\]
\end{lemma}
\begin{proof}
  \begin{align*}
    \left|\int f\ \mathrm{d}\mu \right| &= \left|\int f^+ - f^- \ \mathrm{d}\mu \right| \\
    &\le \left|\int f^+ \ \mathrm{d}\mu \right| + \left|\int f^- \ \mathrm{d}\mu \right| \\
    &= \int f^+ \ \mathrm{d}\mu + \int f^- \ \mathrm{d}\mu \\
    &= \int |f| \ \mathrm{d}\mu
  \end{align*}
\end{proof}

\begin{remark}
  If $f$ is complex valued, then $|f| = \sqrt{(\Re f)^2 + (\Im f)^2}$.
  Then
  \[\left|\int \Re f \right|\le \int |\Re f| \le \int |f|\]
  \[\left|\int \Im f \right|\le \int |\Im f| \le \int |f|\]
  So, 
  \[\left|\int f\ \mathrm{d}\mu \right| \le 2\int |f|\]
\end{remark}

\begin{remark}
  Estimate $\int f \ \mathrm{d}\mu = \alpha + i\beta = re^{i \phi}$, then
  $e^{-i\phi} \int f \ \mathrm{d}\mu$ is real and nonnegative.
  \begin{align*}
    \left|\int f \ \mathrm{d}\mu\right| &= \left|e^{-i\phi} \int f \ \mathrm{d}\mu\right| \\
    % &= \int e^{-i\phi}f \ \mathrm{d}\mu \\
    &= \Re \int e^{-i\phi}f \ \mathrm{d}\mu \\
    &\le \int |e^{-i\phi}f| \ \mathrm{d}\mu \\
    &= \int |f| \ \mathrm{d}\mu
  \end{align*}
\end{remark}

\begin{lemma}
$f \in \mathcal{L}^+$ means non-negative, then $\nu(E) = \int_E f \ \mathrm{d}\mu$ define measure
\end{lemma}

\begin{proof}
Check the $\sigma$-additivity
$E = \biguplus_{n=1}^\infty E_n$, 
\begin{align*}
  \nu\left(\biguplus_{n=1}^\infty E_n\right) &= \int_{\biguplus E_n} f \ \mathrm{d}\mu \\
  &= \int f \mathbbm{1}_{\biguplus E_n} \ \mathrm{d}\mu \\
  &= \int f \left(\sum_{n=1}^\infty \mathbbm{1}_{E_n}\right) \ \mathrm{d}\mu \\
  &= \sum_{n=1}^\infty \int f \mathbbm{1}_{E_n} \ \mathrm{d}\mu \\
  &= \sum_{n=1}^\infty \nu(E_n)
\end{align*}
\end{proof}

\begin{claim}
  If $f \in \mathcal{L}^1 \cap \mathcal{L}^+$ then $\nu$ is a finite measure.
\end{claim}

If $\nu(E) = int f  \ \mathrm{d}\mu$
How does $\int g \ \mathrm{d}\nu$ look like?
$\nu(E) = \int f \ \mathrm{d}\mu = \int E \ \mathrm{d}\nu$
We want ``$f\ \mathrm{d}\mu = \mathrm{d}\nu$''
\begin{lemma}
  If $f \in \mathcal{L}^+$ and $\nu(E) = \int_E f \ \mathrm{d}\mu$ then for any $g \in \mathcal{L}^+$ or $g \in \mathcal{L}^1$ then,   
  \[\int g\ \mathrm{d}\nu = \int g f \ \mathrm{d}\mu\]
\end{lemma}

\begin{proof}
  \begin{itemize}
    \item True for characteristic functions of measure set by the definition of $\nu$.
    Fix $g = \mathbbm{1}_E$ for some $E \in \SM$
    $$
      \int g\ \mathrm{d}\nu =\int \mathbbm{1}_E \ \mathrm{d}\mu= \nu(E) = \int_E f \ \mathrm{d}\mu = \int \mathbbm{1}_E f \ \mathrm{d}\mu = \int g f \ \mathrm{d}\mu
    $$
    \item By linearity of the integral, it is true for simple function.
    Fix $g = \sum_{j=1}^n c_j \mathbbm{1}_{E_j}$, then
    $$
      \int g\ \mathrm{d}\nu = \sum_{j=1}^n c_j \nu(E_j) = \sum_{j=1}^n c_j \int_{E_j} f \ \mathrm{d}\mu = \int g f \ \mathrm{d}\mu
    $$

    \item  $s_n\nearrow g$ if $g \in \mathcal{L}^+$, by Monotone convergence theorem, 
    \[\int s_n \underset{\text{MCT}}\nearrow \int g\]
    \begin{align*}
      \int s_n \ \mathrm{d}\nu &= \int s_n f \ \mathrm{d}\mu \\
      \int g \ \mathrm{d}\nu &= \int g f \ \mathrm{d}\mu
    \end{align*}
    Then extend to general g by linearity
  \end{itemize}
\end{proof}

%TODO: break the integration chapter

\begin{theorem}
  If $X$ is a finite measure space, if $f_n$ measurable, $f_n \in \mathcal{L}^1$ (integrable) and $f_n \to f$ uniformly on $X$. 
  then $$\int |f_n - f| \ \mathrm{d}\mu \to 0$$
  and 
  $$\int f_n\ \mathrm{d}\mu \to \int f\ \mathrm{d}\mu$$
\end{theorem}
  
\begin{remark}
  Uniform convergence means
  \[\lim_{n \to \infty}\sup_{x \in X} |f_n(x) - f(x)| = 0\]
\end{remark}

\begin{proof}
  We can rewrite that term as 
  \begin{align*}
    \int |f_n - f| \ \mathrm{d}\mu &\le \int \sup_{x\in X} |f_n - f|\ \mathrm{d}\mu \\ 
    &= \mu(X) \sup_{x\in X} |f_n - f| \to 0\\
  \end{align*}
  We can rewrite $f$ as $f = (f - f_n) + (f_n)$ since $f-f_n$ converge and $f_n$ integrable so $f$ must be integrable.
  \begin{align*}
    \left|\int f_n - \int f\right| &= \left|\int (f_n - f)\ \mathrm{d}\mu\right| \\
    &\le \int |f_n - f|\ \mathrm{d}\mu 
  \end{align*}
\end{proof}

\begin{definition}
  Suppose that $f_n, f$ are measurable
  $f_n \to f$ almost uniformly if for every $\eps > 0$
  there is a measurable set $E$ such that $\mu(E) < \eps$ and $f_n \to f$ uniformly on $E^\complement$
  ($\sup_{x\in E^\complement} |f_n(x) - f(x)| \to 0$)
\end{definition}

\begin{theorem}[Egorov's Theorem]
  If $\mu(X) < \infty$ and if $f_n \to f$ almost everywhere then 
  $f_n \to f$ almost uniformly
\end{theorem}

\begin{remark}
  $f_n(x) \to f(x)$ if for every $k$ there exists $n = n(k)$ such that
  $|f_m(x) - f(x)| < \frac1k$ for all $m \ge n(k)$
\end{remark}

\begin{proof}
  Fix $\eps > 0$, define 
  \begin{align*}
    E_n(k) &:= \left\{x : |f_m(x) - f(x)| \ge \frac1k \text{ for some }m \ge n\right\} \\
    &= \bigcup_{m \ge n} \left\{x : |f_m(x) - f(x)| \ge \frac1k\right\}
  \end{align*}
  (Given $x$ For sufficiently large $n$, $x \notin E_n(k)$), $E_n(k) \supseteq E_{n+1}(k)$
  $\bigcap_n E_n(k) = \emptyset$ because of $f_n \to f$ everywhere.
  Form the continuity from above~\ref{lem:continuity-from-above}, we get that $\lim_{n\to\infty}\mu(E_n(k)) = 0$.
  Find $n(k)$ such that $\mu(E_{n(k)}(k)) < \frac{\eps}{2^k}$, then $E = \bigcup_k E_{n(k)}(k)$ has measure $< \eps$.
  
  For $x\in \left(\bigcup_k E_{n(k)}(k)\right)^\complement = \bigcap_k E_{n(k)}(k)^\complement$
  I have for all $k$ $|f_m(x) - f(x)| < \frac1k$ for all $m \ge n(k)$.
  So, we get $f_n \to f$ uniformly on $E^\complement$.
\end{proof}

\begin{theorem}[Baby Dominated Convergence Theorem]
  Given $(X, \SM, \mu)$ where $\mu$ is a finite measure ($\mu(X) < \infty$).
  Let $\{f_n\}$ be measurable functions, $f_n \to f$ everywhere.
  \[|f_n| \le C \implies \int |f_n - f| \ \mathrm{d}\mu \to 0\]
  i.e. $f_n$ converges with respect to $L^1$-(semi-)norm.
\end{theorem}

\begin{corollary}
  \[\int_X f_n\ \mathrm{d}\mu \to \int_X f \ \mathrm{d}\mu\]
\end{corollary}

\begin{proof}
  Tools:
  \begin{enumerate}[(i)]
    \item If $f_n \to f$ uniformly then $\int |f_n-f| \ \mathrm{d}\mu \to 0$
    \item Egorov's Theorem
  \end{enumerate}
  $|f(x)| \le C$, $f$ is measurable.
  Given any $\eps > 0$, By Egorov's Theorem, find a set of measure $E$ that $\mu(E) < \frac{\eps}{4C}$ 
  such that $f_n \to f$ uniformly on $E^\complement$.
  Then
  \begin{align*}
    \int |f_n - f| \ \mathrm{d}\mu &\le \int_E |f_n - f| \ \mathrm{d}\mu + \int_{E^\complement} |f_n - f| \ \mathrm{d}\mu  \\
  \end{align*}
  we know that $|f_n - f| \le |f_n| + |f| \le 2C$ then
  \begin{align*}
    \int |f_n - f| \ \mathrm{d}\mu &\le 2C\mu(E) + \int_{E^\complement} |f_n - f| \ \mathrm{d}\mu  \\
    &\le \frac{\eps}{2} + \int_{E^\complement} |f_n - f| \ \mathrm{d}\mu  \\ 
  \end{align*}
  so for large $n$, the second term will be $< \frac{\eps}{2}$.
\end{proof}

\begin{theorem}[Dominated Convergence Theorem]
  Given $(X, \SM, \mu)$ where $\mu$ is a finite measure ($\mu(X) < \infty$).
  Let $\{f_n\}$ be measurable functions, $f_n \to f$ almost everywhere.
  \[\sup_n|f_n| \in \mathcal{L}^1 \implies \int |f_n - f| \ \mathrm{d}\mu \to 0\]
\end{theorem}

\begin{proof}
  Define $g(x) = \sup_n |f_n(x)|$ 
  The trick is $$|f_n - f| =\begin{cases}
     \frac{|f_n-f|}{g}g & \text{if } g > 0 \\
     0 & \text{if } g = 0
  \end{cases}$$
  define a new measure $\nu(E) = \int_E g \ \mathrm{d}\mu$.
  Then $\nu$ is a finite measure, and 
  \begin{align*}
    g\ \mathrm{d}\mu &= \mathrm{d}\nu \\
    \int h \ \mathrm{d}\nu &= \int hg \ \mathrm{d}\mu
  \end{align*}
  then define
  $$h_n = \begin{cases}
    \frac{|f_n - f|}{g} & \text{if } g > 0 \implies |h_n(x)| \le 2\\
    0 & \text{if } g = 0 \implies h_n(x) \to 0
  \end{cases}$$
  Then\begin{align*}
    \int |f_n - f| \ \mathrm{d}\mu &= \int h_n g \ \mathrm{d}\mu \\
    &= \int h_n \ \mathrm{d}\nu \to 0
  \end{align*}
  By Baby Dominated Convergence Theorem
\end{proof}

\section{Integration from Riemann to Lebesgue}

\begin{theorem}
  If $f$ is Riemann integrable on $[a, b]$ then $f$ is Lebesgue integrable.
  \[\int_a^b f(x)\ \mathrm{d}x = \int_{[a, b]}f \ \mathrm{d}\mu\]
  where $\mu$ is Lebesgue measure.
\end{theorem}

\begin{proof}
  Define
  \[U_Pf(x) = \begin{cases}
    M_j & \text{if } x \in [x_{j-1}, x_j) \\
    M_n & \text{if } x \in [x_{n-1}, x_n]
  \end{cases}\]
  Similarly for the lower sum $L_Pf(x)$. 
  If $P'$ is a refinement of $P$ then $U_{P'}f(x) \le U_Pf(x)$ and $L_{P'}f(x) \ge L_Pf(x)$.
  Since $f$ is Riemann integrable, then
  \[\inf_P U(f, P) =: \overline{\mathcal{I}}_a^b(f) = \underline{\mathcal{I}}_a^b(f) := \sup_P L(f, P)\]
  Choose a sequence of partitions $P_n$ such that
  \begin{align*}
    \int U_{P_n}f \to \overline{\mathcal{I}}_a^b(f) \\
    \int L_{P_n}f \to \underline{\mathcal{I}}_a^b(f)
  \end{align*}
  Since that $P_{n+1}$ is a refinement of the $P_n$ then $U_{P_n}f \searrow U(x)$ and 
  $L_{P_n}f \nearrow L(x)$ and $L(x) = U(x)$.
  Notice that from Riemann integrable, $|f| < C$, then
  \begin{align*}
    \int_{[a, b]} U_{P_n}f &\to \overline{\mathcal{I}}_a^b(f) = \int_{[a, b]} U(x)\ \mathrm{d}m\\
    \int_{[a, b]} L_{P_n}f &\to \underline{\mathcal{I}}_a^b(f) =\int_{[a, b]} L(x)\ \mathrm{d}m
  \end{align*}
  If $f$ is Riemann integrable, 
  \[\int U \ \mathrm{d}m = \int L \ \mathrm{d}m = \int_a^b f(x) \ \mathrm{d}x\]
  and $U \ge L$ then
  \[\int (U-L)\ \mathrm{d}m = 0 \implies U(x) = L(x)\]
  almose everywhere, $L(x) \le f(x) \le U(x) \implies f = L$ almost everywhere and $f = U$ almost everywhere.
  Then $f$ is Lebesgue integrable and
  \[\int f \ \mathrm{d}m = \int L \ \mathrm{d}m = \int U \ \mathrm{d}m\]
\end{proof}

\begin{definition}[Improper Riemann integrals]
  \[\int_0^\infty f(x)\ \mathrm{d}x, \int_1^\infty f(x)\ \mathrm{d}x, \int_0^1 f(x)\ \mathrm{d}x\]
  if $f$ is not Riemann-integrable on the domain but on every compact subinterval.
  We can define as
  \[\int_1^\infty f(x)\ \mathrm{d}x = \lim_{R\to\infty}\int_1^Rf(x) \ \mathrm{d}x\]

\end{definition}
\begin{example}
  \[\int_1^\infty \frac{\sin x}{x}\ \mathrm{d}x = \lim_{R\to\infty}\int_1^R\frac{\sin x}x \ \mathrm{d}x\]
  $I_k = [2k\pi + \frac\pi4, 2k\pi + \frac{3\pi}4]$, $\sin x \ge \frac1{\sqrt2}$, so, $\frac{\sin x}{x} \ge \frac1{x\sqrt2}\cdot \frac{1}{2k\pi + \frac{3\pi}4}$.
  We can do integration by parts
  $$
    \int_1^R \frac{\sin x}x \ \mathrm{d}x = \left.-\frac{\cos x}x\right|_1^R - \int_1^R -\frac{\cos x}{x^2} \ \mathrm{d}x \\
  $$
\end{example}

\begin{example}
  \[\int_0^\infty \sin(x^2)\ \mathrm{d}x\]
  consider $\sin(x^2)$
  \[
    \sqrt{2k\pi + \frac\pi2} \le \sqrt{x^2} \le \sqrt{2k\pi + \frac{3\pi}4} 
  \]
  \[\sqrt{2k\pi + \frac{3\pi}4} - \sqrt{2k\pi + \frac{\pi}4} \approx \frac1{\sqrt{k}}\]
\end{example}

\begin{lemma}
  Suppose that if $\int_1^\infty |f(x)|\ \mathrm{d}x < \infty$ then $f\in \mathcal{L}^1$.
\end{lemma}
\begin{proof}
  \begin{align*}
    \int_1^\infty |f(x)|\ \mathrm{d}x &= \int_1^\infty \lim_{n\to\infty}|f(x)|\mathbbm{1}_{[1, n]}(x)\ \mathrm{d}x \\
    &= \lim_{n\to\infty}\int_1^n |f(x)|\ \mathrm{d}x
  \end{align*}
\end{proof}

\begin{theorem}
  If $f$ is integrable on $\RR$, $$\int_\RR |f(x)|\ \mathrm{d}x < \infty$$ $f \in \mathcal{L}^1$ 
  then for every $\eps > 0$, there is a continuous function ($C^\infty$) $g$, vanishes off a compact set, 
  \[\int |f - g| \ \mathrm{d}m < \eps\]
\end{theorem}

\section{Introduction to Outer Measures}

\begin{definition}
  In our axiometic theorem on the Lebesgue measure, $m(I) = \ell(I)$, $m((a, b]) = b-a$
  and for a general Borel set on $\RR$, $m$ is given by the \textbf{outer measure} induced by the collection of intervals
  \[\varrho(E) = \inf \sum_{k=1}^\infty \ell(I_k)\]
  where the $\inf$ is taken over collections $\{I_k\}$, such that $E \subseteq \bigcup_{k=1}^\infty I_k$
\end{definition}

\begin{remark}
  $\widetilde{\varrho}$ defined similarly but we only admit open intervals in the infimum.
  Obviously, $\widetilde{\varrho}(E) \ge \varrho(E)$
  Need to show that $\widetilde{\varrho}(E)\le \varrho(E)$
  we may assume that $\varrho(E) < \infty$, show $\widetilde{\varrho}(E) \le \varrho(E) + \eps$.
  There is a collection of intervals $I_k$ such that
  \[\sum_k \ell(I_n) < \varrho(E) + \frac\eps2\]
  If $I_k = [a_k, b_k]$, then define $J_k = (a_k - \frac{\eps}{2^{k+2}}, b_k + \frac{\eps}{2^{k+2}})$
  Then $\ell(J_k) = \ell(I_k) + \frac{\eps}{2^{k+1}}$ then
  \begin{align*}
    \widetilde{\varrho}(E) \le \sum_{k=1}^\infty \ell(J_k) &\le \sum_{k=1}^\infty \ell(J_k) + \eps2^{-k-1} \\
    &\le \varrho(E) + \frac\eps2 + \frac\eps2
  \end{align*}

\end{remark}



\begin{lemma}
  $m(E) = \sup\{m(K) : K \subseteq E, K \text{ compact}\}$
\end{lemma}

\begin{proof}
  Case where $E = \overline{E}$ and $E$ is bounded, there is nothing to show

  Assume that $E$ is bounded, GOAL: find $K \subseteq E$ such that $m(E\setminus K) < \eps$.
  Consider $\overline{E} \setminus E$, find $O \supseteq \overline{E} \setminus E$, $m(O\setminus (\overline{E} \setminus E)) < \eps$.
  then $O^\complement \cap \overline{E} \subseteq E$ because if $x \in O^\complement$, either $x \in \overline{E}$ or $x \in E$.
  $E \setminus K = E \cap K^\complement \subseteq O \cup \overline{E}^\complement$.
  Since $E \subseteq \overline{E}$ then $E\setminus K \subseteq O$ and $E\setminus K \subseteq O \setminus (\overline{E} \setminus{E})$ 
  has measure $< \eps$.
\end{proof}
\begin{theorem}
  For every Borel set $E$, $m(E) < \infty$, there is an open set $O\supseteq E$ such that $m(O\setminus E) < \eps$.
  where $m(E) = \inf \sum \ell(I_n)$ where $\inf$ take over $I_k$, $I_k$ are open, $E \subseteq \bigcup I_k$
\end{theorem}

\begin{proof}
  Define $E_n = E \cap \overline{B}(0, n)$. Find compact set $K_n \subseteq E_n \setminus E_{n-1}$ then
  $m((E_n \setminus E_{n-1})\setminus K_n) < \eps 2^{-n-1}$.
  The set $H_l = K_1 \cup \cdots \cup K_l$ is compact and increasing, $H_l \subseteq E_l$ and 
  \[m(E_l) - \eps \le m(H_l) \le m(E_l) \to m(E)\]
  % $m(E_l \setminus H_l) < \eps$.
  % Then $H = \bigcup H_l$ is compact and $H \subseteq E$ and $m(E\setminus H) < \eps$.
\end{proof}

\begin{theorem}
  Given an open set $O$, we can decompose $O$ as a disjoint union of ``dyadic cubes''
\end{theorem}

\begin{theorem}
  We can choose the cubes a dyadic cubes such that if $O \neq \RR^n$ such that 
  \[\mathrm{diam}(Q)<\mathrm{dist}(Q, O^C) \le 4 \mathrm{diam}(Q)\]
\end{theorem}

\begin{remark}
  If side length of $Q$ is $2^{-k}$ then the diameter is $\sqrt{n}2^{-k}$.
\end{remark}

\begin{theorem}[Whitney decomposition theorem]
  Given $\Omega$ open set in $\RR^n$, $\Omega \neq \RR^n$,
  there is a family $\SF$ of dyadic cubes such that
  \begin{itemize}
    \item they are disjoint
    \item $\biguplus_{Q \in \SF}Q = \Omega$
    \item For every $Q \in \SF$, $C\mathrm{diam}(Q) < \mathrm{dist}(Q, \Omega^C) \le (2C+2)\mathrm{diam}(Q)$
  \end{itemize}  
\end{theorem}

\begin{proof}
  Define for each $k$ a family of dyadic cubes $\SF_k$ of side length $2^{-k}$ 
  (i.e., the diameter is $\sqrt{n}2^{-k}$)
  intersecting the region 
  $$\Omega_k = \{x : A2^{-k}\sqrt{n} \le \mathrm{dist}(x, \Omega^\complement) \le 2A \cdot 2^{-k}\sqrt{n}\}$$
  Pick a cube in $\SF_k$, $Q_1$ it contains an $x_Q \in \Omega_k$
  $$\mathrm{dist}(Q, \Omega^\complement) \le \mathrm{dist}(x_Q, \Omega^\complement)\le 2A \cdot 2^{-k} \sqrt{n} - 2A \mathrm{diam}(Q)$$
  \begin{align*}
    \mathrm{dist}(Q, \Omega^\complement) &\ge \mathrm{dist}(x_Q, \Omega^\complement) - \mathrm{diam}(Q) \\
    &\ge A2^{-k}\sqrt{n} - 2A \mathrm{diam}(Q) \\
    &= (A-1)\cdot \mathrm{diam}(Q)
  \end{align*}
  Then $\SF_T = \bigcup \SF_k$ and finally $\SF = $ collection of all maximal (with respect to inclusion) cubes in $\SF_T$.
  Fix $Q, Q'$ and assume that $Q \subseteq Q'$ then
  $$(A-1)\mathrm{diam}Q' \le \mathrm{dist}(Q', \Omega^\complement) \le \mathrm{dist}(Q, \Omega^\complement)\le 2A\cdot \mathrm{diam}(Q)$$
  then $\mathrm{diam}(Q') \le \frac{2A}{A-1}\mathrm{diam}(Q)$
\end{proof}

we know that for every $\eps_1 > 0$ we can find a simple function $s$ such that $\int |f - s| \ \mathrm{d}m < \eps_1$.
(For non-negative $f$ use MCT $s_n \nearrow f$ and $s_n \le f$ so $\int s_n \nearrow \int f\implies \int f - s_n \to 0$ and then we use $f = f_+ - f_-$)

$s = \sum c_j \mathbbm{1}_{E_j}$, $E_j \subseteq O_j$, $m(O_j \setminus E_j)<\eps$ 
$\widetilde{s} = \sum c_j  \mathbbm{1}_{O_j}$
\begin{align*}
  \int \widetilde{s} - s \ \mathrm{d}m &= \left|\int \sum c_j \mathbbm{1}_{O_j}-\mathbbm{1}_{E_j} \ \mathrm{d}m\right| \\
  &\leq \sum_{j=1}^N |c_j||m(O_j \setminus E_j)| \\
  &\leq \eps_2
\end{align*}

Then $\mathbbm{1}_{O_j} = \sum_\nu \mathbbm{1}_{Q_{\nu}}$ where $\{Q_{\nu}\}$ are the Whitney cubes in Whitney Theorem.
$|O_j| = \sum_{\nu \in I} m(Q_{\nu})$
There is a finite $\widetilde{I}_j$ such that 
\[\int \left|\mathbbm{1}_{O_j} - \sum_{\nu \in \widetilde{I}_j} \mathbbm{1}_{Q_\nu}\right| < \eps_3\]

replace $\sum_{j=1}^N c_j \mathbbm{1}_{O_j}$ by $\sum_{j=1}^N c_j \mathbbm{1}_{\bigcup_{\nu \in \widetilde{I}_j}Q_\nu }$

then \[\mathbbm{1}_{Q_\nu}(x_1, \dotsc, x_n) = \prod_{i=1}^n \mathbbm{I_{\nu, i}}(x_i)\]

\begin{lemma}
  For any $f \in L^1$ there exists $s$ a step function such that $\int |f-s| \ \mathrm{d}m < \eps$
\end{lemma}

\begin{proof}
  Suppose that $f \in L^1$ I want to show that there exists $s$ a step function such that $\int |f-s| \ \mathrm{d}m < \eps$ 
  for any $\eps >0$. Since $f = f^+ - f^-$, WLOG, 
  $f \ge 0$ (otherwise we can do each positive and negative part and do the sum of both step functions with $\frac\eps2$ bound). 
  Given any $\eps > 0$, there exists $s'$ a simple function such that $\int |f-s'|\ \mathrm{d}m < \frac\eps2$. Then we can write $s' = \sum_{j=1}^N c_j \mathbbm{1}_{E_j}$, then there exists $O_j$ open set such that $E_j \subseteq O_j$ and $m(O_j\setminus E_j) < \frac\eps{4|c_j|N}$. Since $O_j$ is an open set, then $O_j = \bigcup (a_i, b_i)$ then define $K_n = \bigcup_{i=1}^n (a_i, b_i)$ from continuity from below, there exists $n'$ such that $m(K_{n'}) > m(O_j) - \frac\eps{4|c_j|N}$ and $K_{n'}$ contain finite interval, then we define $O_j' := K_{n'}$. Define $s = \sum_{j=1}^N c_j\mathbbm{1}_{O_j'}$ then
    \begin{align*}
        \int |f-s|\ \mathrm{d}m &\le \int |f-s'| \ \mathrm{d}m + \int |s'-s| \ \mathrm{d}m \\
        &\le \frac\eps2 + \int \left|\sum_{j=1}^N c_j(\mathbbm{1}_{E_j} - \mathbbm{1}_{O_j'})\right|\ \mathrm{d}m \\
        &\le \frac\eps2 + \sum_{j=1}^N |c_j| \int \left|\mathbbm{1}_{E_j} - \mathbbm{1}_{O_j'}\right|\ \mathrm{d}m \\
        &= \frac\eps2 + \sum_{j=1}^N |c_j| (m(E_j \setminus O_j') + m(O_j' \setminus E_j))\\
        &\le \frac\eps2 + \sum_{j=1}^N |c_j| (m(O_j \setminus O_j') + m(O_j \setminus E_j))\\
        &\le \frac\eps2 + \sum_{j=1}^N |c_j| \left(\frac\eps{4|c_j|N} +\frac\eps{4|c_j|N}\right)\\
        &< \eps
    \end{align*}
\end{proof}
